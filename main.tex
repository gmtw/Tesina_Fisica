\documentclass[12pt,a4paper]{book}
\usepackage[utf8]{inputenc}
\usepackage[spanish]{babel}
\usepackage{amsmath}
\usepackage{amsfonts}
\usepackage{amssymb}
\usepackage{graphicx}
\usepackage{fourier}
\usepackage[left=2cm,right=2cm,top=2cm,bottom=2cm]{geometry}
\usepackage{hyperref} %Paquete para agregar hipervínculos

\usepackage{subfigure} % subfiguras
\usepackage{float} %fijar figuras
\usepackage{listings} %este paquete esta agregado para que se pueda agregar lineas de código de programa (en este caso FORTRAN) a nuestro documento

\spanishdecimal{.}

%\usepackage{cancel} %cancelar términos
%\usepackage[backend=biber]{biblatex}
%\addbibresource{biblio.bib}

\providecommand{\abs}[1]{\lvert#1\rvert} %agregación para el valor absoluto
\usepackage{xcolor}
\usepackage{graphicx}
%\usepackage{subfigure} % subfiguras
\lstset{language=Fortran, %nuestro lenguaje fortran por su pollo
backgroundcolor=\color{white}, %color de fondo blanco para eso es que ocupe el paquete color
basicstyle=\footnotesize, % Fija el tamaño del tipo de letra utilizado para el código
breakatwhitespace=false,% Activarlo para que los saltos automáticos solo se apliquen en los espacios en blanco
breaklines=true,
commentstyle=\color{blue}, %color de los comentarios del código
keepspaces=true, 
rulecolor=\color{black}, % Si no se activa, el color del marco puede cambiar en los saltos de línea entre textos que sea de otro color
keywordstyle=\color{red},% estilo de las palabras clave
stringstyle=\color{yellow},% Estilo de las cadenas de texto
stringstyle=\color{orange}, 
}
\lstset{numbers=left, numberstyle=\tiny, stepnumber=1, numbersep=-2pt}

\author{Julio César Sosa Mondragón}

\begin{document}

\begin{minipage}{.3\textwidth}
  
  \flushleft
  \center{\includegraphics[scale=.09]{unam.pdf}}

  \vspace{20pt}

  \center{
    \rule{.5pt}{.6\textheight}
    \hspace{7pt}
    \rule{2pt}{.6\textheight}
    \hspace{7pt}
    \rule{.5pt}{.6\textheight}
         } \\

\center{\includegraphics[scale=.22]{ciencias.pdf}}
\end{minipage}
\begin{minipage}{.7\textwidth}

\flushright

\center{

  \center{
    \LARGE{U}\large{NIVERSIDAD} \LARGE{N}\large{ACIONAL} 
    \LARGE{A}\large{UTÓNOMA} \\[10pt]
    \large{DE} 
    \LARGE{M}\large{ÉXICO} 
  } \\
  \rule{\textwidth}{2pt}
  \\
  \hrulefill\\[1cm]
  
  \LARGE{F}\large{ACULTAD DE } \LARGE{C}\large{IENCIAS}\\[2cm]

  \large{
Simulaciones numéricas hidrodinámicas relativistas de Destellos de Rayos Gamma cortos lanzados desde un objeto compacto en varios tipos de medios ambientes. }\\[2cm]

  \huge{
T \hspace{1cm} E \hspace{1cm} S \hspace{1cm} I \hspace{1cm} S  }\\[1cm]

  \large{QUE PARA OBTENER EL TÍTULO DE:}\\[1cm]

  \large{
Físico  }\\[1cm]

  \large{PRESENTA:}\\[1cm]

  \large{
Julio César Sosa Mondragón  }\\[1cm]

  \large{
TUTOR  }\\[1cm]

  \large{
Dr. Diego López Cámara Ramírez}
}

\end{minipage}
\thispagestyle{empty} % para que no se numere esta pagina


\pagenumbering{Roman} % para comenzar la numeracion de paginas en numeros romanos
\tableofcontents % indice de contenidos

%
%%\cleardoublepage
%%\addcontentsline{toc}{chapter}{Lista de figuras} % para que aparezca en el indice de contenidos
%%\listoffigures % indice de figuras
%
%%\cleardoublepage
%%\addcontentsline{toc}{chapter}{Lista de tablas} % para que aparezca en el indice de contenidos
%%\listoftables % indice de tablas
%
%%%%%%%%%%%%%% DEDICATORIA %%%%%%%%%%%%%%%%%%%%%%%%%%%%%%%%%%%
\chapter*{Dedicatoria}

\addcontentsline{toc}{chapter}{Dedicatoria}
%
\begin{flushright}
\textit{Dedicado a \\
mi familia}
\end{flushright}

%%%%%%%%%%%%%%%%%%%%%%%%%%%%%%%%%%%%%%%%%%%%%%%%%%%%%%%%%%%%%%%%%%
%
\chapter*{Agradecimientos} % si no queremos que añada la palabra "Capitulo"
\addcontentsline{toc}{chapter}{Agradecimientos} % si queremos que aparezca en el índice
\markboth{AGRADECIMIENTOS}{AGRADECIMIENTOS} % encabezado
 
¡Muchas gracias a todos!
%%%%%%%%%%%%%%%%%%%%%%%%%%%%%%%%%%%%%%%%%%%%%%%%%%%%%%%%%%%%%%%%%%%%%

\chapter*{Resumen} % si no queremos que añada la palabra "Capitulo"
\addcontentsline{toc}{chapter}{Resumen} % si queremos que aparezca en el índice

\markboth{RESUMEN}{RESUMEN} % encabezado
%
Una bonita historia
%
%%%%%%%%%%%%%%%  CAPITULOS %%%%%%%%%%%%%%%%%%%%%%%%%%%%%%%%%%55
%
%

\chapter{Introducción}

\pagenumbering{arabic} % para empezar la numeración con números
%Érase una vez...

Los destellos de rayos gamma (GRB por su acrónimo en inglés) son eyecciones de rayos gamma del orden de Mev, son cortos, intensos y no repetitivos, fueron descubiertos por los satélites \emph{Vela}\footnote{\url{https://heasarc.gsfc.nasa.gov/docs/heasarc/missions/vela5a.html}} a finales de la decada de los 60's, gracias a los datos recabados por el experimento BATSE\footnote{\url{https://gammaray.msfc.nasa.gov/batse/}} se logró identificar 2 grupos principales de GRBs, los cortos y los largos, en los cuales este último tiene una duracion mayor a los 2 s.

\begin{figure} %fuente de la imagen: https://imagine.gsfc.nasa.gov/science/objects/bursts1.html
  \centering
    \includegraphics[width=0.5\textwidth]{Figuras/burst_durations_labelled.jpg}
  \caption{Gráfico que muestra la diferencia de tiempo que existe entre GRBs cortos y GRBs largos. Se pueden apreciar 2 picos que marcan la diferencia de duracion entre los GRBs}
  \label{fig:Batse_duration_GRBs}
\end{figure}

El objetivo principal del estudio de los GRBs es estudiar las propiedades de los estallidos de los rayos gamma, así como sus progenitores y determinar las propiedades básicas de estos, la fusión de objetos compactos es el modelo progenitor más atractivo. La interacción de flujos de salida relativistas con el medio ambiente que lo rodea produce emisión sincrotrónica que va desde la banda de las ondas de radio a los rayos X, los GRBs son uno de los eventos que más libera energía en el universo, la energía liberada es del orden de $10^{52}$ ergs.

Aunque hay mucha información acerca de los GRBs largos, los cortos son aún un estudio nuevo en el área de astrofísica de altas energías ya que debido a su corta duración son difíciles de estudiar. 
\section{Características de los GRBs}
Muchas de las características de los GRBS ya sean estos largos o cortos, son la duración media que tienen, mientras que en los GRBs largos tienen un promedio de duración de 100 s, los GRBs cortos duran menos de 2 segundos. Los estallidos de Rayos gamma se encuentran a miles de años luz de nuestra galaxia por lo que se puede decir que tienen un orígen cosmologico, mientras los GRBs largos se originan en los centros de las galaxias, los cortos lo hacen lejos de la galaxia donde se originaron y por consecuencia, estos tienen una densidad de ambiente muy bajo. El estudio de los SGRB ha sido muy difícil debido al tiempo de duración que conllevan estos eventos y a la dificultad de asociarlo con galaxias anfitrionas. 

%Las partes principales de los GRBs son la emisión propmt y  afterglow. %Piran2005
 
 
 \begin{figure} %fuente de la imagen:https://astrobites.org/2017/11/13/grb-afterglows-coming-out-of-a-cocoon/
  \centering
    \includegraphics[width=0.5\textwidth]{Figuras/Gamma-ray_burst_by_a_blackhole-768x432.jpg}
  \caption{Diferentes partes de un GRB en general, mostrando las partes más características como el afterflow y la emisión prompt}
  \label{fig:Partes de GRBs}
\end{figure}
 

\subsection{Emisión pronta}
La emisión pronta es definido como el periodo de tiempo donde el detector de rayos $\gamma$ detecta una señal sobre el fondo, mientras que la emisión afterglow es despues de la emisión prompt en el cual se pueden detectar otras longitudes de onda como el óptico, radio y los rayos X \ref{fig:Partes de GRBs}.
\subsection{Emisión tardía}

 En Mayo y Julio del 2005 datos de eyecciones recopiladas por el satélite \emph{swift} al seguir al seguir al GRB 050509B, descubrieron las primeras fases tardías de los GRBs cortos, tiempo después el satélite HETE-2 junto con el observatorio \emph{chandra} de rayos X siguieron al satélite 050709 el cual localizo la fase tardía en rayos X y despues en la banda óptica, con estos datos recabados de la fase tardía se pudo llegar a que los GRBs cortos tienen una escala de densidad y una energía mas baja que los GRBs largos, tambien se llego a la conclusión de que los GRBs cortos tienen orígenes cosmologicos y que las estrellas masivas no son sus progenitores como en el caso de GRBs cortos.
\subsection{Energía y luminosidad}

%
\section{Características de los GRBs cortos}
Los progenitores de los SGRBs tienen una amplia distribución de tiempo de retraso y su taza es influenciada en parte por la actividad de formación de las estrellas. Las patadas natales pueden ser las causantes de las distancias en las que estos nacen y los sitios de explosión de estos sistemas, la probabilidad para una galaxia de brillo $m$  a ser localizada a una separación $\delta R$ de la posición de un SGRB está dada por 

\begin{equation}
P_{cc} = 1 - \exp ^{- \pi (\delta R)^{2} \sum(\leq m)}
\end{equation}

Donde $\sum(\leq m) = 1.3\cdot 10^{0.33(m-24)-2.44 arcsec ^{-2}}$ son el número de cuentas de la galaxia. Los SGRBs sin galaxias anfitrionas se exhiben cerca galaxias de campo con una baja probabilidad de coincidencia, generalmente las distancias medias de separación del SGRB con su galaxia anfitriona es de $ \frac{\delta R}{r_e} \approx 1.5$ donde $r_e$ es el radio estelar. Otra característica que distingue también a las SGRB es que sus anfitriones tienden a ser mas largos que los anfitriones de los largos.

Los parámetros claves de interes de la emisión pronta y tardia:
\begin{itemize}
\item $E_{\gamma}$: Escala de energía
\item $E_k$: Energía cinética de la onda de choque de la emisión tardía
\item $\theta_j$: El ángulo de apertura del jet
\item $n$: Densidad del medio ambiente
\end{itemize}

El espectro de emisión sincrotrónica (flujo relativista interactuando con el medio circundante) se caracteríza por 3 frecuencias:
\begin{itemize}
\item $\nu_a$: Auto absorción
\item $\nu_m$: Factor mínimo de Lorentz
\item $\nu_c$: enfriamiento sincrotrónico
\end{itemize}

La mayoría de estas frecuencias se encuentran entre 1 $\thicksim$ 10 GHz por lo cual ha sido difícil de detectarlas. La colimación del jet también influye en la taza de los SGRBs y proporciona una restricción adicional al modelo progenitor, la firmatura de la colimación de los destellos de los SGRBs son los llamados "Jet Break" que ocurren al tiempo $t_j$ cuando $\Gamma_j(t_j) = 1/ \theta_j$, esto lidera al cambio de emisión del espectro sincrotrónico

\begin{equation}
F_\nu \propto t^{-1} \longrightarrow F_\nu \propto t^{-p}
\end{equation}

La relación entre el "jet break" y el ángulo de apertura esta dado por:

\begin{equation}
\theta_j = 0.13 \left( \frac{t_{j,d}}{1+z} \right)^{3/8}
\left(\frac{n_0}{E_{52}} \right)^{1/8}
\end{equation}

\section{GRB del 17 de agosto del 2017}
%
\chapter{Teoría}
Para describir un sistema de partículas como un fluido bajo ciertas condiciones uno debe de conocer que el camino libre medio debe de ser mucho mas pequeño que la escala de longitud de las fluctuaciones de las variables macroscópicas.

\begin{equation}
\lambda_{mfp} \ll L
\end{equation}

El tiempo entre las colisiones debe de ser pequeña comparada con la escala del tiempo de los cambios en el fluido
\begin{equation}
t_{c} \ll t_f
\end{equation}
La distancia media entre las partículas tiene que ser mas pequeña que la longitud de escala de las variables macroscópicas

\begin{equation}
l = n^{-1/3} \ll L
\end{equation}



\section{Ecuaciones de la hidrodinámica}

Considerando una serie de elementos de volumen fijos, las ecuaciones que describen el movimiento de un fluido sin considerar efectos viscosos son:

La conservación de masa
\begin{equation} \label{conservación_masa_hidrodinamica}
\dfrac{\partial \rho }{\partial t} + \nabla \cdot \left( \rho \mathbf{u} \right)
\end{equation}

El momento
\begin{equation}  \label{conservacion_momento_hidrodinamica}
\dfrac{\partial \left( \rho \mathbf{u} \right) }{\partial t}+ \nabla \cdot \left( \rho \mathbf{u u} \right) + \nabla p = \mathbf{f_{ext}}
\end{equation}

Ecuación de la energía

\begin{equation} \label{conservacion_energia_hidrodinamica}
\dfrac{\partial E }{\partial t} + \nabla \cdot \left[ \mathbf{u} \left( E+p \right) \right] =G-L+\mathbf{f_{ext} \cdot \mathbf{u}}
\end{equation}

Ecuación de estado
\begin{equation}
E=\frac{1}{2} \rho \mathbf{u}^{2} + \frac{p}{\Gamma - 1}
\end{equation}

Con estas ecuaciones podemos formar una matriz de $5x5$ escritas en coordenadas cartesianas:

\begin{equation} \label{euler_cartesianas}
\dfrac{\partial \mathbf{U}}{\partial t}+\dfrac{\partial \mathbf{F}}{\partial x}+\dfrac{\partial \mathbf{G}}{\partial y}+\dfrac{\partial \mathbf{H}}{\partial z}= \mathbf{S}
\end{equation}

Donde\\
\begin{center}


$\mathbf{U}=
\left(\begin{smallmatrix}
\rho \\
\rho u \\
\rho v \\
\rho w \\
E \\
\end{smallmatrix}\right)
$,
$\mathbf{F} =
\left(\begin{smallmatrix}
\rho u \\
\rho u^{2}+P \\
\rho uv \\
\rho uw \\
u(E+P) \\
\end{smallmatrix}\right)
$,
$\mathbf{G} =
\left(\begin{smallmatrix}
\rho v\\
\rho vu \\
\rho v^{2}+P \\
\rho vw \\
v(E+P) \\
\end{smallmatrix}\right)
$,
$\mathbf{H} =
\left(\begin{smallmatrix}
\rho w\\
\rho wu \\
\rho wv \\
\rho w^{2}+P \\
w(E+P) \\
\end{smallmatrix}\right)
$, 
$\mathbf{S} =
\left(\begin{smallmatrix}
0 \\
f_{x} \\
f_{y} \\
f_{z} \\
G-L+\textbf{f} \cdot \textbf{u} \\
\end{smallmatrix}\right)
$
\end{center}

El término de $\mathbf{U}$ son nuestras variables conservadas, los términos $\mathbf{F}$, $\mathbf{G}$, $\mathbf{H}$ son nuestros fluidos  con velocidades en la dirección $x$, $y$ y $z$ y $\mathbf{S}$ son los términos fuente. Para poder resolver computacionalmente estas ecuaciones diferenciales parciales vamos a utilizar el método de las diferencias finitas y el método de Lax sin considerar tos términos fuente es decir $\mathbf{S}=0$ en el siguiente capitulo si se trataran fuentes particulares.

\section{Hidrodinámica relativista}
En esta parte se añadirá a los códigos que ya hemos generado previamente, las próximas secciones abordara acerca de como cambian nuestras primitivas, como afectan a nuestras variables conservadas y como las podemos desacoplar así como varios ejemplos al cambiar varios valores de nuestros parámetros y de las condiciones iniciales

\subsection{Primitivas}
Las ecuaciones que teníamos para fluidos newtonianos se pueden modificar para hacerlos relativistas. Para esto vamos a partir de 2 ecuaciones importantes que son la ecuación de energía-momento y la ecuación de conservacion de masa:

\begin{equation}\label{Ecuacion_conservación_masa}
\left( \rho u^{\alpha} \right)_{, \alpha} =0
\end{equation}

\begin{equation}\label{Ecuacion_momento_energia}
T_{, \beta}^{\alpha \beta}=0
\end{equation}

De la ecuación \ref{Ecuacion_conservación_masa} tenemos la cuadrivelocidad para un sistema de 3 coordenadas y la velocidad $c=1$ lo podemos ver como $u^{\mu}=\gamma \left( 1, \textbf{v}\right)$ y sustituyendo este resultado (en 2 dimensiones espaciales) tendremos las ecuaciones $u_1$, $F_1$ y $G_1$. Para la ecuación \ref{Ecuacion_momento_energia} podemos escribir el tensor de energía-momento como $T^{\mu \nu} = \rho h u^{\mu} u^{\nu} + Pg^{\mu \nu}$ y usando la métrica de Minkowski

\begin{equation}
\eta_{\alpha \beta}= 
\begin{pmatrix}
 -1 & 0 & 0 & 0 \\
 0 & -1 & 0 & 0 \\
 0 & 0 & -1 & 0 \\
 0 & 0 & 0 & 1 \\
\end{pmatrix}
\end{equation}

Con lo que podemos escribir a $T^{\mu \nu}$ matricialmente como:

\begin{equation}
T^{\mu \nu} =
\begin{pmatrix}
\rho h \gamma^2-P & \rho h \gamma^2 v_{x}  & \rho h \gamma^2 v_{y} & \rho h \gamma^2 v_{z} \\

\rho h \gamma^2 v_{x} & \rho h \gamma^2 v_{x}^{2}+P & \rho h \gamma^2 v_{x}v_{y} &  \rho h \gamma^2 v_{x}v_{z} \\

rho h \gamma^2 v_{y} & \rho h \gamma^2 v_{y}v_{x} & \rho h \gamma^2 v_{y}^{2}+P & \rho h \gamma^2 v_{y}v_{z}\\

\rho h \gamma^2 v_{z} & \rho h \gamma^2 v_{z}v_{x} & \rho h \gamma^2 v_{z}v_{y} & \rho h \gamma^2 v_{z}^2 + P
   
\end{pmatrix}
\end{equation}
Entonces nuestras ecuaciones quedarían de la siguiente manera
\begin{align}
u_{1}& = \rho \gamma \\ 
u_{2}& = \rho v_{x} \gamma^{2} h \\ 
u_{3}& = \rho v_{y} \gamma^{2} h \\ 
u_{4}& = \rho \gamma^{2} h - P 
\end{align}

Donde $\rho$ es la densidad, $\gamma$ es el factor de lorentz, $v_{x}$ y $v_{y}$ son las velocidades de nuestros fluidos (en 2 dimensiones pero se puede extender esto a 3 sin ningún problema), $h$ es la entalpía  y $P$ es la presión. Para los fluidos quedan de la siguiente manera
\begin{align}
F_{1}& = \rho v_{x} \gamma \\ 
F_{2}& = \rho v_{x} v_{x} \gamma^{2} h + P\\ 
F_{3}& = \rho v_{x} v_{y} \gamma^{2} h \\ 
F_{4}& = \rho v_{x} \gamma^{2} h 
\end{align}

\begin{align}
G_{1}& = \rho v_{y} \gamma \\ 
G_{2}& = \rho v_{y} v_{x} \gamma^{2} h \\ 
G_{3}& = \rho v_{y} v_{y} \gamma^{2} h + P\\ 
G_{4}& = \rho v_{y} \gamma^{2} h
\end{align}
\begin{align}
H_{1}& = \rho v_{z} \gamma \\ 
H_{2}& = \rho v_{z} v_{x} \gamma^{2} h \\ 
H_{3}& = \rho v_{z} v_{y} \gamma^{2} h \\ 
H_{4}& = \rho v_{z}^{2} \gamma^{2} h + P
\end{align}



\section{Desacoplamiento de las ecuaciones de la hidrodinámica}

Al final del metodo de las diferencias finitas, obtenemos nuestras variables conservadas (\emph{U}), pero para calcular nuestros flujos de nuevo necesitamos recuperar nuestras primitas, es decir, calcular nuestras variables primitivas $(\rho, \, u, \, v,\, w, \, P )$ en función de nuestras variables conservadas $(u_1, \, u_2, \, u_3, \, u_4, \, u_5)$.

Despejar la densidad es sencillo ya que es directo $u_1= \rho$ por lo tanto:

\begin{equation}\label{primitiva_densidad}
\rho = u_1
\end{equation}

Para las velocidades $u_i=\rho \upsilon$, donde $i=2,3,4$ y $\upsilon=u,v,w$, nos da $\upsilon= u_i/ \rho$ y usando la ecuación \ref{primitiva_densidad} queda:

\begin{equation} \label{primitiva_velocidades}
\upsilon = u_1/u_i
\end{equation}
Para la ecuacion de la energía $u_5=E$ combinando con la ecuacion de estado y la ecuación \ref{primitiva_velocidades} obtenemos
\begin{equation}
P = \left( \Gamma - 1 \right) \left[ u_5 - \frac{u_1 \left( \sum_{i=2}^{4} u_1/u_i \right)^2}{2} \right]
\end{equation}

\section{Desacoplamiento de las ecuaciones de la hidrodinámica relativista}
Para poder desacoplar las ecuaciones, partimos de la relación de las densidades de energía total y del módulo de los momentos

\begin{equation}\label{ecuacion_de_energia}
E=W-p
\end{equation}

\begin{equation}\label{modulos de los momentos}
\abs{m}^2= W^{2}\abs{v}^{2}
\end{equation}

Donde $W=D h \gamma$ y $D=\rho \gamma$. Para evitar errores en el límite no relativista se debe resolver la ecuación conservada restando la densidad de masa a la energía para definir una nueva variable conservada ($E^{'}=E-D$), para las cancelaciones en el límite ultra-relativista basados en $\gamma \abs{v^2}$ que se tiene cuando $\abs{v} \rightarrow 1$, se debe de crear otra variable, que en este caso seria $\abs{u}^2=\gamma \abs{v^2}$ e introduciendo las variables $W^{'}=W-D$. Podemos re-escribir la última ecuación de la siguiente manera

\begin{eqnarray*}
 W^{'}& = &D(h \gamma -1)\\
&=& D\left[ \left(1-\epsilon+ \frac{p}{\rho}\right) \gamma - 1 \right]\\
&=& D \left(\gamma-1 \right) \frac{\gamma+1}{\gamma+1}+\frac{D \gamma }{\rho}\left(\rho \epsilon + p \right)
\end{eqnarray*}

Recordando que $D=\rho \gamma$ y que a partir de la variable introducida $u^{2}$ podemos re-scribir el factor de Lorentz como $\gamma^{2} = 1- u^{2}$

\begin{eqnarray}\label{W_prima}
\nonumber W^{'}&=&\frac{D u^{2}}{\gamma + 1}
+\frac{\rho\gamma \gamma}{ \rho }\left(\rho \epsilon + p \right)\\
&=& \frac{D u^{2}}{\gamma + 1} + \gamma^{2} \chi
\end{eqnarray}

Donde $\chi=\rho \epsilon + p$, derivando con respecto a $W^{'}$ la ecuación \ref{ecuacion_de_energia} queda como
\begin{equation}\label{derivada_E_W}
\dfrac{dE}{dW^{'}}=1-\dfrac{dp}{dW^{'}}
\end{equation}

Nosotros no sabemos como es la expresión $\dfrac{dE}{dW^{'}}$, así que asumiremos que $p=p(\rho, \chi)$ por lo que podemos aplicar la regla de la cadena (para mas detalles consulte)
\begin{equation}\label{cadena}
\dfrac{dp}{dW^{'}}=\dfrac{\partial p}{\partial\chi}\Bigg |_{\rho} \dfrac{d\chi}{dW^{'}} + \dfrac{\partial p}{\partial \rho}\Big |_{\chi} \dfrac{d \rho}{d W^{'}}
\end{equation}

Para calcular $\dfrac{dp}{d\chi}$ tenemos que por ser el caso de un gas ideal
\begin{equation}
h=1+\frac{\Gamma}{\Gamma-1}\frac{p}{\rho}
\end{equation}
Donde $h$ también puede ser escrito como
\begin{equation}
h=1+\epsilon+\frac{p}{\rho}
\end{equation}
Si combinamos estas 2 últimas ecuaciones podemos llegar a que 
\begin{equation}
p(\chi,\rho)=\frac{\Gamma-1}{\Gamma}\chi
\end{equation}

Con lo que al derivar con respecto de $\chi$ nos da como resultado
\begin{eqnarray}\label{der_presion}
& \dfrac{d p}{d \chi}&=\frac{\Gamma-1}{\Gamma}\\ &\dfrac{d p}{d \rho}&= 0
\end{eqnarray}

De la ecuación \ref{W_prima} podemos despejar $\chi$, lo que queda como
\begin{equation}
\chi=\frac{W^{'}}{\gamma}- \frac{D u^{2}}{(1+\gamma)\gamma^{2}}
\end{equation}

Derivando implícitamente la ecuación \ref{W_prima} respecto a $W^{'}$ nos quedaría
\begin{eqnarray*}
W^{'}&=&D\left(\gamma-1 \right) + \chi \gamma^{2}\\%saloto de linea
&=& D\left(\frac{1}{\sqrt{1-\nu^{2}}} -1\right)+\chi \dfrac{1}{1-\nu^{2}} \\%saloto de linea
\dfrac{d W^{'}}{d W^{'}} &=& D \dfrac{d (1-v^2)^{-1/2}}{d W^{'}}+\dfrac{d \chi}{dW^{'}}(1-v^2)^{-1}+\dfrac{d (1-v^2)^{-1} }{d W^{'}}\chi \\  %salto de linea 
1 &=& \frac{D(1-v^2)^{-3/2}}{2} \dfrac{d v^{2}}{d W^{'}}+\dfrac{d \chi}{dW^{'}}(1-v^2)^{-1}+ \chi (1-v^2)^{-2}  \dfrac{d v^{2}}{d W^{'}} \\ %salto de linea
\frac{1}{\gamma ^2} &=& \frac{D \gamma}{2} \dfrac{d v^{2}}{d W^{'}} + \dfrac{d \chi}{dW^{'}} + \chi \gamma^2 \dfrac{d v^2}{dW^{'}} \\ %salto de linea
\end{eqnarray*}

Con lo que al final la ecuación se puede escribir de la siguiente manera

\begin{equation}\label{der_chi}
\dfrac{d \chi}{dW^{'}}=\frac{1}{\gamma^2}-\frac{\gamma}{2}(D-2\gamma \chi) \dfrac{d v^2}{dW^{'}}
\end{equation}

Y para

\begin{equation}\label{der_rho}
\dfrac{d \rho}{d W^{'}}= D \dfrac{d\left(1/ \rho \right) }{d W^{'}} = - \frac{D \gamma}{2}  \dfrac{d v^2}{dW^{'}}
\end{equation}

despejando la ecuación \ref{modulos de los momentos} podemos llegar a escribir el módulo de la velocidad al cuadrado de la siguiente manera
\begin{equation}
\abs{v^{2}} = \frac{\abs{m^{2}}}{W^{'}} 
\end{equation}
Donde $m_i= \rho v_i \gamma h$ para $i=x,y$.

Podemos demostrar que $\dfrac{d \abs{v^2}}{W}=\dfrac{d \abs{v^2}}{W^{'}}$ para esto vamos a partir de de lo siguiente 
\begin{eqnarray*}
\abs{v^2}&=& \abs{m^{2}} \left(W^{'} + D \right)^{-2} \\
\dfrac{d \abs{v^2}}{d W^{'}} &=& \frac{-2 \abs{m}^2}{W^{'}+D^{3}} \\
&=& \frac{2 \abs{m}^2}{W^{3}} \\
&=& \dfrac{d \abs{v^2}}{d W^{'}}
\end{eqnarray*}

Con lo que podemos decir que 

\begin{equation}\label{der_v2}
\dfrac{d\abs{v}^2}{d W^{'} }=-\frac{2 \abs{m}^{2}}{W^{3}}
\end{equation}
Con todo esto ya sabemos cuanto es lo que vale la ecuación \ref{derivada_E_W}, con esto ya podemos usar el método de Newton-Raphson para poder encontrar $W^{'}$.\\

El método de Newton-Raphson es un algoritmo iterativo que se usa para encontrar raíces  de una función real:

\begin{equation}
W^{'(k+1)}=W^{'}-\frac{f(W^{'})}{\dfrac{d f(W^{'})}{d W^{'}}}
\end{equation}

De la ecuación de la energía, podemos utilizarla como a la función a la que queremos encontrar la raíz

\begin{equation} \label{ecuación_f}
f(W^{'})=W^{'}-E^{'}-p
\end{equation}

Donde $E^{'}=W^{'}-p$ y que $\dfrac{d f(W^{'})}{d w} \equiv \dfrac{dE}{dW^{'}}$ dado por la ecuación \ref{derivada_E_W}. Para iniciar el proceso de iteración se tiene que hacer una suposición, para esto, con ayuda de las ecuaciones \ref{ecuacion_de_energia} y \ref{modulos de los momentos} podemos llegar a que la presión es:
%===============================================================================================

\begin{equation} \label{presion_de_newton}
p=\frac{\abs{m}^{2}-W^{2}\abs{v}^{2}+4W^{2}-4EW}{4W}
\end{equation}

Como podemos ver el denominador es una función convexa cuadrática que depende $\abs{v}^{2}$ y $W$ y cumple con que $W$ este fuera del intervalo de las raíces positivas y negativas.

Al denominador de la  ecuación podemos encontrar $W$ ya que $\abs{v}^{2}=1-1/\gamma^{2}$ suponiendo $\gamma \rightarrow \infty$ podemos despejar $W$

\begin{equation}\label{suposicion_de_W}
W=\frac{-(-2E)+\sqrt{(-2E)^{2}-(3)(\abs{m}^{2})}}{3}
\end{equation}

Con esto ya se puede hacer las aproximaciones para obtener $W$, y podemos calcular las siguientes relaciones 
\begin{eqnarray}
\abs{v}^{2} &=& \frac{\abs{m}^{2}}{W^{2}}\label{prim_v2}\\ 
u^{2}&=&\frac{\abs{v}^{2}}{1-\abs{v}^{2}}\label{u2}\\
\gamma &=& \sqrt{1+u^{2}}
\end{eqnarray}
Y las nuevas primitivas.\\

Velocidades:
\begin{eqnarray}
v_{x}&=&\frac{u_{2}}{W}\\
v_{y}&=&\frac{u_{3}}{W}\\
\end{eqnarray}

Densidad de masa 
\begin{equation}
\rho=\frac{D}{\gamma}
\end{equation}

Presión térmica
\begin{eqnarray}
\chi&=&\frac{W-D}{\gamma^{2}}-\frac{D \abs{u}^{2}}{(1+\gamma)\gamma^{2}}\\
p&=&\frac{\Gamma-1}{\Gamma} \chi
\end{eqnarray}



\section{Diferencias finitas}
Si tenemos una función $f(x)$ lo suficientemente diferenciable la podemos aproximar por el Teorema de Taylor  en la vecindad de un punto $x_0+\Delta x$ entonces si conocemos todas sus $n$ derivadas de la función $f(x)$ en el punro $x_0$ podemos aproximar de la siguiente manera

\begin{equation}\label{Serie_Taylor}
f\left( x_0 + \Delta x\right) = f\left( x_0 \right)+
\sum_n \frac{\left( \Delta x \right) ^2}{k!}f^{(k)} \left(x_0
\right)
\end{equation}

Si truncamos la serie de Taylor y quitamos los términos de segundo orden podemos escribir la ecuación \ref{Serie_Taylor} como:

\begin{equation}
f\left( x_0 + \Delta x \right) = f(x_0) - \Delta x f^{(1)} (x_0) + O \left( \Delta x \right)
\end{equation}

Despejando $f(x_0)$ queda lo que se conoce como diferrencia finita hacia adelante
\begin{equation}\label{fwd}
f_{fwd}^{'}=\frac{f\left(x + \Delta x \right) - f(x) }{\Delta x}=\frac{f_{i+1}-f_{i}}{\Delta x}
\end{equation}

Tambien se puede hacer en el entorno $x_0- \Delta x$ y siguiendo los mismos pasos anteriores llegamos a lo que se le conoce como diferencia finita hacia atras.



\begin{equation}\label{bwd}
f_{back}^{'}=\frac{f\left(x \right) - f(x - \Delta x) }{\Delta x}=\frac{f_{i}-f_{i-1}}{\Delta x}
\end{equation}

Si obtenemos el promedio de las ecuaciones \ref{fwd} y \ref{bwd} obtenemos la central:

\begin{equation} \label{Centrada}
f_{central}^{'}=\frac{f\left(x + \Delta x\right) - f(x - \Delta x) }{2\Delta x}=\frac{f_{i+1}-f_{i-1}}{2 \Delta x}
\end{equation}



\subsection{Lax-Friederich}

Si consideramos la siguiente ecuación diferencial parcial

\begin{equation} \label{ecu_conser}
u_t+ f\left(u \right)_t=0
\end{equation}

Un método conservativo para resolver este tipo de ecuación diferencial es 

\begin{equation}\label{esquema conservativo}
u_i^{n+1} = u_i^{n} -\frac{\Delta t}{\Delta x} \left(f_{i-\frac{1}{2}} - f_{i+\frac{1}{2}}\right)
\end{equation}



Si hacemos la siguiente elección de flujo 

\begin{equation}
f_{i+\frac{1}{2}} = f_{i+\frac{1}{2}} \left(
u_{i} , u_{i+1}\right) =\frac{1}{2} \left(f_{i} + f_{i+1} \right) 
\end{equation}

Y para 

\begin{equation}
f_{i-\frac{1}{2}} = f_{i-\frac{1}{2}} \left(
u_{i} , u_{i-1}\right) =\frac{1}{2} \left(f_{i} - f_{i-1} \right) 
\end{equation}

Y lo sustituimos en la ecuación \ref{esquema conservativo} nos queda el siguiente resultado

\begin{equation}\label{ec_inestable}
u_i^{n+1} = u_i^{n} + \frac{1}{2}\frac{\Delta t}{\Delta x} \left(f_{i+1} - f_{i-1} \right)
\end{equation}

Pero esta solución es inestable, por el primer término del lado derecho de la ecuación, para hacerlo estable  Peter Lax y Kurt O. Friedrichs sustituyerón este término por $(u_{i+1}^n-u_{i-1}^n)$  por lo que podemos reescribir la ecuación \ref{ec_inestable} como

\begin{equation}\label{ec_estable}
u_i^{n+1} =(u_{i+1}^n-u_{i-1}^n) + \frac{1}{2}\frac{\Delta t}{\Delta x} \left(f_{i+1} - f_{i-1} \right)
\end{equation}


\subsection{HLL}

	Otro método para resolver las ecuaciones de la hidrodinámica es usar el método de Harten-Van-Leer. Definiendo el flujo numérico intercelda de Gudonov

\begin{equation}
F_{i+\frac{1}{2}}=F \left( U_{i+\frac{1}{2}} \right)
\end{equation}

Para el cual $U_{i+\frac{1}{2}}(0)$ tiene la misma solución para $U_{i+\frac{1}{2}}(x/t)$ con lo que el problema de Riemann se reduce a :

\begin{equation} \label{Ecuacion_discreta_conservada}
\begin{array}{ll}
U_t + F \left( U \right)_x = 0 \\
U \left(x,0 \right) = 
\left\lbrace
\begin{array}{rr}
U_L \quad \textup{si} \quad x<0  \\
U_R \quad \textup{si} \quad x>0
\end{array}
\right.
\end{array}
\end{equation}

\begin{figure} %fuente de la imagen:libro del toro/
  \centering
    \includegraphics[width=0.5\textwidth]{Figuras/HLL_onda.png}
  \caption{Plano x-t que muestra que muestra un volumen definido}
  \label{fig:Plano x_t}
\end{figure}

Si consideramos un control de volumen $\left[x_L, x_R \right]\times \left[ 0 , T \right]$, tales que $x_L \leq TS_L$ y $x_R \geq TS_R$ (ver figura \ref{fig:Plano x_t}) donde $S_L$ y $S_R$ son las velocidades de las ondas mas rápidas de los estados iniciales $U_L$ y $U_R$ respectivamente y $T$ es un tiempo definido. Si usamos la forma integral de la ecuacion \ref{Ecuacion_discreta_conservada} en nuestro volumen definido $\left[x_L, x_R \right]\times \left[ 0 , T \right]$

\begin{equation*}\label{Forma_integral_conservadas}
\int_{x_L}^{x_R} \left[ U\left( x, T \right) -
 U\left( x, 0 \right) \right] dx = 
 \int_{0}^{T} \left[ F \left(U\left( x_L, t \right) \right) -
 F \left(U\left( x_R, t \right) \right) \right] dt 
\end{equation*}
Entonces
\begin{equation}\label{integral_consistencia}
\int_{x_L}^{x_R} U\left( x, T \right) dx =\int_{x_L}^{x_R} U\left( x, 0 \right) dx+
\int_{0}^{T}  F \left(U\left( x_L, t \right) \right)dt -
\int_{0}^{T}  F \left(U\left( x_R, t \right) \right) dt
\end{equation}

Usando las condiciones de la ecuación \ref{Ecuacion_discreta_conservada} podemos evualuar la integral

\begin{equation*}
\int_{x_L}^{x_R} U\left( x, T \right) dx = 
x_R U_R -x_L U_L+T F_L-T F_R
\end{equation*}
Donde $F_L = F \left( U_L \right)$ y $F_R = F \left( U_R \right)$, entonces

\begin{equation}\label{Condición_de_consistencia}
\int_{x_L}^{x_R} U\left( x, T \right) dx = 
x_R U_R -x_L U_L+T \left( F_L- F_R \right)
\end{equation}

Si separamos ahora la ecuación \ref{integral_consistencia} en 3 integrales de la siguiente manera:

\begin{equation}
\int_{x_L}^{x_R} U\left( x, T \right) dx = 
\int_{x_L}^{T S_L} U \left(x, T \right)dx+
\int_{T S_L}^{T S_R} U \left(x, T \right)dx+
\int_{T S_R}^{x_R} U \left(x, T \right)dx
\end{equation}

Si ahora evualuamos el tercer y el primer término en el lado derecho, obtenemos:

\begin{equation}\label{condicion_consistencia_2}
\int_{x_L}^{x_R} U\left( x, T \right) dx =
\int_{T S_L}^{T S_R} U \left(x, T \right)dx+
\left( T S_L - x_L \right) U_L+
\left( x_L - T S_R \right) U_R
\end{equation}

Si combinamos la ecuación \ref{Condición_de_consistencia} y 
\ref{condicion_consistencia_2}

\begin{equation*}
x_R U_R -x_L U_L+T \left( F_L- F_R \right) =
\int_{T S_L}^{T S_R} U \left(x, T \right)dx+
\left( T S_L - x_L \right) U_L+
\left( x_L - T S_R \right) U_R
\end{equation*}

Entonces 

\begin{equation*}
\int_{T S_L}^{T S_R} U \left(x, T \right)dx=
\left( T S_L - x_L \right) U_L+ x_L U_L +
\left( x_L - T S_R \right) U_R-x_R U_R -
T \left( F_L- F_R \right)
\end{equation*}

Con lo que al final nos queda

\begin{equation} \label{ull_sin_promedio}
\int_{T S_L}^{T S_R} U \left(x, T \right)dx=
T \left( S_R U_R - S_L U_L + F_L - F_R \right)
\end{equation}

Si dividimos la ecuación \ref{ull_sin_promedio} por la diferencia de las velocidades maximas de las señales de las ondas, obtenemos el promedio de la funcion que esta entre las velocidades de la onda, entonces

\begin{equation}
\frac{1}{T \left( S_R -S_L \right)}\int_{T S_L}^{T S_R} U \left(x, T \right)dx =
\frac{S_R U_R - S_L U_L + F_L - F_R}{S_R - S_L}
\end{equation}

Si conocemos las velocidades de la onda podemos escribir la ecuación como 

\begin{equation}\label{u_hll}
U^{hll} = \frac{S_R U_R - S_L U_L + F_L - F_R}{S_R - S_L}
\end{equation}

Ahora si aplicamos la forma integral ( como en el caso de la ecuación \ref{Condición_de_consistencia}) a lado izquierdo de nustro plano obtenemos lo siguiente

\begin{equation}
\int_{T S_L}^{0} U\left( x, T \right) dx = 
-T S_L U_L+T \left( F_L- F_{0L} \right)
\end{equation}
Donde $F_{0L}$ es el flujo a lo largo del eje $t$. Si despejamos $F_{0L}$, nos queda lo siguiente

\begin{equation}\label{ec F_0L}
F_{0L} = F_L - S_L U_L + \frac{1}{T}  \int_{T S_L}^{0} U\left( x, T \right) dx
\end{equation}

Esta última ecuación nos servira para calcular los flujos usando el método de Harteen-Van-Leer, el cual dividian el plano en tres espacios:

\begin{figure} %fuente de la imagen:libro del toro/
  \centering
    \includegraphics[width=0.5\textwidth]{Figuras/HLL.png}
  \caption{Aproximación de 3 estados distintos en el plano x-t, en el cual se trata de calcular los flujos en la región $U^{hll}$ limitados por las velocidades de señal de la onda}
  \label{fig:HLL}
\end{figure}

\begin{equation}
U \left(x,t \right) = 
\left\lbrace
\begin{array}{rr}
U_L \quad \textup{si} \quad \frac{x}{t}< S_L  \\
U_{hll} \quad \textup{si} \quad S_L< \frac{x}{t} <S_R \\
U_R \quad \textup{si} \quad  \frac{x}{t} > S_R
\end{array}
\right.
\end{equation}

Los flujos $F_R$ y $F_L$ pueden ser calculados directamente ya que solo dependen de $U_R$ y $U_L$ respectivamente pero $F_{hll} \neq F \left( U_{hll} \right)$, asi que resolvemos la integral de la ecuación \ref{ec F_0L} para asi obtener el flujo a traves del eje t

\begin{equation*}
F_{hll} = F_L -S_L U_L+ \frac{1}{T}U_{hll}\left(0- TS_L\right)
\end{equation*}

Entonces
\begin{equation}\label{f_hll 1}
F_{hll} = F_L +S_L \left( U_{hll} -U_L \right)
\end{equation}

Si sustituimos \ref{u_hll} en \ref{f_hll 1} obtenemos:

\begin{equation*}
F_{hll} = F_L +S_L \left( \frac{S_R U_R - S_L U_L + F_L - F_R}{S_R - S_L} -U_L \right)
\end{equation*}
Entonces
\begin{equation*}
F_{hll} = \frac{F_L S_R -F_L S_L+S_L S_R U_R-S_L^2 U_L+S_L  F_L- S_L F_R-S_R S_L U_L + S_L^2 U_L}{S_R-S_L}
\end{equation*}

Eliminando términos semejantes queda
\begin{equation}
F_{hll} = \frac{S_R F_L -S_L F_R + S_L S_R \left(U_R-U_L \right)}{S_R -S_L}
\end{equation}

Con lo que el flujo intermedio de la celda de Gudonov esta dado por:

\begin{equation}
F_{i+\frac{1}{2}}^{hll} = 
\left\lbrace
\begin{array}{rr}
F_L \quad \textup{si} \quad 0 \leq S_L  \\
\frac{S_R F_L -S_L F_R + S_L S_R \left(U_R-U_L \right)}{S_R -S_L} \quad \textup{si} \quad S_L \leq 0 \leq S_R \\
F_R \quad \textup{si} \quad  0 \geq  S_R
\end{array}
\right.
\end{equation}


%\section{Hidrodinámica con fuentes}


\chapter{Verificación del código}

Las siguientes pruebas verifican el buen funcionamiento del código al hacer tanto Lax como HLL en relativista y no relativista

\section{Onda de choque} %Shock Wave and High Pressure Phenomena) Isabelle Sochet (eds.
La onda de choque se produce por la liberación rápida de energía comprimida en un volumen fijo. Esta explosión divide a nuestro dominio en 2 zonas distintas como se muestra en la Figura 3.1. La zona I, estará determinado por la densidad, presión, y velocidades del medio ambiente ($\rho_m, \, P_m\, v_{x_m}, \, v_{y_m}$, respectivamente). La zona II (i.e. la región de la onda de choque), se produce al depositar una energía $E_{in}$ dentro de una región definida por un radio interno ($R_{in}$) en reposo. Para ver más detalles véase el cuadro 3.1.

\begin{table}[htbp]\label{Tabla_parametros}
\begin{center}
\begin{tabular}{|c|c|c|}
\hline 
\textbf{Parámetro} & \textbf{Descripción} & \textbf{valor} \\ 
\hline 
$R_{in}$ & Radio interno de la onda expansiva & 0.2 \\ 
\hline 
$\rho_{in}$ &  Densidad interna de la onda expansiva & 5.0 \\ 
\hline 
$\rho_{out}$ &  Densidad del medio  & 1.0 \\
\hline 
$P_{int}$ & Presión interna de la onda expansiva & 10.0 \\ 
\hline 
$P_{out}$ &  Presión del medio  & 0.1 \\ 
\hline 
$v_{x_{int}}$ & Velocidad interna en el eje x de la onda expansiva & 0.0 \\ 
\hline 
$v_{y_{int}}$ & Velocidad interna en el eje y de la onda expansiva & 0.0 \\ 
\hline 
$v_{x_{out}}$ & Velocidad en el eje x del medio & 0.0 \\
\hline 
$v_{y_{out}}$ & Velocidad en el eje y del medio & 0.0 \\ 
\hline 
Co & Número de Courant & 0.7 \\ 
\hline 
\end{tabular}
\caption{Parámetros que se utilizarán en las siguientes pruebas que se van a realizar para la onda de choque, estos valores son para fluidos no relativistas.}
\end{center}
\end{table}

La $E_{in}$ de la onda de choque está determinada por la $\rho_{in}$, $P_{in}$, y $v_{x_{in}}$ y $v_{y_{in}}$ ya que es la suma de la energía cinética más la energía térmica
$E_{in} = {\frac{1}{2}} \rho_{in} v_{in}^2 + {\frac{P_{in}}{\Gamma -1 }}$).
Dado que la zona II está en reposo $v_{x_{in}}=0$ y $v_{y_{in}}=0$, se tiene:
$E_{in} = {\frac{P_{in}}{\Gamma -1 }}$)
Cabe señalar que los valores de la densidad y presión del medio ambiente tendrán valores mas bajos que sus respectivos de la onda de choque. 

\begin{figure}[H]
\centering
\includegraphics[width=0.5\textwidth]{./Figuras/Pruebas/Prueba_onda_choque/onda_choque}
\caption{La zona I (medio ambiente), se caracterizará por tener valores de densidad, presión y velocidades ($\rho_m, \, P_m \, v_{x_m}, \, v_{y_m}$, respectivamente). La zona II (onda de choque) se produce al depositar una energía ($E_{in}$) dentro de un circulo (con $R_{in}$).} \label{fig:onda_choque_example}
\end{figure}

La onda de choque se dejó evolucionar con el paso del tiempo, y se analizó la interacción que hay entre las zonas I y II. En específico, se siguió la evolución de una onda de choque con una velocidad de expansión muy por debajo de la velocidad de la luz (denominado como caso newtoniano), y además la evolución de una onda de choque cuya velocidad de expansión estaba cercana a la velocidad de la luz (caso relativista). Cabe señalar que en ambos casos se siguió la evolución utilizando el método numéricos de Lax y el de HLL (véase la Sección tal tal para mayor información). La condición a la frontera utilizada en todos los bordes fue la de \emph{outflow}\footnote{Para ver más detalles acerca de esta frontera y/o de otro tipo de fronteras veáse el apéndice \ref{aped.B}}. 


%tiempo = 0.5 s
%lapso = tiempo/20

\begin{table}[htbp]\label{Tabla_parametros}
\begin{center}
\begin{tabular}{|c|c|c|}
\hline 
\textbf{Parámetro} & \textbf{Descripción} & \textbf{valor} \\ 
\hline 
$R_{in}$ & Radio interno de la onda expansiva & 0.2 \\ 
\hline 
$\rho_{in}$ &  Densidad interna de la onda expansiva & 5.0 \\ 
\hline 
$\rho_{out}$ &  Densidad del medio  & 1.0 \\
\hline 
$P_{int}$ & Presión interna de la onda expansiva & 10.0 \\ 
\hline 
$P_{out}$ &  Presión del medio  & 0.1 \\ 
\hline 
$v_{x_{int}}$ & Velocidad interna en el eje x de la onda expansiva & 0.0 \\ 
\hline 
$v_{y_{int}}$ & Velocidad interna en el eje y de la onda expansiva & 0.0 \\ 
\hline 
$v_{x_{out}}$ & Velocidad en el eje x del medio & 0.0 \\
\hline 
$v_{y_{out}}$ & Velocidad en el eje y del medio & 0.0 \\ 
\hline 
Co & Número de Courant & 0.7 \\ 
\hline 
\end{tabular}
\caption{Parámetros que se utilizarán en las siguientes pruebas que se van a realizar para la onda de choque, estos valores son para fluidos no relativistas.}
\end{center}
\end{table}


\subsection{Caso newtoniano} 
\subsubsection{Lax} \label{subsec: Prueba Lax}
Como primeras pruebas se realiza: explicas la 3.2, y además se llevó a cabó (explicas la figura 3.5).

Las dos primeras pruebas: 

1. Newtoniano + Lax + onda choque centrada 2. Newtoniano + Lax + onda choque borde

pruebas fueron usando el método de Lax, usando las ecuaciones relativistas con la constante de relación específica $\Gamma = 3/4$, así como las no relativistas $\Gamma = 5/3$, la descripción del fenómeno es de una onda expansiva sobre un medio constante (ver figura\ref{fig:onda_choque1_t_0}) con una densidad de energía de $E = 0.15 \frac{\mathrm{g}}{\mathrm{s}^2 \mathrm{cm}}$, densidad $\rho = 1 \, \mathrm{g}/\mathrm{cm}^3$ , velocidad $\vec{v} = 0 \, \mathrm{cm}/\mathrm{s}$ y una presión $P=0.1 \frac{\mathrm{g}}{\mathrm{s}^2 \mathrm{cm}} $la simulación es sobre un dominio de 0 a 1 cm sobre el eje \emph{x} y de 0 a 1 cm sobre el eje \emph{y}, con una resolución de 200x200 celdas\footnote{Las celdas de nuestra malla miden $\mathrm{dominio}/200= 0.005 \, \mathrm{cm}$} al tiempo $t=0 \, \mathrm{s}$  y en la onda interna, sus valor de la densidad es de $\rho=5 \, \mathrm{g}/\mathrm{cm}^3$, su velocidad interna será la misma que la del ambiente, la presión interna de la onda $P=10 \frac{\mathrm{g}}{\mathrm{s}^2 \mathrm{cm}}$ y una densidad de energía interna $E = 3.75 \frac{\mathrm{g}}{\mathrm{s}^2 \mathrm{cm}}$,la onda de choque esta centrada en el punto $(x,y) = (0.5, 0.5) \, \mathrm{cm}$, y se correrá en el intervalo de tiempó $t \in \left[ 0 , 0.5 \right]$ s.
%===================IMAGE_LAX_CENTRADO_T=0===============================
\begin{figure}[H]
\centering
\includegraphics[width=0.5\textwidth]{./Figuras/Pruebas/Prueba_onda_choque/onda_choque1_t_0}
\caption{Condición inicial ($t = 0$) de una onda de choque (con $\rho_{in}=5.0, \, v_{x_{in}}=v_{y_{in}}=0.0, \, P_{in}=10.0$ y $E_{in}=3.75$), de radio interno $R_{in} = 0.2$ cm. que se va a expandir sobre un medio (con $\rho_{out}=1.0$  y $ P_{out}=0.1$) estático. Véase el Cuadro \ref{Tabla_parametros} para más detalles.} \label{fig:onda_choque1_t_0}
\end{figure}
%===================IMAGE_LAX_CENTRADO_T=0===============================

La evolución para la onda de choque no relativista muestra una expansión de la onda simétrica. El radio de la onda al tiempo inicial fue de $r=0.2$~cm. A $t=0.05$~s la onda se expandió a un radio de $r=0.3$~cm mientras que para el tiempo $t=0.10$~s se expandió a un radio de $r=0.38$~cm. Debido a lo anterior, la velocidad de expansión de la onda fue de $v = 2.0 \, \mathrm{cm}/\mathrm{s}$ y $v = 1.8 \, \mathrm{cm}/\mathrm{s}$, respectivamente.% FALTA MENCIONAR A FORWARD SHOCK Y LA REVERSE SHOCK.. y por ende el HUECO (ver figura \ref{fig:Lax-newtoniano-prueba1}). ADMÄS FALTA MENCIONAR QUE LA PARTE CENTRAL SE DIFUMINA... LO CUAL ES ESPERABLE DEBIDO A LA VISCOSIDAD NUMERICA.


%===================IMAGE_LAX_CENTRADO_EVOLUCION===============================
\begin{figure}[H]
\centering
\subfigure[$t = 0.05$ s]{\includegraphics[scale = 0.5]{./Figuras/Pruebas/Prueba_onda_choque/Lax/Lax-sedov3}}
\subfigure[$t = 0.1$ s]{\includegraphics[scale = 0.5]{./Figuras/Pruebas/Prueba_onda_choque/Lax/Lax-sedov5}}
\caption{Evolución de la onda de choque mostrada en la \ref{fig:onda_choque1_t_0} usando el método de Lax no relativista. En el panel a) se muestra la onda de choque a t=0.05s, mientras que en el panel b) se muestra para t=0.10s.} \label{fig:Lax-newtoniano-prueba1}
\end{figure}
%===================IMAGE_LAX_CENTRADO_EVOLUCION===============================


Se puede ver que cuando la onda se empieza a expandir por el medio constante, aparecen 4 áreas distintas que interactuan entre nuestro medio ambiente y la onda. El área 1 (ver figura \ref{fig:Lax-newtoniano-prueba1-analisis}) es el material al cual la onda no ha tocado. En el área 2 se ve el material que es chocado con la onda a la que podemos como \emph{forward shock} y esta se aleja del centro de la onda. El área 3 presenta el material chocado con el mismo centro de la onda, esta se caracteriza por ir en reversa y llamaremos \emph{reverse shock}, el material contra el cual aún no ha chocado esta onda es el área 5. La separación que hay entre estas 2 ondas vas creando un vacio que se ve difuminado, esto debido a la viscosidad que presenta nuestro resolvedor de Riemann.


%===================IMAGE_LAX_CENTRADO_EVOLUCION_ANALISIS===============================
\begin{figure}[H]
\centering
\subfigure[$t = 0.05$ s]{\includegraphics[scale = 0.495]{./Figuras/Pruebas/Prueba_onda_choque/Lax/Lax-sedov3-analisis}}
\subfigure[$t = 0.1$ s]{\includegraphics[scale = 0.5]{./Figuras/Pruebas/Prueba_onda_choque/Lax/Lax-sedov5-analisis}}
\caption{En la evolución de la onda usando el método de Lax, podemos encontrar 4 áreas distintas. El área 1 la identificaremos como el material que no ha sido chocado todavía por nuestra onda. El área 2 será el material que es chocado por nuestra onda, esta onda que se llama \emph{forward shock}. En el área 3 podemos ver una onda que va contra si misma que es la \emph{reverse shock}, esta onda arrastra el interior de la onda, va en sentido contrario a la \emph{forward shock}, el material que esta onda aun no ha chocado es al área 5 y el área 4 es el vacío que se va formando cuando se separan la \emph{forward shock} y \emph{reverse shock}.  } \label{fig:Lax-newtoniano-prueba1-analisis}
\end{figure}
%===================IMAGE_LAX_CENTRADO_EVOLUCION_ANALISIS===============================


La segunda prueba consiste en realizar la misma prueba de la onda de choque resuelta con el método de Lax pero ubicando a la onda de choque en una esquina para poder analizar como se comporta la misma al interaccionar con las condiciones a la frontera. El centro de la nueva prueba se colocó en las coordenadas $(x, y) = (0.25, 0.75)$. Cabe señalar que las condiciones a la fronteras tienen la configuración de \textit{outflow}.

%===================IMAGE_LAX_NO_CENTRADO_T=0===============================
\begin{figure}[H]
\centering
\includegraphics[width=0.6\textwidth]{./Figuras/Pruebas/Prueba_onda_choque/onda_choque2_t_0}
\caption{Condición inicial de la prueba número 2 para el resolvedor Lax no relativista. En esta prueba se tiene la misma condición inicial que la mostrada en la figura \ref{fig:onda_choque1_t_0}, pero con las coordenadas de la onda de choque ubicadas en $(x, y) = (0.25, 0.75)$.} \label{fig:onda_choque2_t_0}
\end{figure}
%===================IMAGE_LAX_NO_CENTRADO_T=0===============================


%LAS FRONTERAS NO METEN RUIDO. Y LA EVOLUCIÓN ES CONSISTENTE CON LOS RESULTADOS DE ANTES, Y POR ENDE ESTA BIEN. 
Se puede observar que al evolucionar la onda de choque, esta no muestra ninguna anomalía en las fronteras(ver apéndice \ref{aped.B}) y se nota un buen funcionamiento del código no relativista usando Lax ya que tiene el mismo comportamiento que de la onda centrada. El radio al tiempo $t = 0.07$ s es $r = 0.34$ cm y su velocidad es $v = 2 \, \mathrm{cm}/\mathrm{s}$ y para el tiempo $t = 0.14$ s el radio $r = 0.42$ y su velocidad $v = 1.83 \, \mathrm{cm}/\mathrm{s}$. Con lo que podemos observar que tiene las mismas particularidades que la onda mostrada en la figura \ref{fig:Lax-newtoniano-prueba1}.


%===================IMAGE_LAX_NO_CENTRADO_EVOLUCION===============================
\begin{figure}[H]
\centering
\subfigure[]{\includegraphics[scale = 0.5]{./Figuras/Pruebas/Prueba_onda_choque/Lax/Lax-sedov2_07}}
\subfigure[]{\includegraphics[scale = 0.5]{./Figuras/Pruebas/Prueba_onda_choque/Lax/Lax-sedov2_12}}
\caption{Evolución de la onda de choque mostrada en la Figura \ref{fig:onda_choque2_t_0} usando el método de Lax no relativista. En el panel a) se muestra la onda de choque a t=0.07s, mientras que en el panel b) se muestra para t=0.12s. Al evolucionar la onda se observa que al pasar la frontera no hay ruido o errores y que la velocidad asi como el radio son los mismos que en la onda centrada.} \label{fig:Lax-prueba2_no_centrado}
\end{figure}
%===================IMAGE_LAX_NO_CENTRADO_EVOLUCION===============================

Al salir de nuestro dominio, la onda sigue conservando su forma y las áreas que habiamos visto en la figura \ref{fig:Lax-newtoniano-prueba1-analisis}. El aspecto de la onda no se deforma a pesar de que ya no se encuentra en nuestro dominio.

%===================IMAGE_LAX_NO_CENTRADO_EVOLUCION_ANALISIS===============================
\begin{figure}[H]
\centering
\subfigure[]{\includegraphics[scale = 0.5]{./Figuras/Pruebas/Prueba_onda_choque/Lax/Lax-sedov2_07-analisis}}
\subfigure[]{\includegraphics[scale = 0.5]{./Figuras/Pruebas/Prueba_onda_choque/Lax/Lax-sedov2_12-analisis}}
\caption{Al igual que en la figura \ref{fig:Lax-newtoniano-prueba1-analisis}, la onda presenta 5 áreas descritas, y estas se vuelven más grandes con el paso del tiempo. En el área 4 se puede ver la difuminidad que presenta la onda, al separarse la \emph{forward shock} y la \emph{reverse shock}.} \label{fig:Lax-prueba2_no_centrado-analisis}
\end{figure}
%===================IMAGE_LAX_NO_CENTRADO_EVOLUCION_ANALISIS===============================


%===================LAX_RELATIVISTA===============================
%
%Ahora toca el turno de probar la onda de choque, en la cual las velocidades a la que se expande la onda sean cercanas a la velocidad de la luz.
%
%===================LAX_RELATIVISTA===============================


\subsubsection{HLL} \label{subsec: Prueba HLL}
Las siguientes pruebas son usando HLL, similarmente como en el caso de Lax, se presentarán casos relativistas y no relativistas. La condicion inicial es la misma que la de Lax, es decir, nuestra onda de choque tendra los mismos valores en la que habiamos dicho en la sección \ref{subsec: Prueba Lax}	

%===================IMAGE_HLL_CENTRADO_T=0===============================
\begin{figure}[H]
\centering
\includegraphics[width=0.5\textwidth]{./Figuras/Pruebas/Prueba_onda_choque/onda_choque3_t_0}
\caption{Onda de choque en el tiempo $t = 0$, los valores con los que fue suministrado nuestra onda de choque son los mismos que habiamos dicho en la sección \ref{subsec: Prueba Lax}. Véase el Cuadro \ref{Tabla_parametros} } \label{fig:onda_choque3_t_0}
\end{figure}
%===================IMAGE_HLL_CENTRADO_T=0===============================


La onda de choque fue centrada en el punto $(x,y)=(0.5,0.5)$ cm, su radio es de $r = 0.2 \, \mathrm{cm}$ y al igual que en Lax se presentan 2 ondas, una onda contra el medio, la cual tiene una velocidad constante de $\vec{v}  = 2 \, \mathrm{cm}/\mathrm{s}$ y una onda contra el centro, el radio de la onda se expande a $r = 0.3$ cm al tiempo $t=0.05$ s, mientras que al tiempo $t = 0.1$ s, su radio aumenta hasta $r = 0.38$ cm y disminuye la velocidad de la onda a más de la mitad $\vec{v}=1.8 \, \mathrm{cm}/\mathrm{s}$, entre estas 2 ondas se va creando un vacio (vea figura \ref{fig:HLL-prueba1}).

%===================IMAGE_HLL_CENTRADO_EVOLUCION===============================
\begin{figure}[H]
\centering
\subfigure[$t = 0.05$ s]{\includegraphics[scale = 0.5]{./Figuras/Pruebas/Prueba_onda_choque/HLL/HLL-sedov_05.png}}
\subfigure[$t = 0.1$ s]{\includegraphics[scale = 0.5]{./Figuras/Pruebas/Prueba_onda_choque/HLL/HLL-sedov_10.png}}
\caption{Evolución de la onda de choque mostrada en la figura \ref{fig:onda_choque3_t_0} usando el método de HLL no relativista. En el panel a) se muestra la onda de choque a t=0.05s, mientras que en el panel b) se muestra para t=0.10s.} \label{fig:HLL-prueba1}
\end{figure}
%===================IMAGE_HLL_CENTRADO_EVOLUCION===============================


Al igual que lo que paso con Lax, el método HLL muestra el mismo comportamiento. Se pueden ver 2 ondas distintas, una que avanza hacia adelante (\emph{forward shock}) y otra que avanza en sentido contrario, es decir, se contrae y choca con la misma onda (\emph{reverse shock}), la cavidad que se va formando entre estas 2 ondas es prueba de la viscosidad que tiene nuestro método.

%===================IMAGE_HLL_CENTRADO_EVOLUCION_ANALISIS===============================
\begin{figure}[H]
\centering
\subfigure[$t = 0.05$ s]{\includegraphics[scale = 0.45]{./Figuras/Pruebas/Prueba_onda_choque/HLL/HLL-sedov_05-analisis.png}}
\subfigure[$t = 0.1$ s]{\includegraphics[scale = 0.44]{./Figuras/Pruebas/Prueba_onda_choque/HLL/HLL-sedov_10-analisis.png}}
\caption{En la onda se logran distinguir 2 ondas distintas al evolucionar la misma. Una onda hacia adelante (\emph{forward shock}) y otra onda que se contrae (\emph{reverse shock}), la separación entre las 2 ondas dejan ver un hueco en la onda de menor densidad. La discusión de las áreas de las ondas es la misma que ya habiamos dicho en figura \ref{fig:Lax-newtoniano-prueba1-analisis} y \ref{fig:Lax-prueba2_no_centrado-analisis}. Se puede ver también como, con el paso del tiempo, las áreas donde barren las ondas incrementan su tamaño.} \label{fig:HLL-prueba1-analisis}
\end{figure}
%===================IMAGE_HLL_CENTRADO_EVOLUCION_ANALISIS===============================


En la siguiente prueba se movió el centro de la onda de choque y este se colocó en el punto $(x,y) = (0.25,0.25)$, las fronteras funcionaron apropiadamente, el radio de la expansión de la onda al tiempo $t=0.05$ s es de $r = 0.30$ cm y su velocidad a ese instante es de $v = 2.0 \, \mathrm{cm}/\mathrm{s}$. En el tiempo $t = 0.1$  el radio de la onda se expandió a $r = 0.38$ cm, su velocidad de $v = 1.6 \, \mathrm{cm}/\mathrm{s}$

%===================IMAGE_HLL_NO_CENTRADO_T=0===============================
\begin{figure}[H]
\centering
\includegraphics[width=0.5\textwidth]{./Figuras/Pruebas/Prueba_onda_choque/onda_choque4_t_0}
\caption{Onda de choque en el tiempo $t = 0$, los valores con los que fue suministrado nuestra onda de choque son los mismos que habiamos dicho en la sección \ref{subsec: Prueba Lax}. Véase el Cuadro \ref{Tabla_parametros} } \label{fig:onda_choque4_t_0}
\end{figure}
%===================IMAGE_HLL_NO_CENTRADO_T=0===============================

Al pasar el dominio, la onda se sigue comportando como si estuviera centrada, ya que tiene los mismos radios y velocidades de la misma y al llegar al final de nuestro dominio se nota que no hay ninguna irregularidad de nuestras fronteras.


%===================IMAGE_HLL_NO_CENTRADO_EVOLUCION===============================
\begin{figure}[H]
\centering
\subfigure[$t = 0.05$ s]{\includegraphics[scale = 0.5]{./Figuras/Pruebas/Prueba_onda_choque/HLL/HLL-sedov2_05.png}}
\subfigure[$t = 0.1$ s]{\includegraphics[scale = 0.5]{./Figuras/Pruebas/Prueba_onda_choque/HLL/HLL-sedov2_10.png}}
\caption{Evolución de la onda de choque mostrada en la figura \ref{fig:onda_choque4_t_0} usando el método de HLL no relativista. En el panel a) se muestra la onda de choque a t=0.05s, mientras que en el panel b) se muestra para t=0.10s.}  \label{fig:HLL-prueba2}
\end{figure}
%===================IMAGE_HLL_NO_CENTRADO_EVOLUCION===============================

Al expandirse la onda, al igual que Lax se separan en 2 ondas. La \emph{forward shock} que va contra el medio a la que hemos llamado el área 1 (veáse figura \ref{fig:HLL-prueba2-analisis}). El área 2, es el material del área 1 que va barriendo la \emph{forward shock}. El área 3 es lo contrario, es decir, es el material barrido del área 5 por la \emph{reverse shock} y entre estas 2 ondas se va formando un hueco entre estas 2 ondas que es el área 4 muestra la viscosidad de nuestro método. Dado  que no se muestran anomálias en la evolución de nuestra onda cerca de las fronteras y que estas tienen las velocidades y radios a los instantes de tiempo especificados, podemos decir que nuestro código funciona bien.
%===================IMAGE_HLL_NO_CENTRADO_EVOLUCION_ANALISIS===============================
\begin{figure}[H]
\centering
\subfigure[$t = 0.05$ s]{\includegraphics[scale = 0.5]{./Figuras/Pruebas/Prueba_onda_choque/HLL/HLL-sedov2_05-analisis.png}}
\subfigure[$t = 0.1$ s]{\includegraphics[scale = 0.49]{./Figuras/Pruebas/Prueba_onda_choque/HLL/HLL-sedov2_10-analisis.png}}
\caption{Las áreas mostradas en el panel a) y en el panel b) son las mismas discutidas en la figura \ref{fig:HLL-prueba1-analisis}.}\label{fig:HLL-prueba2-analisis}
\end{figure}
%===================IMAGE_HLL_NO_CENTRADO_EVOLUCION_ANALISIS===============================




\subsection{Puebas no relativistas Lax vs HLL}

En esta sección, vamos a comparar las diferencias que hay entre el método de Lax y el HLL. Las ondas que vamos a ocupar son las de las figuras \ref{fig:onda_choque1_t_0} y la \ref{fig:onda_choque3_t_0}. La onda esta centrada en el punto $(x,y)=(0.5,0.5)$

\begin{figure}[H]
\centering
\subfigure[]{\includegraphics[width=0.45 \textwidth]{./Figuras/Pruebas/Prueba_onda_choque/onda_choque1_t_0}}
\subfigure[]{\includegraphics[width=0.465 \textwidth]{./Figuras/Pruebas/Prueba_onda_choque/onda_choque3_t_0}}
\caption{Condición inicial ($t = 0$) misma para las 2 ondas de choque usando los dos métodos que hemos utilizado que es el de Lax y el de HLL (con $\rho_{in}=5.0, \, v_{x_{in}}=v_{y_{in}}=0.0, \, P_{in}=10.0$ y $E_{in}=3.75$), de radio interno $R_{in} = 0.2$ cm. que se va a expandir sobre un medio (con $\rho_{out}=1.0$  y $ P_{out}=0.1$) estático. Véase el Cuadro \ref{Tabla_parametros} para más detalles.} \label{fig:comparacion_inicial_lax_hll_NR}
\end{figure}


La onda de Lax al tiempo (ver figura \ref{fig:Lax-newtoniano-prueba1}) $t = 0.05$ s tiene un radio $r=0.31$ cm y una velocidad $v = 2.2 \, \mathrm{cm}/\mathrm{s}$, mientras que la onda de HLL (ver figura \ref{fig:HLL-prueba1}) tiene un radio $r = 0.31$ y una velocidad $v = 2.2 \, \mathrm{cm}/\mathrm{s}$. Al avanzar la onda se ve una diferencia entre el método de Lax y HLL, la onda contra el centro de la onda se ve más difusa en HLL que con Lax, por lo que HLL es menos viscoso y resuelve mejor las discontinuidades.


\begin{figure}[H]
\centering
\subfigure[]{\includegraphics[scale = 0.50]{./Figuras/Pruebas/Prueba_onda_choque/Lax-HLL-new/comparacion1}}
\subfigure[]{\includegraphics[scale = 0.50]{./Figuras/Pruebas/Prueba_onda_choque/Lax-HLL-new/comparacion2}}
\caption{Las ondas de choque están establecidas al tiempo $t = 0.5$, ambas imágenes tienen una resolución de 1000x1000, mientras que Lax (panel a)) es un poco más rápido que HLL (panel b)), HLL tiene menor viscosidad en la zona de discontinuidad. La zona de discontinuidad es donde tanto la \emph{forward shock} y la \emph{reverse shock} se separan. En las 2 imagenes, aunque es mínima, se puede notar una difuminación mayor en Lax en esta zona. Por esto podemos decir que HLL es un método para resolver este tipo de discontinuidades.}  \label{fig:Lax-hll-newtoniano1}
\end{figure}

Al dispersarse por completo el centro de la onda, para ambas, se puede ver con mejor claridad que hay 2 diferentes tipos de onda, una contra el medio y otra contra en centro de la misma, para Lax, el radio toma el valor $r = 0.38$ y una velocidad $v = 1.8 \, \mathrm{cm}/\mathrm{s} $. En el caso de HLL es el mismo radio $r = 0.38 $ cm y la misma velocidad $v = 1.8 \, \mathrm{cm}/\mathrm{s}$.



\subsection{Puebas relativistas Lax vs HLL} %200X200

La onda de choque usando relatividad tiene su centro en el punto $(x,y) = (0.5, 0.2)$ y sus valores son los mismos propuestos en el cuadro \ref{Tabla_parametros}

\begin{figure}[H]
\centering
\includegraphics[width=0.5\textwidth]{./Figuras/Pruebas/Prueba_onda_choque/onda_choque6_t_0}
\caption{Onda de choque en el tiempo $t = 0$} \label{fig:onda_choque6_t_0}
\end{figure}

En las coordenadas $(x,y) = (0.5, 0.0)$ se puede ver que el algoritmo de Lax tiene algunos errores puesto que el centro de la onda se deforma en estos puntos, mientras que HLL es casi imperceptible, el radio al tiempo $t=0.06$ s usando el método de Lax es $r = 0.26$ cm y su velocidad $v = 1.0 \, \mathrm{cm}/\mathrm{s}$, mientras que usando HLL su radio es de $r = 0.26$ y su velocidad $v = 1.0 \, \mathrm{cm}/\mathrm{s} $

\begin{figure}[H]
\centering
\subfigure[$t = 0.06$ s]{\includegraphics[scale = 0.5]{./Figuras/Pruebas/Prueba_onda_choque/Lax-HLL-rel/bwlax4}}
\subfigure[$t = 0.06$ s]{\includegraphics[scale = 0.5]{./Figuras/Pruebas/Prueba_onda_choque/Lax-HLL-rel/bwhll4}}
\caption{En las pruebas relativistas, al comparar las imágenes se pueden apreciar una diferencia entre los métodos mas significativa.} \label{fig:Lax-hll-relativista}
\end{figure}

Al avanzar la onda, al tiempo $t = 0.2$ s, en ambos métodos se sigue mostrando 2 ondas una contra el medio y una contra el centro, el radio para Lax $r = 0.39$cm su velocidad $v = 0.95 \, \mathrm{cm}/\mathrm{s}$ el radio usando HLL es de $r = 0.39$ cm y su velocidad $v = 0.95 \, \mathrm{cm}/\mathrm{s}$
\begin{figure}[H]
\centering
\subfigure[$t = 0.2$ s]{\includegraphics[scale = 0.5]{./Figuras/Pruebas/Prueba_onda_choque/Lax-HLL-rel/bwlax10}}
\subfigure[$t = 0.2$ s]{\includegraphics[scale = 0.5]{./Figuras/Pruebas/Prueba_onda_choque/Lax-HLL-rel/bwhll10}}
\caption{El centro de la onda de HLL se ve más difuminada que al usar Lax} \label{fig:Lax-hll-relativista}
\end{figure}

\section{Jet}

En esta parte vamos a simular un jet con las siguientes caracteristicas
\chapter{Medios de densidad}

\section{Condiciones de frontera}
\subsection{SGRB en medio de densidad 1}
\subsection{SGRB en medio de densidad 2}

\section{Comparación con el GRB170817}

\chapter{Resultados}
\chapter{Conclusiones}



\appendix
\chapter{Código}\label{aped.A}

El programa está escrito en lenguaje FORTRAN se compone de un módulo principal el cual está compuesto de un programa principal y este a su vez llamará a varias subrutinas:
\begin{itemize}
\item \textbf{initconds}: Esta subrutina calculará los valores iniciales que le demos al programa

\item \textbf{output}: Devuelve un archivo con los datos que se calculan con el método de Lax

\item \textbf{Courant}: Calcula el paso temporal

\item \textbf{ulax}: Calcula el paso siguiente de las variables conservadas

\item \textbf{boundaries}: En esta parte puedes definir las fronteras a utilizar como outflow o las condiciones para las del jet

\item \textbf{fluxes}: Calculo de los flujos
\end{itemize}

Al usar las ecuacuaciones hidrodinámica relativistas, se agregan 2 subrutinas más:

\begin{itemize}
\item \textbf{uprim}: Este módulo es agregado para poder desacoplar las variables conservadas

\item \textbf{newraph}: Calcula el método de Newton-Rapson será de gran utilidad en el desacoplamineto de las variables conservadas y así obtener nuestras primitivas
\end{itemize}

\begin{lstlisting}[frame=single] 
do i=0,nx+1
  do j=0,ny+1
   
   x=float(i)*dx 	! obtain the position x_i
   y=float(j)*dy 	! obtain the position y_j
   rad=sqrt((x-xc)**2+(y-yc)**2)
   
   if (rad < 0.1) then
   
     lorin=1/sqrt(1-(vxin**2+vyin**2))
     hin=1.+gamma/(gamma-1.)*pin/rhoin
           
     u(1,i,j)=rhoin*lorin
     u(2,i,j)=rhoin*vxin*lorin**2*hin
     u(3,i,j)=rhoin*vyin*lorin**2*hin
     u(4,i,j)=rhoin*lorin**2*hin-pin
    
    else
    
     lorout=1./sqrt(1.-(vxout**2+vyout**2))
     hout=1.+gamma/(gamma-1.)*pout/rhoout
     
     u(1,i,j)=rhoout*lorout
     u(2,i,j)=rhoout*vxout*lorout**2*hout
     u(3,i,j)=rhoout*vyout*lorout**2*hout
     u(4,i,j)=rhoout*lorout**2*hout-pout
     

\end{lstlisting}
y para los fluidos en la subrutina de fluxes
\begin{lstlisting}[frame=single]
          f(1,i,j)=rho*vx*lor
          f(2,i,j)=rho*vx*vx*lor**2*h+P
          f(3,i,j)=rho*vx*vy*lor**2*h
          f(4,i,j)=rho*vx*lor**2*h

          g(1,i,j)=rho*vy*lor
          g(2,i,j)=rho*vx*vy*lor**2*h
          g(3,i,j)=rho*vy*vy*lor**2*h+P
          g(4,i,j)=rho*vy*lor**2*h
\end{lstlisting}
Como el código es una iteración, solo la primera vez que itere estaremos bien, pero, al siguiente bucle saldrá mal debido a que nuestros resultados nos están arrojando en principio las variables conservadas, y lo que se requiere es obtener las primitivas.

\section{Condición inicial}
En la subrutína \textit{initconds} se calcularán las condiciones iniciales, tomando los valores de los parámetros del módulo de \textit{globals}, que en este caso son: la densidad $(\rho)$, las velocidades tanto en $x$ como en $y$ $(v_x, v_y)$, la presión $(p)$ y $\Gamma$. Con estas constantes dadas se calcularán nuestras variables conservadas.

\begin{lstlisting}[frame=single] 
!==============================================================================
! In this module we set the initial condition
!------------------------------------------------------------------------------
      subroutine initconds(time,tprint,itprint)
      use globals
      implicit none
      real, intent(out) :: time, tprint
      integer, intent (out) :: itprint
      integer ::i,j
      real :: x,y, rad

!------------------------------------------------------------------------------
! For the 2D circular blast:
! u(1,i,j) = rho(i,j)
! u(2,i,j) = vx(i,j)
! u(3,i,j) = vy(i,j)
! u(4,i,j) = etot(i,j) = eint + ekin = P/(gamma-1)
!------------------------------------------------------------------------------
      do i=0,nx+1
        do j=0,ny+1
          x=float(i)*dx          ! obtain the position $x_i$
          y=float(j)*dy          ! obtain the position $y_j$
          rad=sqrt((x-xc)**2+(y-yc)**2)

          if (rad < 0.3) then
            u(1,i,j)=rhoin
            u(2,i,j)=rhoin*vxin
            u(3,i,j)=rhoin*vyin
            u(4,i,j)=pin/(gamma-1.)+0.5*u(2,i,j)*u(2,i,j)/u(1,i,j) + 0.5/u(1,i,j)*u(3,i,j)*u(3,i,j)
          else
            u(1,i,j)=rhoout
            u(2,i,j)=rhoout*vxout
            u(3,i,j)=rhoout*vyout
            u(4,i,j)=pout/(gamma-1.) + 0.5/u(1,i,j)*u(2,i,j)*u(2,i,j) + 0.5/u(1,i,j)*u(3,i,j)*u(3,i,j)

          end if

        end do
      end do

!------------------------------------------------------------------------------
! end of the 2D circular blast initial condition
! reset the counters and time to 0
!------------------------------------------------------------------------------
      time=0
      tprint=0
      itprint=0

      return
      end subroutine initconds
!------------------------------------------------------------------------------
! end of the init condition module
!==============================================================================

\end{lstlisting}
En esta parte dan los valores iniciales para nuestra malla tanto en $x$ como en $y$ en el tiempo $t=0$

\subsection{Condición de Courant}
Esta parte del código tiene que ver con los incrementos $\Delta t$, los cuales se van a calcular en este módulo, para poder calcularlos tenemos que tener en cuenta la convergencia y la estabilidad de nuestras ecuaciones diferenciales parciales (ecuación \ref{u_posterior_tensor}). La condición de convergencia establece que la solución de la ecuación numérica se aproxima a la solución con ecuación diferencial parcial original si todos los intervalos finitos tienden a cero, una condición necesaria para la convergencia es que los errores, por ejemplo los debidos al redondeo, no se incrementen con en tiempo. Esta es la llamada la condición de estabilidad. Es una condición tan importante que implica ciertas restricciones al tamaño del paso de tiempo en un proceso explícito. Un análisis de estabilidad para esquemas explícitos a partir de la teoría de las características para soluciones continuas lleva a la conclusión que dichos esquemas, para ser estables, deben cumplir la condición de Courant, que es: 

\begin{equation}
\Delta t \leq \frac{\Delta x}{u+C}
\end{equation}

Donde $C$ es el número de Courant y nos limita a que nuestros $\Delta t$ no sean tan grandes
\begin{lstlisting}[frame=single]
 !==============================================================================
! CFL criterium module
!------------------------------------------------------------------------------
      subroutine courant(dt)
      use globals
      implicit none
      real, intent(out) ::dt
      real :: rho, vx, vy, P, cs
      integer :: i,j

!------------------------------------------------------------------------------
! Calculate the CFL criterium
!------------------------------------------------------------------------------
      dt=1E30
      do i=0,nx+1
        do j=0,ny+1
          rho=u(1,i,j)
          vx=u(2,i,j)/rho
          vy=u(3,i,j)/rho
          P=(u(4,i,j)-0.5*rho*(vx**2+vy**2))*(gamma-1.)
          cs=sqrt(gamma*P/rho) !Speed of sound
          dt=min( dt,Co*dx/(abs(vx)+cs) )
          dt=min( dt,Co*dy/(abs(vy)+cs) )

        end do
      end do

      return
      end subroutine courant

\end{lstlisting}



\chapter{Condiciones de frontera} \label{aped.B}


Las condiciones de frontera se usará para obtener los valores de nuestras variables conservadas en los extremos de nuestra malla de puntos, con el fin de evitar errores numéricos, las condiciones de frontera que generalmente se usan son de cuatro tipos las de \textit{outflow}, las de \textit{reflexión}, las \textit{periódicas} y las de \textit{jet}. Las del tipo \textit{outflow} serán aquellas en las que una vez los valores sobre la malla (ondas) queden fuera de esta, ya no sabremos que pasó después con estos datos, las de \textit{reflexión} serán aquellas en las que nuestros datos en vez de salir se reflejarán y las \textit{periódicas} serán parecidas a las de \textit{reflexión} solo que en vez de reflejarse las ondas, estas entrarán del lado contrario de donde salieron  y las de tipo \textit{jet}, será para que de un lado de nuestra malla salga una fuente de partículas.

Para entender lo que son las condiciones de frontera, vamos suponer una malla de puntos, esta malla tendrá $n+2$ filas y $m+2$ columnas, para identificar los puntos vamos a indexarlos empezando desde el 0 hasta $n+1$ en el caso de las filas y de 0 hasta $m+1$ para el de las columnas.

\begin{figure}[H]
\centering
\includegraphics[width=0.5\textwidth]{./Figuras/malla.png}
\caption{Malla de puntos que resalta las fronteras, los puntos más oscuros representan representan una frontera \emph{fantasma}, es decir, ese conjunto de puntos estará allí como apoyo para resolver los cálculos que hará la computadora dado que tanto el método de Lax, como el método de HLL usan los puntos posteriores y anteriores en el espacio del punto que queremos saber su valor, pero no se graficarán} \label{fig: malla de puntos}
\end{figure}

La figura \ref{fig: malla de puntos} muestra los puntos más negros como unos puntos que nos van a servir de apoyo para calcular los valores de los puntos más grises, esto debido a que para calcular los valores de algún punto $(x_i, y_j)$ necesitamos el punto posterior $x_{i+1}, y_{j+1}$ y el punto anterior $x_{i-1}, y_{j-1}$.

Ahora, cuando llegamos a los últimos puntos de nuestro frontera, por ejemplo, el punto $x_{0}, y_{0}$, no podremos calcularlo debido a que no tendremos conocimiento acerca del punto anterior $x_{(-1)}, y_{(-1)}$, entonces tendremos que darles valores específicos a estos puntos, pero no podemos darles cualquier valor, en las siguientes secciones vamos a ver que valores válidos le podemos dar para el funcionamiento del código. 

\subsection{Condiciones de frontera \emph{outflow}}

Las condiciones \emph{outflow}, son en las que los valores de nuestra frontera fantasma(puntos negros) que toman los  mismos valores que su antecesor (puntos grises), los fenómenos físicos que atraviesan nuestra frontera, pasarán como si tuvieramos más dominio hacia afuera y una vez que salgan perderemos información sobre esta.

\begin{figure}[H]
\centering
\subfigure[onda expandiendose antes de cruzar la frontera]{\includegraphics[scale = 0.50]{./Figuras/Apendice/outflow1}}
\subfigure[onda expandiendose después de cruzar la frontera]{\includegraphics[scale = 0.50]{./Figuras/Apendice/outflow2}}
\caption{Al pasar la onda nuestra frontera, esta sigue su trayecto normal como si el dominio fuera infinito} \label{fig:reflexion}
\end{figure}

Las ecuaciones que obedece nuestra frontera son las siguientes:
\begin{eqnarray}
\textbf{U}(0,j)&=&\textbf{U}(1,j) \\
\textbf{U}(n+1,j)&=&\textbf{U}(n,j) \\
\textbf{U}(i,0)&=&\textbf{U}(i,1) \\
\textbf{U}(i,m+1)&=&\textbf{U}(i,m) 
\end{eqnarray}

Donde $i,j \leq n,m \in \mathbb{N}$

\subsection{Condiciones de frontera \emph{reflexión}}

Las condiciones de reflexión funcionan como una pared en la que no se le permite al fenómeno físico escapar y al toparse con estas fronteras se reflejarán. Los valores que tendrán los puntos negros, serán los valores negativos de los puntos grises.

\begin{figure}[H]
\centering
\subfigure[onda expandiendose antes de llegar a la frontera]{\includegraphics[scale = 0.50]{./Figuras/Apendice/reflexion1}}
\subfigure[onda reflejandose en las fonteras]{\includegraphics[scale = 0.50]{./Figuras/Apendice/reflexion2}}
\caption{Al llegar la frontera la onda se reflejará y chocará con la misma} \label{fig:reflexion}
\end{figure}

Las ecuaciones que obedece nuestra frontera son las siguientes:
\begin{eqnarray}
\textbf{U}(0,j)&=&-\textbf{U}(1,j) \\
\textbf{U}(n+1,j)&=&-\textbf{U}(n,j) \\
\textbf{U}(i,0)&=&-\textbf{U}(i,1) \\
\textbf{U}(i,m+1)&=&-\textbf{U}(i,m) 
\end{eqnarray}


\subsection{Condiciones de frontera \emph{Periódicas}}
Las condiciones periódicas son cuando nuestra frontera fantasma toma los valores de la frontera opuesta del lado que estan, es decir, si un fenómeno físico pasa a través de la parte de arriba de nuestro dominio (ver figura \ref{fig:periodicas}), este, saldrá por la parte de abajo y viceversa, lo mismo aplica para los fenómenos pasen por la parte izquierda o derecha de nuestro dominio.
 
\begin{figure}[H]
\centering
\subfigure[Onda antes de tocar la frontera de abajo]{\includegraphics[scale = 0.50]{./Figuras/Apendice/periodicas1}}
\subfigure[La onda pasa la frontera de abajo y sale por la parte de arriba]{\includegraphics[scale = 0.50]{./Figuras/Apendice/periodicas2}}
\caption{Al llegar la onda abajo se puede ver que se transporta al lado de arriba, se eligió esa posición de la onda, solo para resaltar la periodicidad de la onda, ya que si se huebiera puesto en el centro, la onda chocaría contra si mismo y no podriamos ver el paso de la onda.} \label{fig:periodicas}
\end{figure}

Las ecuaciones que representan este tipo de frontera son las siguientes:

\begin{eqnarray}
\textbf{U}(0,j)&=&\textbf{U}(n,j) \\
\textbf{U}(n+1,j)&=&\textbf{U}(1,j) \\
\textbf{U}(i,0)&=&\textbf{U}(i,m) \\
\textbf{U}(i,m+1)&=&\textbf{U}(i,1) 
\end{eqnarray}


\subsection{Condiciones de frontera \emph{Jet}}

Las condiciones de tipo jet, son en las que dado un lado de nuestra frontera (pueden se varios), se va a inyectar un energía y masa a una cierta velocidad constante todo el tiempo en una de las partes de la frontera.

 
\begin{figure}[H]
\centering
\subfigure[La densidad del medio sin la inyección del jet]{\includegraphics[scale = 0.50]{./Figuras/Apendice/Jet1}}
\subfigure[Jet inyectandose en el medio]{\includegraphics[scale = 0.50]{./Figuras/Apendice/Jet2}}
\caption{El jet, es basicamente inyectar masa y energía en una parte de nuestra frontera, a cada tiempo que evoluciona.} \label{fig:periodicas}
\end{figure}

Para obtener las ecuaciones de esta frontera, igualamos los valores de la frontera \emph{fantasma} con las variables primitivas de nuestro jet, las siguientes ecuaciones son para la parte de abajo de nuestro dominio.

No relativista

\begin{eqnarray}
\textbf{U}(1,0,j)&=&\rho_{jet} \\
\textbf{U}(2,0,j)&=& \rho_{jet} v_{x_{jet}}\\
\textbf{U}(3,0,j)&=& \rho_{jet} v_{y_{jet}}\\
\textbf{U}(4,0,j)&=& E_{jet}
\end{eqnarray}

Relativista

\begin{eqnarray}
\textbf{U}(1,0,j)&=&\rho_{jet} \gamma \\
\textbf{U}(2,0,j)&=& \rho_{jet} v_{x_{jet}} \gamma^2 h \\
\textbf{U}(3,0,j)&=& \rho_{jet} v_{y_{jet}} \gamma^2 h \\
\textbf{U}(4,0,j)&=& \rho_{jet} \gamma^2 h-P
\end{eqnarray}

donde $j\in \left[ a,b \right]$ y $a\geq 0, \, b\leq n+1$. Los puntos de la frontera que no sean del jet se pueden combinar con los 3 tipos de frontera mencionados anteriormente.


\chapter{Condiciones de Rankine-Hugoniot no relativistas}\label{aped.C}
Las ecuaciones de Rankine-Hugoniot parten de las ecuaciones de la hidrodinámica considerando un sistema cerrado usando las ecuaciones \ref{conservación_masa_hidrodinamica}, \ref{conservacion_momento_hidrodinamica} y \ref{conservacion_energia_hidrodinamica} donde no varia con el tiempo, es decir, que $\dfrac{\partial}{\partial t}=0$, con lo que se pueden reescribir de la siguiente manera:


La conservación de masa
\begin{equation}
 \nabla \cdot \left( \rho \mathbf{u} \right)=0
\end{equation}

El momento
\begin{equation}
 \nabla \cdot \left( \rho \mathbf{u u} \right) + \nabla p = 0
\end{equation}

Ecuación de la energía

\begin{equation}
 \nabla \cdot \left[ \mathbf{u} \left( E+P \right) \right] = 0
\end{equation}

Considerando una dimensión:

\begin{equation}
\dfrac{d \left( \rho u \right)}{d x} = 0
\end{equation}

\begin{equation}
\dfrac{d \left( \rho u^2 \right)}{d x}+ \dfrac{d P}{d x}=0
\end{equation}

\begin{equation}
\dfrac{d \left( u\left[E+P \right] \right)}{d x} = 0
\end{equation}

Si integramos, la ecuaciones se van a igualar a constantes y podemos reescribir las ecuaciones del siguiente modo usando la ecuación de estado $E = \frac{1}{2} \rho u^2 + \frac{P}{\Gamma-1}$ donde $E \left[\frac{\mathrm{g}}{\mathrm{cm} \cdot \mathrm{s}^2}\right]$ es la densidad de energía por unidad de volumen

\begin{equation}\label{RH_masa}
\rho_j u_j = \rho_m u_m
\end{equation}

\begin{equation}\label{RH_momento}
\rho_j u_{j}^{2}+P_j = \rho_m u_{m}^{2}+P_m
\end{equation}

\begin{equation}\label{RH_Energia}
\frac{1}{2} u_{j}^{2}+ \frac{\Gamma}{\Gamma-1} \frac{P_{j}}{\rho_{j}} =
 \frac{1}{2} u_{m}^{2}+ \frac{\Gamma}{\Gamma-1} \frac{P_{m}}{\rho_{m}}
\end{equation}

Donde $\rho \left[\frac{\mathrm{g}}{\mathrm{cm}^3}\right]$ representa la densidad, $u \left[\frac{\mathrm{cm}}{\mathrm{s} }\right]$ la velocidad, $P \left[\frac{\mathrm{g}}{\mathrm{cm} \cdot \mathrm{s}}\right]$ la presion, $\Gamma$ el indice adiabático adimensional donde para velocidades ultrarrelativistas $\Gamma = 4/3$ y para no relativistas $\Gamma = 5/3$ y los índices \textit{j, m} que  relacionan a las propiedades del jet y del medio respectivamente. Usando la ecuacion \ref{RH_masa}, podemos definir el flujo como $j \equiv \rho_j u_j = \rho_m u_m$, sustituyendo en la ecuación \ref{RH_momento} podemos reescribirla como:

\begin{equation}\label{RH_momento_j}
P_{j}+\frac{j^2}{\rho_{j}}=P_{m}+\frac{j^2}{\rho_{m}}
\end{equation}

y la ecuación \ref{RH_Energia} llegamos a:

\begin{equation}\label{RH_Energia_j}
\frac{1}{2} \frac{j^{2}}{\rho_{j}^2}+\frac{\Gamma}{\Gamma-1} \frac{P_{j}}{\rho_{j}}=
\frac{1}{2} \frac{j^{2}}{\rho_{m}^2}+\frac{\Gamma}{\Gamma-1} \frac{P_{m}}{\rho_{m}}
\end{equation}

Despejando $j$ de la ecuación \ref{RH_momento_j} obtenemos:
\begin{equation}\label{j^2}
-j^{2}=\frac{P_{j}-P_{m}}{\frac{1}{\rho_{m}}-\frac{1}{\rho_{j}}}
\end{equation}

Ahora sustituyendo la ecuación en \ref{j^2} en \ref{RH_Energia_j} obtenemos:

\begin{equation*}
\frac{1}{2} \left( \frac{P_{m}-P_{j}}{\frac{1}{\rho_{j}}-\frac{1}{\rho_{m}}} \right)
\left(\frac{1}{\rho_{j}^{2}}-\frac{1}{\rho_{m}^{2}} \right)
=
\frac{\Gamma}{\Gamma-1}
\left( \frac{P_{m}}{\rho_{m}}-\frac{P_{j}}{\rho_{j}} \right)
\end{equation*}

$\Rightarrow$

\begin{equation*}
\frac{1}{2}	\left( P_{m} - P_{j} \right)
\left( \frac{1}{\rho_{j}}+\frac{1}{\rho_{m}} \right)
=
\frac{\Gamma}{\Gamma-1}
\left( \frac{P_{m}}{\rho_{m}}-\frac{P_{j}}{\rho_{j}} \right)
\end{equation*}

$\Rightarrow$

\begin{equation*}
\frac{1}{\rho_{m}} \left( \frac{1}{2} P_{m}- \frac{1}{2} P_{j}-
\frac{\Gamma}{\Gamma-1} P_{m} \right)
=
\frac{1}{\rho_{j}} \left( \frac{1}{2} P_{j}- \frac{1}{2} P_{m}-
\frac{\Gamma}{\Gamma-1} P_{j} \right)
\end{equation*}

$\Rightarrow$

\begin{equation*}
\frac{1}{\rho_{m}} \left[  \left(\frac{\Gamma + 1}{\Gamma - 1} \right) P_{m} + P_{j} \right]
=
\frac{1}{\rho_{j}} \left[  \left(\frac{\Gamma + 1}{\Gamma - 1} \right) P_{j} + P_{m} \right]
\end{equation*}

Con lo que nos queda:

\begin{equation}\label{RH_no_rel_choque_no_fuerte}
\frac{\rho_{m}}{\rho_j} =
\frac{\left( \Gamma +1 \right) P_{m}+ \left( \Gamma -1 \right) P_{j
}}{\left(\Gamma +1 \right) P_{j}+ \left( \Gamma -1 \right) P_{m}}
= \frac{u_j}{u_m}
\end{equation}

Si consideramos choque fuerte, es decir, $P_j \gg P_m$, implicaría que $P_m \simeq 0$, por lo que

\begin{equation}
\rho_j = \frac{\Gamma +1}{\Gamma-1} \rho_m 
\end{equation}
Tomando a $\Gamma = 5/3$ da

\begin{equation}
\rho_j = 4 \rho_m
\end{equation}

\begin{equation}
u_j = \frac{1}{4} u_m
\end{equation}

\begin{equation}
P_{j} = \frac{3}{4}\rho_m u_m^{2}
\end{equation}

\chapter{Condiciones de Rankine-Hugoniot relativistas}\label{aped.C}

Para poder encontrar las relaciones de Rankine-Hugoniot, usaremos la siguiente ecuación de estado:
\begin{equation}\label{EoS_relativista}
\Gamma = 1+\frac{P}{\rho \epsilon}
\end{equation}
Donde $\epsilon \left[ \frac{s^2}{\mathrm{cm}^2}\right]$ es la energía interna del sistema\\
\textit{Ecuación de conservación de número de partículas o de masa}:
\begin{equation}\label{Ecuacion_particulas_relativista}
n_j u_j = n_m u_m
\end{equation}
Donde $n \left[ \mathrm{cm}^{-3}\right]$ es la densidad del número de partículas y $u$ es la cuadrivelocidad donde $u=\beta \gamma$, teniendo en cuenta que $\beta = \frac{v}{c}$ y $\gamma = \left( 1 - \beta \right)^{-1/2}$ es el factor de lorentz.\\

Ecuación de momento: 
\begin{equation}\label{Ecuacion_momento_relativista}
\omega_j u_{j}^{2}+P_j=\omega_j u_{j}^{2}+P_j
\end{equation}
Donde $\omega = e + P$ es la entalpia específica por unidad de volumen, donde $e$ es la densidad de energía interna, usando la ecuación \ref{EoS_relativista}  que puede ser escrita como:

\begin{equation}\label{Energia_densidad_fluido}
e = \rho \left( c^2+ \epsilon \right)= nm_{0}c^{2} + \frac{\Gamma}{\Gamma -1}P
\end{equation}

Donde $m_0$ es la masa en reposo y $c$ es la velocidad de la luz

Ecuación de la Energía

\begin{equation}\label{Ecuacion_energia_relativista}
\omega_j \gamma_j u_j=\omega_m \gamma_m u_m
\end{equation}

Definiendo una nueva variable $x=\frac{\omega}{n^{2}}$ y usando la ecuación \ref{Ecuacion_particulas_relativista} tenemos $\overline{j}=n_j u_j = n_m u_m$, por lo que usando  la ecuación \ref{Ecuacion_momento_relativista} la podemos escribir como:

\begin{equation*}
n_{j}^2 x_{j} u_{j}^{2} + P_{j}
=
n_{m}^2 x_{m} u_{m}^{2} + P_{m}
\end{equation*}

$\Rightarrow$

\begin{equation}
x_{j} \overline{j}^{2}+P_{j}
= 
x_{m} \overline{j}^{2}+P_{m}
\end{equation}

Con lo que al final nos queda:
\begin{equation} \label{j_modificada_relativista}
-\overline{j}^{2}=\frac{P_{m}-P_{j}}{x_{m}-x_{j}}
\end{equation}

Ahora usando la ecuación \ref{Ecuacion_energia_relativista}
podemos reescribirla como:

\begin{equation*}
\frac{\omega_j}{n_j}\gamma_j \overline{j}= \frac{\omega_m}{n_m}\gamma_m \overline{j}
\end{equation*}

$\Rightarrow$

\begin{equation} \label{Ecuacion_energia_relativista_mod}
\frac{\omega_m^2}{n_j^2}\gamma_m^2- \frac{\omega_m^2}{n_j^2}\gamma_j^2= 0
\end{equation}

La ecuación \ref{Ecuacion_energia_relativista_mod} la usaremos mas tarde, ahora volviendo a \ref{j_modificada_relativista} la podemos multiplicar por  $x_m^2-x_j^2$ queda lo siguiente:

\begin{equation*}
- \left( x_m^2 \overline{j}^{2}-x_j^2 \overline{j}^{2} \right) =
\frac{P_m-P_j}{x_m-x_j} \left( x_m-x_j \right) \left( x_m+x_j \right)
\end{equation*}

$\Rightarrow$

\begin{equation} \label{Ecuacion_A}
-x_m^2 u_m^2 n_m^2 + x_j^2 u_j^2 n_j^2 =
\left( P_m-P_j \right) \left( x_m+x_j \right) 
\end{equation}

restando la ecuación \ref{Ecuacion_energia_relativista_mod} de la \ref{Ecuacion_A}

\begin{equation}
-\left( \frac{\omega_{m}}{n_m^2}\right)^2 u_m^2 n_m^2 +
\left( \frac{\omega_{j}}{n_j^2}\right)^2 u_j^2 n_j^2 -
\frac{\omega_m^2}{n_j^2}\gamma_m^2 +
\frac{\omega_m^2}{n_j^2}\gamma_j^2
=
\left( P_m-P_j \right) \left( x_m+x_j \right) 
\end{equation}

$\Rightarrow$

\begin{equation} \label{ec_mod_beta-lor}
-\frac{\omega_{m}}{n_m^2} \left( u_m^2-\gamma_m^2 \right)
+\frac{\omega_{j}}{n_j^2} \left( u_j^2-\gamma_j^2 \right)
=\left( P_m-P_j \right) \left( x_m+x_j \right) 
\end{equation}


Sabemos que:

\begin{equation}
u = \beta \gamma
\end{equation}

$\Rightarrow$

\begin{equation}
u^2 = \beta^2 \gamma^2 
\end{equation}

$\Rightarrow$

\begin{equation}
u^2 = \frac{\beta^2}{1-\beta^2} 
\end{equation}

$\Rightarrow$

\begin{equation}
u^2-\gamma^2 = \frac{\beta^2}{1-\beta^2}-\frac{1}{1-\beta^2}
\end{equation}

Entonces al final:
\begin{equation} \label{beta-lor}
u^2-\gamma^2 = -1
\end{equation}

Sustituyendo la ecuación \ref{beta-lor} a la \ref{Ecuacion_energia_relativista_mod} nos queda:

\begin{equation} \label{Choque adiabatico}
x_m \omega_m- x_j \omega_j = \left( P_m-P_j \right) \left( x_m+x_j \right) 
\end{equation}

Podemos reescribir a $x$ usando la ecuación  como \ref{Energia_densidad_fluido} como:
\begin{equation} \label{var_x_complicada}
x= \frac{m_0^2 c^2}{\rho}+ \frac{\Gamma}{\Gamma-1}\frac{m_0^2 P}{\rho^2}
\end{equation}

Sustituyendo la ecuación \ref{var_x_complicada} en \ref{j_modificada_relativista}


\begin{thebibliography}{20}


\bibitem{Berger:2013jza} 
  E.~Berger,
  %``Short-Duration Gamma-Ray Bursts,''
  Ann.\ Rev.\ Astron.\ Astrophys.\  {\bf 52}, 43 (2014)
  doi:10.1146/annurev-astro-081913-035926
  [arXiv:1311.2603 [astro-ph.HE]].
  %%CITATION = doi:10.1146/annurev-astro-081913-035926;%%
  %424 citations counted in INSPIRE as of 02 Jan 2019
 
%\bibitem{2012ApJ...760..122G} Gao, Y., \& Law, C.~K.\ 2012, \apj , 760, 122

 
\end{thebibliography}

\end{document}