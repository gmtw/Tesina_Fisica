\documentclass[12pt,a4paper]{book}
\usepackage[utf8]{inputenc}	
\usepackage[spanish]{babel}
\usepackage{amsmath}
\usepackage{amsfonts}
\usepackage{amssymb}
\usepackage{graphicx}
\usepackage{fourier}
\usepackage[left=2.7cm,right=2.7cm,top=2.7cm,bottom=2.7cm]{geometry}
\usepackage{hyperref} %Paquete para agregar hipervínculos

\usepackage{subfigure} % subFiguras
\usepackage{float} %fijar Figuras
\usepackage{listings} %este paquete esta agregado para que se pueda agregar lineas de código de programa (en este caso FORTRAN) a nuestro documento
\usepackage{setspace}



\spanishdecimal{.}

%\usepackage{cancel} %cancelar términos
%\usepackage[backend=biber]{biblatex}
%\addbibresource{biblio.bib}

\providecommand{\abs}[1]{\lvert#1\rvert} %agregación para el valor absoluto
\usepackage{xcolor}
\usepackage{graphicx}
%\usepackage{subfigure} % subFiguras
\lstset{language=Fortran, %nuestro lenguaje fortran por su pollo
backgroundcolor=\color{white}, %color de fondo blanco para eso es que ocupe el paquete color
basicstyle=\footnotesize, % Fija el tamaño del tipo de letra utilizado para el código
breakatwhitespace=false,% Activarlo para que los saltos automáticos solo se apliquen en los espacios en blanco
breaklines=true,
commentstyle=\color{blue}, %color de los comentarios del código
keepspaces=true, 
rulecolor=\color{black}, % Si no se activa, el color del marco puede cambiar en los saltos de línea entre textos que sea de otro color
keywordstyle=\color{red},% estilo de las palabras clave
stringstyle=\color{yellow},% Estilo de las cadenas de texto
stringstyle=\color{orange}, 
}
\lstset{numbers=left, numberstyle=\tiny, stepnumber=1, numbersep=-2pt}

\author{Julio César Sosa Mondragón}

\begin{document}

\begin{minipage}{.3\textwidth}
  
  \flushleft
  \center{\includegraphics[scale=.09]{unam.pdf}}

  \vspace{20pt}

  \center{
    \rule{.5pt}{.6\textheight}
    \hspace{7pt}
    \rule{2pt}{.6\textheight}
    \hspace{7pt}
    \rule{.5pt}{.6\textheight}
         } \\

\center{\includegraphics[scale=.22]{ciencias.pdf}}
\end{minipage}
\begin{minipage}{.7\textwidth}

\flushright

\center{

  \center{
    \LARGE{U}\large{NIVERSIDAD} \LARGE{N}\large{ACIONAL} 
    \LARGE{A}\large{UTÓNOMA} \\[10pt]
    \large{DE} 
    \LARGE{M}\large{ÉXICO} 
  } \\
  \rule{\textwidth}{2pt}
  \\
  \hrulefill\\[1cm]
  
  \LARGE{F}\large{ACULTAD DE } \LARGE{C}\large{IENCIAS}\\[2cm]

  \large{
Simulaciones numéricas hidrodinámicas relativistas de Destellos de Rayos Gamma cortos lanzados desde un objeto compacto en varios tipos de medios ambientes. }\\[2cm]

  \huge{
T \hspace{1cm} E \hspace{1cm} S \hspace{1cm} I \hspace{1cm} S  }\\[1cm]

  \large{QUE PARA OBTENER EL TÍTULO DE:}\\[1cm]

  \large{
Físico  }\\[1cm]

  \large{PRESENTA:}\\[1cm]

  \large{
Julio César Sosa Mondragón  }\\[1cm]

  \large{
TUTOR  }\\[1cm]

  \large{
Dr. Diego López Cámara Ramírez}
}

\end{minipage}
\thispagestyle{empty} % para que no se numere esta pagina


\pagenumbering{Roman} % para comenzar la numeracion de paginas en numeros romanos
\tableofcontents % indice de contenidos

\onehalfspace

%
%%\cleardoublepage
%%\addcontentsline{toc}{chapter}{Lista de Figuras} % para que aparezca en el indice de contenidos
%%\listoffigures % indice de Figuras
%
%%\cleardoublepage
%%\addcontentsline{toc}{chapter}{Lista de tablas} % para que aparezca en el indice de contenidos
%%\listoftables % indice de tablas
%
%%%%%%%%%%%%%% DEDICATORIA %%%%%%%%%%%%%%%%%%%%%%%%%%%%%%%%%%%
\chapter*{Dedicatoria}

\addcontentsline{toc}{chapter}{Dedicatoria}
%
\begin{flushright}
\textit{Dedicado a \\
mi familia}
\end{flushright}

%%%%%%%%%%%%%%%%%%%%%%%%%%%%%%%%%%%%%%%%%%%%%%%%%%%%%%%%%%%%%%%%%%
%
\chapter*{Agradecimientos} % si no queremos que añada la palabra "Capitulo"
\addcontentsline{toc}{chapter}{Agradecimientos} % si queremos que aparezca en el índice
\markboth{AGRADECIMIENTOS}{AGRADECIMIENTOS} % encabezado
 
¡Muchas gracias a todos!
%%%%%%%%%%%%%%%%%%%%%%%%%%%%%%%%%%%%%%%%%%%%%%%%%%%%%%%%%%%%%%%%%%%%%

\chapter*{Resumen} % si no queremos que añada la palabra "Capitulo"
\addcontentsline{toc}{chapter}{Resumen} % si queremos que aparezca en el índice

\markboth{RESUMEN}{RESUMEN} % encabezado
%
Una bonita historia
%
%%%%%%%%%%%%%%%  CAPITULOS %%%%%%%%%%%%%%%%%%%%%%%%%%%%%%%%%%55
%
%
%----------------------------------------------------------------------------------
\chapter{Introducción}
%----------------------------------------------------------------------------------
\pagenumbering{arabic} % para empezar la numeración con números
%Érase una vez...

%----------------------------------------------------------------------------------
\section{Jets relativistas}
%----------------------------------------------------------------------------------
En astrofísica, los jets son uno de los fenómenos más energéticos del universo en donde concentra masa, energía, momento y flujo magnético desde los objetos estelares, 
galácticos y extragalácticos hacia el medio exterior. Son los conductos que conectan los agujeros negros supermasivos y sus discos de acreción con las galaxias anfitrionas y más allá del universo.
Estos tienen forma de protuberancias cónicas o cilíndricas 
o semicilíndricas estrechas con ángulo de apertura pequeño. Sus principales características es que tienen una 
amplia gama de luminosidad y grado de colimación. 
Los ejemplos más poderosos observados que emergen de los núcleos de galaxias activas (o AGN). Tambien se encuentran asociados a otros objetos estelares tales como la fusion de agujeros negros y/o
estrellas de neutrones, sistemas binarios, estrellas simbióticas y microquasares. %! ASTROPHYSICAL JETS AND OUTFLOWS Elisabete M. de Gouveia Dal Pino

%----------------------------------------------------------------------------------
% Acá mete lo de aguayo de los jets

{\color{blue} Los jets y los flujos de salida son fenómenos naturales en los que la materia, la energía y otras cantidades físicas (como el flujo magnético) se expulsan a un medio alrededor de un cuerpo central gigante.
Parecen ser un evento astrofísico omnipresente, ya que ocurren en una amplia gama de entornos astrofísicos.
Desde objetos estelares hasta extragalácticos, con velocidades que van desde unas unos pocos $m \cdot s^{-1}$ hasta velocidades cercanas a la de la luz, y longitudes que van desde unas pocas unidades astronómicas 
hasta kiloparsecs, todos vinculados por algún tipo de proceso de acreción.
Un jet se diferencia de un flujo de salida en que el primero se refiere a una eyección altamente colimada, mientras que la segunda es menos colimada y más lenta.

En comparación con sus contrapartes menos colimados, los jets altamente colimados se encuentran en entornos más particulares. Los jets son impulsados por la acreción de un agujero negro en regímenes relativistas, 
ya sea un agujero negro de masa estelar como en las binarias de rayos X o un agujero negro supermasivo como en los AGN. Los estallidos cortos de rayos gamma, 
que resultan de la colisión de dos sistemas binarios de estrellas de neutrones, y los GRB largos, que resultan del colapso de una estrella masiva que gira rápidamente en sus últimas etapas de vida, 
contienen jets relativistas.

Aunque la presencia de un objeto central masivo y un disco de acreción de gas giratorio a su alrededor es el componente principal en todos los escenarios astrofísicos donde se encuentran jets y flujos de salida, 
el campo magnético y su interacción con todo el sistema parece ser un componente fundamental en la necesidad de estas eyecciones.
Esto permite la formación y estabilidad del disco, así como el lanzamiento y colimación del chorro. Los procesos más aceptados para el lanzamiento del jet son los mecanismos magnetorrotacionales propuestos 
por REF1 y REF2}
%----------------------------------------------------------------------------------


Los jets de AGNs se forman cuando el agujero negro, con una masa de $10^6$-$10^{10}$ masas solares, gira y atrae masa de la galaxia que lo rodeas el cual forma un disco de acreción
que gira ortogonal al eje de rotación del agujero negro debido a la conservación del momento angular y que contiene un fuerte campo magnético. %!Search for Signatures of Extra-Terrestrial Neutrinos with a Multipole Analysis of the AMANDA-II Sky-Map 2007-07-03 - 2007-07-11
Las propiedades de los jets también abarcan rangos muy amplios. Un caso particular son cuando los jets de AGN pueden permanecer bien colimados a distancias de cientos de kiloparsecs. %!The content of astrophysical jets


%\begin{figure} 
%  %!Search for Signatures of Extra-Terrestrial Neutrinos with a Multipole Analysis of the AMANDA-II Sky-Map 2007-07-03 - 2007-07-11
%    \centering
%      \includegraphics[width=0.3\textwidth]{Figuras/Introduccion/Schematic-picture-of-an-AGN-with-a-jet-Different-types-of-accelerated-particles-and.png}
%    \caption{Figura esquemática de un un AGN con un jet el cual contiene diferentes tipos de particulas aceleradas debido al campo magnético producido por el disco de acreción}
%    \label{fig:Schematic-picture-of-an-AGN-with-a-jet-Different-types-of-accelerated-particles-and}
%\end{figure}


Los destellos de rayos gamma (GRB por su acrónimo en inglés) son eyecciones de rayos gamma del orden de Mev, son cortos, intensos y no repetitivos.
Estos consisten en la emisión de energías altas como los rayos $\gamma$ y los rayos X, asi como energías bajas como el óptico, el radio, entre otros \cite{PGRB-piran}. %!Piran2005
% Fueron descubiertos por los satélites \emph{Vela}\footnote{\url{https://heasarc.gsfc.nasa.gov/docs/heasarc/missions/vela5a.html}} a finales de la decada de los 60's, gracias a los datos recabados por el experimento BATSE\footnote{\url{https://gammaray.msfc.nasa.gov/batse/}} 
% se logró identificar 2 grupos principales de GRBs, los cortos y los largos, en los cuales este último tiene una duracion menor a los 2 s \cite{Berger:2013jza}.

% \begin{figure} 
% %!fuente de la imagen: https://imagine.gsfc.nasa.gov/science/objects/bursts1.html
%   \centering
%     \includegraphics[width=0.5\textwidth]{Figuras/burst_durations_labelled.jpg}
%   \caption{Gráfico que muestra la diferencia de tiempo que existe entre GRBs cortos y GRBs largos. Se pueden apreciar 2 picos que marcan la diferencia de duracion entre los GRBs}
%   \label{fig:Batse_duration_GRBs}
% \end{figure}

La fusión de objetos compactos es el modelo progenitor más atractivo.
% Las categorías de los GRBs largos y cortos son aproximadamente 75\% y 25\% de la población \cite{GRB:PPP}. %!Gamma Ray Burst Progress, problems and prospects Zhang, meszaros
% La interacción de flujos de salida relativistas con el medio ambiente que lo rodea produce emisión sincrotrónica que va desde la banda de las ondas de radio a los rayos X.
Los jets de los  GRBs son uno de los eventos que más libera energía en el universo y su energía liberada es del orden de $10^{52}$ ergs \cite{Berger:2013jza}.
Se infiere que los jets relativistas de los blázares tienen un factor de Lorentz de hasta $\Gamma \thicksim 50$. 
Pero en el modelo más ampliamente aceptado para GRB, una explosión altamente enfocada asociada con la formación de un agujero negro impulsa jets relativistas colimados con $\Gamma \lesssim 400$ 
%!A Simulation Study of Ultra-Relativistic Jets - I. A New Code for Relativistic Hydrodynamics

Los ángulos de estos jets que vienen tanto de GRBs cortos como largos tienen, en su mayoría, 5° de apertura, el cual puede variar hasta los 35° con un promedio de 
$\langle \theta_j \rangle \gtrsim 10$.
En el medio intergaláctico dentro de las proximidades de las galaxias y dentro de los cúmulos de galaxias se tienen densidades numéricas que oscilan entre $5 \times 10^{-6}$ a 
$ \sim 10^3 \, \text{cm}^{-3}$. Con las ecuaciones de Rankine-Hugoniot se pueden conocer las densidades del jet a partir de sus velocidades y el medio ambiente que lo rodea, para mas información 
vea la sección \ref{sec:Ecuacion_Rankine_Hugoniot}
%! Relativistic AGN jets I. The delicate interplay between jet structure,cocoon morphology and jet-head propagation



\begin{table}
  \begin{center}
    \begin{tabular}{ c c c c c } 
      \hline
      Propiedad                                   & GRBs                          & Microquasares                    & YSO                            & AGN                 \\
      \hline
      Masa del acretor [$\text{M}_{\odot }$]      & $\thicksim 10$                &    $\thicksim 10$                &  $\thicksim 1-10$              & $\thicksim 10^{6}-10^{9}$                       \\ 
      Tamaño del acretor [cm]                     & $\thicksim 10^{6}$            &    $\thicksim 10^{6}$            &  $\thicksim 10^{11}$           & $\thicksim 10^{11}-10^{15}$                      \\ 
      Máximo campo magnético [G]                  & $\thicksim 10^{16}$           &    $\thicksim 10^{7}$            &  $\thicksim 10^{3}$            & $\thicksim 10^{3}-10^{5}$                         \\ 
      Energía del jet [$\text{erg} \, s^{-1}$]    & $\thicksim 10^{50}-10^{52}$   &    $\thicksim 10^{37}-10^{40}$   &  $\thicksim 10^{32}-10^{36}$   & $\thicksim 10^{42}-10^{46}$                        \\ 
      Tiempo de vida [años]                       & $\thicksim 10^{-6}$           &    $\thicksim 10^{4}-10^{6}$     &  $\thicksim 10^{4}-10^{5}$     & $\thicksim 10^{7}-10^{8}$                           \\ 
      Tamaño del jet [pc]                         & $\thicksim 10$                &    $\thicksim 10$                &  $\thicksim 1$                 & $\thicksim 10^{5}$                                   \\ 
    \end{tabular}
  \caption{Propiedades de los jet de distintas fuentes. Se puede observar que, dependiendiendo de su origen, sus propiedades pueden variar considerablemente} \label{table:propiedades_jets}
  \end{center}

\end{table}
  
%\begin{figure} 
  %!fuente de la imagen: https://imagine.gsfc.nasa.gov/science/objects/bursts1.html
%    \centering
%      \includegraphics[width=0.5\textwidth]{Figuras/Introduccion/Distribucion_angulo_apertura.png}
%    \caption{Distribucion de los ángulos de apertura para GRBs largos (azul) y cortos (rojos) basados en su emisión tardía y su \emph{jet break}}
%    \label{fig:Distribucion_angulo_apertura}
%\end{figure}
  
  
%--------------------------------------------------------------------------
\section{Estudios previos de jets relativistas}
%--------------------------------------------------------------------------

%--------------------------------------------------------------------------
%Ejemplo de estudio de jets tomando en cuenta distintos métodos numéricos
%--------------------------------------------------------------------------
En el pasado se han realizado estudios de jets relativistas tomando en cuenta distintos métodos numéricos \cite{SURJ-I, MB-HLLC-I}, en el marco de AGNS (REFS), así como en el marco de GRBs (REFs), 
por mencionar algunos ejemplos. Un ejemplo de simulaciones de jets relativistas se presenta en \cite{SURJ-I} donde se estudia la evolución de un jet relativista inyectado horizontalmente en un medio 
uniforme utilizando distintos métodos numéricos (Weno JS, Weno Z, y Weno ZA). El jet el cilíndrico, tiene una densidad de $\rho_{jet} = 10^{-2}$, una velocidad $v_{x_{\text{jet}}},v_{y_{\text{jet}}} = 0.99, 0$ y 
una presión $p_{\text{jet}} = 10^{-3}$, donde la $v_{x_{\text{jet}}} = 0.99$ representa un factor de Lorentz $\gamma = 7$. El medio ambiente es estático, tiene una densidad $\rho_m = 1$ y una presión igual al del jet. 
En la Figura \ref{fig:jet_ejemplo_2} se muestra la configuración final utilizando los tres diferentes métodos numéricos.

{\color{blue} En la parte  de arriba se muestra un jet con un tamaño de 3.15 unidades, usando el método de WENO JS. El ancho del jet mide 0.2 unidades y presenta una turbulencia similar al que usa WENO Z.
La gráfica de en medio muestra un jet con longitud un poco mayor de 3.2 unidades y un ancho mayor de 0.21 unidades. Para el caso de la gráfica de abajo, donde usa el método de WENO ZA, el largo del jet, mide 3.21 unidades
siendo el más largo de los tres  y mostranto una mayor turbulencia que los 2 anteriores. Su ancho es un poco menor ya que mide 1.8 unidades. }



\begin{figure}
    \begin{center}
      \includegraphics[width=0.95\textwidth]{Figuras/Introduccion/jet_ejemplo_2.png}
    \end{center}
    \caption{Mapa de densidad ($\log \rho$) del jet con CFL = 0.8, con un dominio computacional $\left[0, 3.5\right] \times \left[0, 1\right]$, con una resolución de 1050 X 300 píxeles.}
    \label{fig:jet_ejemplo_2}
\end{figure}



%--------------------------------------------------------------------------
%Ejemplo de estudio de jets en el marco de GRBs.
%--------------------------------------------------------------------------
Otro ejemplo de un simulaciones de jets relativistas es el estudio de \cite{JBEMGRB} en el cual se estudió la evolución de jets que son lanzados tras la fusión de estrellas de neutrones así como en el 
marco de las Colapsares \cite{JBEMGRB}. 
En la Figura~\ref{fig:jet_models} se muestra la evolución de distintos jets relativista a través del medio producido tras la fusión de dos estrella de neutrones o de la envolvente del Colapsar. 
\begin{figure} 
  \centering
    \includegraphics[width=0.7\textwidth]{Figuras/Introduccion/jet_models.png}
  \caption{Mapa de densidad de el jet, donde los paneles de arriba muestran ambientes de una fusión de estrellas de neutrones mientras que los paneles de abajo muestran para los Colapsares.}
  \label{fig:jet_models}
\end{figure}

{\color{blue} El jet que se lanza tras la fusión de las dos estrellas de neutrones (paneles de arriba) tiene una luminosidad de $L_{\rm iso, 0} = 5.0 \times10^{50}$~erg~s$^{-1}$, 
un ángulo de apertura de $\theta_0=6.8$ (panel de arriba a la izquierda) o de $\theta_0=18.8$ (panel de arriba a la derecha), 
es lanzado desde una distancia $r_{in}=1.2\times10^{8}$~cm, y atraviesa un medio con masa es de $M_{m} = 0.02 M_{\odot}$ que se mueve con velocidad $v_{m} = 0.34 c$ (ver Cuadro~\ref{table:jet_models} 
para más detalles). El jet que se lanza dentro del Colapsar (paneles de abajo) tiene una luminosidad de $L_{\rm iso, 0} = 7.83 \times10^{52}$~erg~s$^{-1}$ (panel de abajo a la izquierda) o 
de $L_{\rm iso, 0} = 1.27 \times10^{52}$~erg~s$^{-1}$ (panel de abajo a la derecha), tiene un ángulo de apertura de $\theta_0=9.2$ (panel de arriba a la izquierda) o de $\theta_0=22.9$ (panel de arriba a la derecha), 
es lanzado desde una distancia $r_{in}=10^{9}$~cm, atraviesa un medio que está en reposo y cuya masa es de $M_{m} = 13.95 M_{\odot}$.}

{\color{blue} En el panel de arriba a la izquierda muestra un jet al tiempo $t = 0.22$~s con un tamaño de $3.75 \times 10^9 \, \text{cm}$, el capullo tiene un ancho de $3.75 \times 10^9 \, \text{cm}$. 
Tambien se observa que en el centro, alrededor del capullo, la densidad de concentra mas, ya que llega a $\log_{10} \approx 3$ disminuye a $\log_{10} \approx 2$ a partir de $3.0 \times 10^9 \, \text{cm}$. 
Otra observación es que hay una mayor turbulencia desde el radio de inyección y disminuye conforme se va alejando. Mantiene un ancho aproximado de jet de $0.05 \times 10^9 \, \text{cm}$. 
El panel de arriba a la derecha al tiempo $t = 0.22$~s muestra un jet con un capullo de $1.0 \times 10^9 \, \text{cm}$ de ancho. El largo del jet es ligeramente mas grande que el anterior ya que 
mide $5.5 \times 10^9 \, \text{cm}$. El jet no es completamente colimado, debido a que forma una figura ovaloide y alcanza un ancho máximo de $0.3 \times 10^9 \, \text{cm}$ a una altura de $3.0 \times 10^9 \, \text{cm}$. 
Se muestra una mayor turbulencia que en la Figura de la izquierda, siendo más caótico al alejarse del radio de inyección del objeto compacto. Y la densidad de mayor concentración que 
rodea el capullo $\log_{10} \approx 3$ solo llega hasta $2.1 \times 10^9 \, \text{cm}$.
El panel de abajo a la izquierda muestra un jet de un Colapsar al tiempo $t = 3.8$~s. Su largo es de $3.75 \times 10^{10} \, \text{cm}$. El jet es completamente colimado con un ancho de $0.05 \times 10^{10} \, \text{cm}$. 
Pero a diferencia de los formados por la fusión de estrellas de neutrones, mantiene una densidad máxima de $\log_{10} \approx 3$ sobre todo el capullo presentando una maoy turbulencia al inicio del radio de
inyección.
Para el panel de abajo a la derecha se muestra, al tiempo $t = 9.60$~s, un jet con un largo de $3.75 \times 10^9 \, \text{cm}$. El ancho del jet es de $0.1 \times 10^{10} \, \text{cm}$ y muestra una mayor densidad alrededor
del capullo con $\log_{10} \approx 3$ sobre todo al inicio del radio de inyección del jet. El capullo mide $1 \times 10^{10} \, \text{cm}$ y muestra choques de ondas de colimación en los puntos
$1.5 \times 10^{10} \, \text{cm}$, $2.5 \times 10^{10} \, \text{cm}$ y $3.5 \times 10^{10} \, \text{cm}$.}

El perfil de densidad en la región interior lo aproximan a una función de ley de potencias con un índice $n<3$ y $n \approx 1.5$ para el caso de Colapsares. Cabe destacar 
que los modelos de los jet se asumen como no magnéticos, asi como otras limitaciones tales como los efectos de los neutrinos, el proceso r, el viento viscoso, la relatividad general, 
la rotación estelar, el campo magnético estelar, etc.






\begin{table}
  \begin{center}
    \begin{tabular}{ c c c c c c c} 
      \hline
      Modelo & Tipo                          & $\text{M}_{m} [\text{M}_{\bigodot}]$  & $\theta_0 [\text{grad}]$ & $\text{L}_{\text{iso},0}  [\text{erg}\,s^{-1}]$ & $r_{in} $ [cm]             & $v_m$ [c]               \\
      \hline
      T-03H  & Estrella Binaria de Neutrones & 0.02                                  &    6.8                   &               $5.0   \times 10^{50}$            &     $1.2 \times 10^8$      &   0.34  \\ 
      T-12H  & Estrella Binaria de Neutrones & 0.02                                  &    18.8                  &               $5.0   \times 10^{50}$            &     $1.2 \times 10^8$      &   0.34  \\ 
      A      & Colapsar                      & 13.950                                &    9.2                   &               $7.83  \times 10^{52}$            &     $10^9$                 &       0                 \\ 
      B      & Colapsar                      & 13.950                                &    22.9                  &               $1.27  \times 10^{52}$            &     $10^9$                 &       0                 \\ 
    \end{tabular}
  \caption{Valores que se usa las simulaciones de la Figura \ref{fig:jet_models} donde $\text{M}_{\bigodot}$ son masas solares, $\theta$ es el ángulo de apertura inicial, \emph{L} es la luminosidad isotrópica inicial,
  $r_{\text{in}}$ es el radio de inyección del jet y $v_m$ la velocidad del medio ambiente.} \label{table:jet_models}
  \end{center}
\end{table}

%--------------------------------------------------------------------------
%Ejemplo de estudio de jets en el marco de AGNs. FALTA
%--------------------------------------------------------------------------
{\color{blue} En la Figura \ref{fig:jet_agn} siguiente ejemplo muestra la simulación de un jet de núcleos activos de galaxias. A escalas parsec, los jets de AGN parecen estar altamente colimados con un punto brillante 
(el núcleo) en un extremo del jet y una serie de componentes que se separan del núcleo, a veces a velocidades superlumínicas. El jet se inyecta con un radio inicial de 100 pc con velocidades de flujo 
$v_j = 0.984$~c  y una relación de densidad típica entre el material del chorro y el entorno de $\rho_j / \rho_a = 5\times10^{-4}$. La inyección de densidad es $\rho_j = 8.3 \times 10^{-29} \, \text{erg}\,s^{-1}$ y una 
luminosidad $10^{46} \, \text{ergs} \, \text{s}^{-1}$. Las condiciones de frontera, son la reflexión en la base del jet, la reflexión en el eje del jet y outflow al final del dominio en las direcciones axial y radial.
En el panel izquierdo se muestra el logaritmo de densidad al tiempo t = 1.1 Myr con un largo de 900 kpc. El capullo mide aproximadamente 400 kpc y muestra pequeñas cavidades dentro de esta y turbulencia alrededor del jet.
En el panel derecho, el cual es un reflejo del izquierdo pero tomando en cuenta la temperatura, se pueden observar las mismas propiedades que en el logaritmo y una temperatura más alta en la zona de turbulencia.}



\begin{figure}
  \begin{center}
    \includegraphics[width=0.95\textwidth]{Figuras/Introduccion/jet_agn.png}
  \end{center}
  \caption{Mapa del logaritmo de densidad (izquierda) y temperatura (derecha) al tiempo t = 1.1 Myr. Las figuras muestran una imagen reflejada alrededor del eje de simetría. Donde se inyecta un jet con un radio inicial de 100 pc 
  con velocidades de flujo $v_j = 0.984$~c y una densidad $\rho_j = 8.3 \times 10^{-29} \, \text{erg}\,s^{-1}$.}
  \label{fig:jet_agn}
\end{figure}


%----------------------------------------------------------------------------------
\section{Estudio de un jet relativista en dos dimensiones (Mignone et al. 2005).}
%----------------------------------------------------------------------------------
En esta tesis se pretendrá hacer un código que reproduzca los jet que muestra Mignone \emph{et al} 2005 \cite{MB-HLLC-I} donde ellos consideran la propagación de un jet relativista axisimétrico ligero en 
coordenadas cilíndrica 2D (ver Figura \ref{fig:jet_mignone}). Los valores que usó para poder simular el jet fueron los siguientes:

\begin{equation}
  \left(\rho, v_{r}, v_{z}, p\right)=\left\{\begin{array}{ll}
  \left(0.1,0,0.99,10^{-2}\right) & \text { para } r, z<1 \\
  \left(10,0,0,10^{-2}\right) & \text { caso contrario. }
  \end{array}\right.
\end{equation}

Donde $\rho$ es la densidad, $v_r$ es la velocidad radial, $v_z$ es la velocidad sobre el eje z y $p$ es la presión. El índice politrópico que usaron fue $\Gamma = 5/3$ con un número de Courant $Co = 0.5$.
Los métodos que emplearon fue el método de HLL y HLLC (Para mas información ver la sección \ref{secc:HLL}). El tiempo de integración fue de $t = 80$ con condiciones de frontera \emph{outflow} excepto donde se 
inyecta el jet. La velocidad de avance promedio  del jet fue de 0.39 con una resolución de 400X750. Un aspecto que jugó a favor del formalismo HLLC es la eficiencia computacional, particularmente crucial en 
simulaciones a largo plazo en dos y tres dimensiones, también requirió pocos costos adicionales, entre 4 y 7 por ciento en una dimensión, con respecto al esquema HLL. 


\begin{figure}
  \begin{center}
    \includegraphics[width=0.5\textwidth]{Figuras/Introduccion/jet_mignone.png}
  \end{center}
  \caption{Logaritmo de la densidad de la masa en reposo para un jet relativista. El panel de arriba muestra el jet al tiempo t=40, mientras que el de abajo al tiempo t = 80.
  La mitad de  (de ambos páneles) muestra la simulación hecha con el esquema HLLC, mientras que la parte de abajo muestra para el esquema HLL.
  La resolución fue de 240x740 pixeles}
  \label{fig:jet_mignone}
\end{figure}

{\color{blue}El panel de arriba muestra un jet con un tamaño sobre el eje radial de 19 unidades, se presenta una mayor turbulencia, asi como un ancho de 5 unidades y Su capullo
mide un máximo de 7 unidades para el esquema HLLC.
Para el caso, en el que se usa HLL, el tamaño del jet es de 18 unidades. Mientras que su ancho alcanza las 4 unidades. Tambien se presenta una menor turbulencia con respecto a HLLC pero presenta un 
ancho mayor, ya que alcanza un máximo de 7.5 unidades.}


{\color{blue} Para el panel de abajo, en el caso de HLLC, presenta un largo de 32 unidades. Mantiene una mayor tubulencia, un ancho es de 7 unidades y su capullo mide aproximadamente 11 unidades. Para la parte donde 
se presenta HLL, se puede observar que tiene una menor turbulencia. El largo y ancho miden 31.5 y 8 respectivamente. Y el capullo presenta un tamaño de 11.5 unidades. }

{\color{blue} Con base en lo anterior se enuncian los objetivos a desarrollar en esta tesis:
\begin{itemize}
  \item Desarrollar un código, que sea capaz de resolver numéricamente las ecuaciones de la hidrodinámica, tanto para los casos newtonianos así como los relativistas en una y dos dimensiones.
  \item Probar el código con distintos problemas físicos ya resueltos analíticamente para comparar y verificar su correcto funcionamiento del código.
  \item Reproducir un jet con los parametros del estudio de Mignone y compararlo con el código.
  \item Estudiar el comportamiento de un jet en un medio no homogeono y compararlo con el mismo jet que se desarrolla en un medio homogéneo.
\end{itemize}

}








% Aunque hay mucha información acerca de los GRBs largos, los cortos son aún un estudio nuevo en el área de astrofísica de altas energías ya que debido a su corta duración son difíciles de estudiar. 
% \section{Características de los GRBs}
% Muchas de las características de los GRBS ya sean estos largos o cortos, son la duración media que tienen, mientras que en los GRBs largos tienen un promedio de duración de 100 s \cite{PGRB-piran}, los GRBs cortos duran menos de 2 segundos. Los estallidos de Rayos gamma se encuentran a miles de años luz de nuestra galaxia por lo que se puede decir que tienen un orígen cosmologico, mientras los GRBs largos se originan en los centros de las galaxias, los cortos lo hacen lejos de la galaxia donde se originaron y por consecuencia, estos tienen una densidad de ambiente muy bajo. El estudio de los SGRB ha sido muy difícil debido al tiempo de duración que conllevan estos eventos y a la dificultad de asociarlo con galaxias anfitrionas. 

% %Las partes principales de los GRBs son la emisión propmt y  afterglow. %Piran2005

% \begin{figure} %fuente de la imagen:https://astrobites.org/2017/11/13/grb-afterglows-coming-out-of-a-cocoon/
%   \centering
%     \includegraphics[width=0.5\textwidth]{Figuras/Gamma-ray_burst_by_a_blackhole-768x432.jpg}
%   \caption{Diferentes partes de un GRB en general, mostrando las partes más características como el afterflow y la emisión prompt}
%   \label{fig:Partes de GRBs}
% \end{figure}
 

% \subsection{Emisión pronta} %Azzam
% La emisión pronta es definido como el periodo de tiempo donde el detector de rayos $\gamma$ 
% detecta una señal sobre el fondo. Estas consisten en fotones de alta energía, no tiene espectro térmico \cite{GRB:CAP}. El espectro térmico es producido por un gas en equilibio térmico. Además, tiene bastantes carterísticas observacionales.

% El flujo de energía varía dentro de cientos de Kev. La emisión de las curvas de luz son notoriamente irregulares sobre escalas de tiempo muy pequeñas \cite{SGRBr-Avanzo}(ver Figura \ref{lightcurve}). %Short gamma-ray bursts: A review; P. D’Avanzo
% En los GRBs cortos, su espectro de emisión, se ha encontrado más duro que en los GRBs largos, esto debido a que es una combinación de componentes espectrales duros de baja energía. Una característica de principal es la duración de esta emisión, esta llega a ser del orden $\geq \, 10^{-2} $ s a $10^{3}$ s \cite{GRB:PPP}. Los valores promedio de dichas emisiones son $t \sim 20 $ s para GRBs largos y $t \sim 0.2 $ s para los cortos \cite{GRB:PPP}. %Gamma Ray Burst Progress, problems and prospects Zhang, meszaros

% \begin{figure}
% \centering %Fuente Zhang
% \includegraphics[width=1.0\textwidth]{Figuras/lightcurve_different}
% \caption{\label{lightcurve} Diferentes curvas de luz captadas por el detector Batse detectadas en la emisión pronta de los GRBs. Las curvas de luz difieren bastantes unas de otras, mientras que BATSE 332 es un decaimiento exponencial , BATSE Trigger 1989  tiene 2 picos de intensidad. }
% \end{figure}
% \subsection{Emisión tardía}
% La emisión tardía,conocida también como \emph{afterglow},  es después de la emisión prompt en el cual se pueden detectar otras longitudes de onda como el óptico, radio y los rayos X \cite{Berger:2013jza, PGRB-piran, SGRBr-Avanzo}. %\ref{fig:Partes de GRBs}.
% La propiedad de "multiples longitudes de onda", hace una propiedad única de la emisión tardía. Estos cubren un amplio rango de frecuencias, desde las ondas de radio hasta los Tev. El espectro de banda ancha es descrito como una ley de potencias inversa, ya que la fuerza del choque se reduce a medida que la onda expansiva va desacelerando. Con lo que se espera que todas las longitudes de onda de las curvas de luz decaigan como ley de potencias. Entonces la densidad de flujo es proporcional a:

% \begin{equation}
% F_{\nu}(t,\nu) \propto t^{-\alpha} \nu^{\beta}
% \end{equation}

% Donde $\alpha$ y $\beta$ son reales positivos.


% En Mayo y Julio del 2005 datos de eyecciones recopiladas por el satélite \emph{swift} al seguir al seguir al GRB 050509B, descubrieron las primeras fases tardías de los GRBs cortos, tiempo después el satélite HETE-2 junto con el observatorio \emph{chandra} de rayos X siguieron al satélite 050709 el cual localizo la fase tardía en rayos X y despues en la banda óptica, con estos datos recabados de la fase tardía se pudo llegar a que los GRBs cortos tienen una escala de densidad y una energía mas baja que los GRBs largos, también se llego a la conclusión de que los GRBs cortos tienen orígenes cosmologicos \cite{Berger:2013jza} y que las estrellas masivas no son sus progenitores como en el caso de GRBs cortos.
 
% \subsection{Energía y luminosidad}
% Los GRBs largos tienen una luminosidad isotrópica $L_{iso} \sim 10^{49}-10^{54} \, \mathrm{ergs} \cdot \mathrm{s}^{-1}$. También  hay GRBs largos con una luminosidad baja, sus valores rondan acerca de $L_{iso} \sim 5 \times 10^{46}-10^{49} \, \mathrm{ergs} \cdot \mathrm{s}^{-1}$, pero estos son solo una pequeña fracción de los GRBs largos observados. Los GRBs cortos tiene generalmente una luminosidad de $L_{iso} \sim 10^{49} \, \mathrm{ergs} \cdot \mathrm{s}^{-1}$.

% La distribución de energías es amplia para los GRBs, estos cubren valores $E_{iso} \sim 10^{49}-10^{55}$ ergs para los largos y para los cortos $E_{iso} \sim 3.3 \times 10^{46}-10^{53}$ ergs \cite{Zhang:PGRB}. %the physics of gamma ray burst

% %
% \section{Características de los GRBs cortos}
% Los GRBS cortos, aunque tiene muchas carcteristicas en común con los GRBs largos, también tienen sus propias peculariedades que lo hacen diferente del otro.

% \subsection{Progenitores} %ghrels2009
% Los GRBs cortos por lo general se localizan en medios de baja formación estelar por lo que no se espera poder encontrarlos ambientes estelares jovenes además de que, conllevan una carencia para poder asociarlos con SNs. El modelo más aceptado en la literatura de los progenitores de los GRBs cortos son la fusión de objetos compactos binarios. Estos pueden comprender a dos estrellas de neutrones (NS-NS), una estrella de neutrones y un agujero negro (NS-BH) o dos agujeros negros (BH-BH) \cite{GRB-SE}. La fusión se logra a cabo debido a la pérdida de momento angular y energía debido a las ondas gravitacionales. La interacción de este fenómeno lleva a la extracción de energía por procesos magnetohidrodinámicos (MHD), los cuales impulsan un flujo relativista colimado. 

% Las patadas natales pueden ser las causantes de las distancias en las que estos nacen y los sitios de explosión de estos sistemas \cite{Berger:2013jza}, la probabilidad para una galaxia de brillo $m$  a ser localizada a una separación $\delta R$ de la posición de un SGRB está dada por 

% \begin{equation}
% P_{cc} = 1 - \exp ^{- \pi (\delta R)^{2} \sum(\leq m)}
% \end{equation}
% Donde $\sum(\leq m) = 1.3\cdot 10^{0.33(m-24)-2.44} \mathrm{arcsec} ^{-2}$ son el número de cuentas de la galaxia. Los SGRBs sin galaxias anfitrionas se exhiben cerca galaxias de campo con una baja probabilidad de coincidencia, generalmente las distancias medias de separación del SGRB con su galaxia anfitriona es de $ \frac{\delta R}{r_e} \approx 1.5$ donde $r_e$ es el radio estelar. 

% \subsection{Propiedades de las explosiones de los SGRB}

% Los parámetros claves de interés de la emisión pronta y tardia \cite{Berger:2013jza}
% \begin{itemize}
% \item $E_{\gamma}$: Energía de rayos $\gamma$.
% \item $E_k$: Energía cinética de la onda de choque de la emisión tardía.
% \item $\theta_j$: El ángulo de apertura del jet.
% \item $n$: Densidad del medio ambiente.
% \end{itemize}

% La emisión del sincrotrón de la emisión tardía incluye dos parámetros libres que tienen relación con la física del choque relativista:
% \begin{itemize}
% \item $\epsilon_{B}$: Fracción de energía del post-choque de los campos magnéticos.
% \item $\epsilon_e$: Radiación relativista de los electrones. 
% \end{itemize}



% El flujo relativista interactuando con el medio circundante es el espectro de emisión sincrotónico. Esta se caracteríza por 3 frecuencias de corte:
% \begin{itemize}
% \item $\nu_a$: Auto absorción
% \item $\nu_m$: Factor mínimo de Lorentz
% \item $\nu_c$: enfriamiento sincrotrónico
% \end{itemize}

% Tomando en cuenta los valores de estos parámetros y la pendiente del espectro para $\nu < \nu_m$ se pueden llegar a encontrar los siguientes valores $E_{k, \, \mathrm{iso}}$, $n$, $\epsilon_{B}$ y $\epsilon_e$. La evolución temporal del espectro también se puede usar para encontrar el ángulo de apertura del jet $\theta_j$. La relacción que existe entre el ángulo de apertura $\theta_j$ y el tiempo $(t_j)$ al cual ocurre el jet break  está dado por la siguiente fórmula.

% %La mayoría de estas frecuencias se encuentran entre 1 $\thicksim$ 10 GHz por lo cual ha sido difícil de detectarlas. La colimación del jet también influye en la taza de los SGRBs y proporciona una restricción adicional al modelo progenitor, la firmatura de la colimación de los destellos de los SGRBs son los llamados "Jet Break" que ocurren al tiempo $t_j$ cuando $\Gamma_j(t_j) = 1/ \theta_j$, esto lidera al cambio de emisión del espectro sincrotrónico
% %
% %\begin{equation}
% %F_\nu \propto t^{-1} \longrightarrow F_\nu \propto t^{-p}
% %\end{equation}
% %
% %La relación entre el "jet break" y el ángulo de apertura esta dado por:
% %
% \begin{equation}
% \theta_j = 0.13 \left( \frac{t_{j,d}}{1+z} \right)^{3/8}
% \left(\frac{n_0}{E_{52}} \right)^{1/8}
% \end{equation}
% %
% %\section{GRB del 17 de agosto del 2017}
% %

% \begin{figure}
% \centering
% \includegraphics[scale=0.7]{./Figuras/Distribucion_angulo_apertura}
% \caption{La Figura muestra la distribución de los ángulos de apertura para los GRBs largos (rojos) y los GRBs cortos (azules). Las flechas muestran el límite superior e inferior de los ángulos. }
% \end{figure}

\chapter{Teoría}

Los GRBs, ya sean largos o cortos, que se originan en el medio interestelar, se pueden describir como un fluido sin viscosidad y estos pueden ser descritos bajo
las ecuación  es de la hidrodinámica.
Para describir un sistema de partículas como un fluido bajo ciertas condiciones uno debe de conocer que el camino libre medio debe de ser mucho mas pequeño que la escala de longitud de las fluctuaciones de las variables macroscópicas.

\begin{equation}
\lambda_{mfp} \ll L
\end{equation}

El tiempo entre las colisiones debe de ser pequeña comparada con la escala del tiempo de los cambios en el fluido
\begin{equation}
t_{c} \ll t_f
\end{equation}
La distancia media entre las partículas tiene que ser mas pequeña que la longitud de escala de las variables macroscópicas

\begin{equation}
l = n^{-1/3} \ll L
\end{equation}



\section{Ecuaciones de la hidrodinámica}

Considerando una serie de elementos de volumen fijos, las ecuación  es que describen el movimiento de un fluido sin considerar efectos viscosos son:

La conservación de masa
\begin{equation} \label{conservación_masa_hidrodinamica}
\dfrac{\partial \rho }{\partial t} + \nabla \cdot \left( \rho \mathbf{u} \right)
\end{equation}

El momento
\begin{equation}  \label{conservación  _momento_hidrodinamica}
\dfrac{\partial \left( \rho \mathbf{u} \right) }{\partial t}+ \nabla \cdot \left( \rho \mathbf{u u} \right) + \nabla p = \mathbf{f_{ext}}
\end{equation}

Ecuación de la energía

\begin{equation} \label{conservación  _energia_hidrodinamica}
\dfrac{\partial E }{\partial t} + \nabla \cdot \left[ \mathbf{u} \left( e+p \right) \right] =G-L+\mathbf{f_{ext} \cdot \mathbf{u}}
\end{equation}

Ecuación de estado
\begin{equation}
e=\frac{1}{2} \rho \mathbf{u}^{2} + \frac{p}{\Gamma - 1}
\end{equation}

Con estas ecuación  es podemos formar una matriz de $5x5$ escritas en coordenadas cartesianas:

\begin{equation} \label{euler_cartesianas}
\dfrac{\partial \mathbf{U}}{\partial t}+\dfrac{\partial \mathbf{F}}{\partial x}+\dfrac{\partial \mathbf{G}}{\partial y}+\dfrac{\partial \mathbf{H}}{\partial z}= \mathbf{S}
\end{equation}

Donde
\begin{center}


$\mathbf{U}=
\left(\begin{smallmatrix}
\rho \\
\rho u \\
\rho v \\
\rho w \\
E \\
\end{smallmatrix}\right)
$,
$\mathbf{F} =
\left(\begin{smallmatrix}
\rho u \\
\rho u^{2}+P \\
\rho uv \\
\rho uw \\
u(e+P) \\
\end{smallmatrix}\right)
$,
$\mathbf{G} =
\left(\begin{smallmatrix}
\rho v\\
\rho vu \\
\rho v^{2}+P \\
\rho vw \\
v(e+P) \\
\end{smallmatrix}\right)
$,
$\mathbf{H} =
\left(\begin{smallmatrix}
\rho w\\
\rho wu \\
\rho wv \\
\rho w^{2}+P \\
w(e+P) \\
\end{smallmatrix}\right)
$, 
$\mathbf{S} =
\left(\begin{smallmatrix}
0 \\
f_{x} \\
f_{y} \\
f_{z} \\
G-L+\textbf{f} \cdot \textbf{u} \\
\end{smallmatrix}\right)
$
\end{center}

El término de $\mathbf{U}$ son nuestras variables conservadas, los términos $\mathbf{F}$, $\mathbf{G}$, $\mathbf{H}$ son nuestros fluidos  con velocidades en la dirección $x$, $y$ y $z$ y $\mathbf{S}$ son los términos fuente. Para poder resolver computacionalmente estas ecuación  es diferenciales parciales vamos a utilizar el método de las diferencias finitas y el método de Lax sin considerar tos términos fuente es decir $\mathbf{S}=0$ en el siguiente capitulo si se trataran fuentes particulares.

La implementación en el código se hará en 2 dimensiones por lo que la variable \emph{H} que corresponde a los flujos en el eje z será cero el bloque de código esta en una subrutina llamada \emph{fluxes}, 
donde en el mismo bucle se desacoplan las primitivas para que queden en función de las conservadas.Cabe señalar que tanto en $f(4,i,j)$ y $g(4,i,j)$ se usa la
la variable conservada $u(4,i,j)$, ya que $u(4,i,j) = e$ . 


\begin{lstlisting}[frame=single]

  f(1,i,j)=rho*vx
  f(2,i,j)=rho*vx*vx+P
  f(3,i,j)=rho*vx*vy
  f(4,i,j)=vx*(u(4,i,j)+P)

  g(1,i,j)=rho*vy
  g(2,i,j)=rho*vx*vy
  g(3,i,j)=rho*vy*vy+P
  g(4,i,j)=vy*(u(4,i,j)+P)

\end{lstlisting}

Las variables \emph{i}, \emph{j} corren sobre todo el dominio espacial dentro de un bucle, el término $\mathbf{H}$ no es incluido, debido a que es computacionalmente 
caro. El término \emph{u} hace alusión a las variables conservadas.

\subsection{Desacoplamiento de las ecuación  es de la hidrodinámica}

Al final del método de las diferencias finitas, obtenemos nuestras variables conservadas (\emph{U}), pero para calcular nuestros flujos de nuevo necesitamos recuperar nuestras primitivas, es decir, calcular nuestras variables primitivas $(\rho, \, u, \, v,\, w, \, P )$ en función de nuestras variables conservadas $(u_1, \, u_2, \, u_3, \, u_4, \, u_5)$.

Despejar la densidad es sencillo ya que es directo $u_1= \rho$ por lo tanto:

\begin{equation}\label{primitiva_densidad}
\rho = u_1
\end{equation}

Para las velocidades $u_i=\rho \upsilon$, donde $i=2,3,4$ y $\upsilon=u,v,w$, nos da $\upsilon= u_i/ \rho$ y usando la ecuación \ref{primitiva_densidad} queda:

\begin{equation} \label{primitiva_velocidades}
\upsilon = u_1/u_i
\end{equation}
Para la ecuación   de la energía $u_5=E$ combinando con la ecuación   de estado y la ecuación \ref{primitiva_velocidades} obtenemos
\begin{equation}
P = \left( \Gamma - 1 \right) \left[ u_5 - \frac{u_1 \left( \sum_{i=2}^{4} u_1/u_i \right)^2}{2} \right]
\end{equation}

En el código se implementa de la siguiente manera.
Las variables \emph{i}, \emph{j} son contadores, y antes de que se calculen los flujos, se tienen que calcular las primitivas. Dado que el desacoplamiento
se implementa directamente en la subrutina \emph{fluxes}, para que se puedan calcular los flujos.

\begin{lstlisting}[frame=single] 
  

    !Desacoplamiento de las primitivas
    rho = u (1,i,j)
    vx  = u(2,i,j)/rho
    vy  = u(3,i,j)/rho
    P   = (u(4,i,j)-0.5*rho*(vx**2+vy**2))*(gamma-1.)

\end{lstlisting}

donde que la variable \emph{u(4,i,j)} es la densidad de energía.
  
\section{Hidrodinámica relativista}
Los GRBs generalmente tienen velocidades relativistas, por lo que la parte newtoniana no nos va a servir, pero si nos servirá como apoyo para poder pasar a la hidrodinámica relativista.
En esta parte se añadirá a los códigos que ya hemos generado previamente, las próximas secciones abordara acerca de como cambian nuestras primitivas, como afectan a nuestras variables conservadas y como las podemos desacoplar así como varios ejemplos al cambiar varios valores de nuestros parámetros y de las condiciones iniciales.

\subsection{Primitivas}
Las ecuación  es que teníamos para fluidos newtonianos se pueden modificar para hacerlos relativistas. Para esto vamos a partir de 2 ecuación  es importantes que son la ecuación de energía-momento y la ecuación de conservación de masa:

\begin{equation}\label{ecuación  _conservación_masa}
\left( \rho u^{\alpha} \right)_{, \alpha} =0
\end{equation}

\begin{equation}\label{ecuación  _momento_energia}
T_{, \beta}^{\alpha \beta}=0
\end{equation}

De la ecuación \ref{ecuación  _conservación_masa} tenemos la cuadrivelocidad para un sistema de 
3 coordenadas y considaremos a la velocidad de la luz como $c=1$ lo podemos ver como $u^{\mu}=\gamma \left( 1, \textbf{v}\right)$ y sustituyendo este resultado (en 2 dimensiones espaciales) tendremos las ecuación  es $u_1$, $F_1$ y $G_1$. Para la ecuación \ref{ecuación  _momento_energia} podemos escribir el tensor de energía-momento como $T^{\mu \nu} = \rho h u^{\mu} u^{\nu} + Pg^{\mu \nu}$ y usando la métrica de Minkowski

\begin{equation}
\eta_{\alpha \beta}= 
\begin{pmatrix}
 -1 & 0 & 0 & 0 \\
 0 & -1 & 0 & 0 \\
 0 & 0 & -1 & 0 \\
 0 & 0 & 0 & 1 \\
\end{pmatrix}
\end{equation}

Con lo que podemos escribir a $T^{\mu \nu}$ matricialmente como:

\begin{equation}
T^{\mu \nu} =
\begin{pmatrix}
\rho h \gamma^2-P & \rho h \gamma^2 v_{x}  & \rho h \gamma^2 v_{y} & \rho h \gamma^2 v_{z} \\

\rho h \gamma^2 v_{x} & \rho h \gamma^2 v_{x}^{2}+P & \rho h \gamma^2 v_{x}v_{y} &  \rho h \gamma^2 v_{x}v_{z} \\

rho h \gamma^2 v_{y} & \rho h \gamma^2 v_{y}v_{x} & \rho h \gamma^2 v_{y}^{2}+P & \rho h \gamma^2 v_{y}v_{z}\\

\rho h \gamma^2 v_{z} & \rho h \gamma^2 v_{z}v_{x} & \rho h \gamma^2 v_{z}v_{y} & \rho h \gamma^2 v_{z}^2 + P

\end{pmatrix}
\end{equation}
Entonces nuestras ecuación  es quedarían de la siguiente manera
\begin{align}
u_{1}& = \rho \gamma \\ 
u_{2}& = \rho v_{x} \gamma^{2} h \\ 
u_{3}& = \rho v_{y} \gamma^{2} h \\ 
u_{4}& = \rho \gamma^{2} h - P 
\end{align}

Donde $\rho$ es la densidad, $\gamma$ es el factor de Lorentz, $v_{x}$ y $v_{y}$ son las velocidades de nuestros fluidos (en 2 dimensiones pero se puede extender esto a 3 sin ningún problema), $h$ es la entalpía  y $P$ es la presión. Para los fluidos quedan de la siguiente manera
\begin{align}
F_{1}& = \rho v_{x} \gamma \\ 
F_{2}& = \rho v_{x} v_{x} \gamma^{2} h + P\\ 
F_{3}& = \rho v_{x} v_{y} \gamma^{2} h \\ 
F_{4}& = \rho v_{x} \gamma^{2} h 
\end{align}

\begin{align}
G_{1}& = \rho v_{y} \gamma \\ 
G_{2}& = \rho v_{y} v_{x} \gamma^{2} h \\ 
G_{3}& = \rho v_{y} v_{y} \gamma^{2} h + P\\ 
G_{4}& = \rho v_{y} \gamma^{2} h
\end{align}
\begin{align}
H_{1}& = \rho v_{z} \gamma \\ 
H_{2}& = \rho v_{z} v_{x} \gamma^{2} h \\ 
H_{3}& = \rho v_{z} v_{y} \gamma^{2} h \\ 
H_{4}& = \rho v_{z}^{2} \gamma^{2} h + P
\end{align}

La implementación en el código es de la siguiente manera, el factor de Lorentz $\gamma$ es la variable $lor$, $h$ la entalpía, al igual que en la hidrodinámica newtoniana solo se considerará 2 dimensiones:

\begin{lstlisting}[frame=single]
  lor=1/sqrt(1-(vx**2+vy**2))
  h=1.+gamma/(gamma-1.)*P/rho
  
  f(1,i,j)=rho*vx*lor
  f(2,i,j)=rho*vx*vx*lor**2*h+P
  f(3,i,j)=rho*vx*vy*lor**2*h
  f(4,i,j)=rho*vx*lor**2*h

  g(1,i,j)=rho*vy*lor
  g(2,i,j)=rho*vx*vy*lor**2*h
  g(3,i,j)=rho*vy*vy*lor**2*h+P
  g(4,i,j)=rho*vy*lor**2*h

\end{lstlisting}

A diferencia de los flujos newtonianos, para los flujos relativistas se tienen que calcular antes el factor de Lorentz \emph{lor} y la entalpía \emph{h},
las variables \emph{i}, \emph{j} son contadores sobre el espacio.

\subsection{Desacoplamiento de las ecuación  es de la hidrodinámica relativista} \label{Cap_Desacoplamiento_de_las_ecuación  es_de_la_hidrodinámica_relativista}
Para poder desacoplar las ecuación  es, partimos de la relación de las densidades de energía total y del módulo de los momentos

\begin{equation}\label{ecuación  _de_energia}
  e=W-p
\end{equation}

\begin{equation}\label{modulos de los momentos}
\abs{m}^2= W^{2}\abs{v}^{2}
\end{equation}

Donde $W=D h \gamma$ y $D=\rho \gamma$. Para evitar errores en el límite no relativista se debe resolver la ecuación conservada restando la densidad de masa 
a la energía para definir una nueva variable conservada ($e^{'}=e-D$), para las cancelaciones en el límite ultra-relativista basados en $\gamma \abs{v^2}$ que se tiene cuando $\abs{v} \rightarrow 1$, se debe de crear otra variable, que en este caso seria $\abs{u}^2=\gamma \abs{v^2}$ e introduciendo las variables $W^{'}=W-D$. Podemos reescribir la última ecuación de la siguiente manera

\begin{eqnarray*}
 W^{'}& = &D(h \gamma -1)\\
&=& D\left[ \left(1-\epsilon+ \frac{p}{\rho}\right) \gamma - 1 \right]\\
&=& D \left(\gamma-1 \right) \frac{\gamma+1}{\gamma+1}+\frac{D \gamma }{\rho}\left(\rho \epsilon + p \right)
\end{eqnarray*}

Recordando que $D=\rho \gamma$ y que a partir de la variable introducida $u^{2}$ podemos 
reescribir el factor de Lorentz como $\gamma^{2} = 1-u^{2}$

\begin{eqnarray}\label{W_prima}
\nonumber W^{'}&=&\frac{D u^{2}}{\gamma + 1}
+\frac{\rho\gamma \gamma}{ \rho }\left(\rho \epsilon + p \right)\\
&=& \frac{D u^{2}}{\gamma + 1} + \gamma^{2} \chi
\end{eqnarray}

Donde $\chi=\rho \epsilon + p$, derivando con respecto a $W^{'}$ la ecuación \ref{ecuación  _de_energia} queda como
\begin{equation}\label{derivada_E_W}
\dfrac{de}{dW^{'}}=1-\dfrac{dp}{dW^{'}}
\end{equation}

Nosotros no sabemos como es la expresión $\dfrac{de}{dW^{'}}$, así que asumiremos que $p=p(\rho, \chi)$ por lo que podemos aplicar la regla de la cadena (para mas detalles consulte)
\begin{equation}\label{cadena}
\dfrac{dp}{dW^{'}}=\dfrac{\partial p}{\partial\chi}\Bigg |_{\rho} \dfrac{d\chi}{dW^{'}} + \dfrac{\partial p}{\partial \rho}\Big |_{\chi} \dfrac{d \rho}{d W^{'}}
\end{equation}

Para calcular $\dfrac{dp}{d\chi}$ tenemos que por ser el caso de un gas ideal
\begin{equation} \label{entalpía_función_presion_densidad}
h=1+\frac{\Gamma}{\Gamma-1}\frac{p}{\rho}
\end{equation}
Donde $h$ también puede ser escrito como
\begin{equation}
h=1+\epsilon+\frac{p}{\rho}
\end{equation}
Si combinamos estas 2 últimas ecuación  es podemos llegar a que 
\begin{equation}
p(\chi,\rho)=\frac{\Gamma-1}{\Gamma}\chi
\end{equation}

Con lo que al derivar con respecto de $\chi$ nos da como resultado
\begin{eqnarray}\label{der_presion}
& \dfrac{d p}{d \chi}&=\frac{\Gamma-1}{\Gamma}\\ &\dfrac{d p}{d \rho}&= 0
\end{eqnarray}

De la ecuación \ref{W_prima} podemos despejar $\chi$, lo que queda como
\begin{equation}
\chi=\frac{W^{'}}{\gamma}- \frac{D u^{2}}{(1+\gamma)\gamma^{2}}
\end{equation}

Derivando implícitamente la ecuación \ref{W_prima} respecto a $W^{'}$ nos quedaría
\begin{eqnarray*}
W^{'}&=&D\left(\gamma-1 \right) + \chi \gamma^{2}\\%saloto de linea
&=& D\left(\frac{1}{\sqrt{1-\nu^{2}}} -1\right)+\chi \dfrac{1}{1-\nu^{2}} \\%saloto de linea
\dfrac{d W^{'}}{d W^{'}} &=& D \dfrac{d (1-v^2)^{-1/2}}{d W^{'}}+\dfrac{d \chi}{dW^{'}}(1-v^2)^{-1}+\dfrac{d (1-v^2)^{-1} }{d W^{'}}\chi \\  %salto de linea 
1 &=& \frac{D(1-v^2)^{-3/2}}{2} \dfrac{d v^{2}}{d W^{'}}+\dfrac{d \chi}{dW^{'}}(1-v^2)^{-1}+ \chi (1-v^2)^{-2}  \dfrac{d v^{2}}{d W^{'}} \\ %salto de linea
\frac{1}{\gamma ^2} &=& \frac{D \gamma}{2} \dfrac{d v^{2}}{d W^{'}} + \dfrac{d \chi}{dW^{'}} + \chi \gamma^2 \dfrac{d v^2}{dW^{'}} \\ %salto de linea
\end{eqnarray*}

Con lo que al final la ecuación se puede escribir de la siguiente manera

\begin{equation}\label{der_chi}
\dfrac{d \chi}{dW^{'}}=\frac{1}{\gamma^2}-\frac{\gamma}{2}(D-2\gamma \chi) \dfrac{d v^2}{dW^{'}}
\end{equation}

Y para

\begin{equation}\label{der_rho}
\dfrac{d \rho}{d W^{'}}= D \dfrac{d\left(1/ \rho \right) }{d W^{'}} = - \frac{D \gamma}{2}  \dfrac{d v^2}{dW^{'}}
\end{equation}

despejando la ecuación \ref{modulos de los momentos} podemos llegar a escribir el módulo de la velocidad al cuadrado de la siguiente manera
\begin{equation}
\abs{v^{2}} = \frac{\abs{m^{2}}}{W^{'}} 
\end{equation}
Donde $m_i= \rho v_i \gamma h$ para $i=x,y$.

Podemos demostrar que $\dfrac{d \abs{v^2}}{W}=\dfrac{d \abs{v^2}}{W^{'}}$ para esto vamos a partir de lo siguiente 
\begin{eqnarray*}
\abs{v^2}&=& \abs{m^{2}} \left(W^{'} + D \right)^{-2} \\
\dfrac{d \abs{v^2}}{d W^{'}} &=& \frac{-2 \abs{m}^2}{W^{'}+D^{3}} \\
&=& \frac{2 \abs{m}^2}{W^{3}} \\
&=& \dfrac{d \abs{v^2}}{d W^{'}}
\end{eqnarray*}

Con lo que podemos decir que 

\begin{equation}\label{der_v2}
\dfrac{d\abs{v}^2}{d W^{'} }=-\frac{2 \abs{m}^{2}}{W^{3}}
\end{equation}
Con todo esto ya sabemos cuanto es lo que vale la ecuación \ref{derivada_E_W}, con esto ya podemos usar el método de Newton-Rapson para poder encontrar $W^{'}$.\\

El método de Newton-Rapson es un algoritmo iterativo que se usa para encontrar raíces  de una función real:

\begin{equation} \label{eq_Newton_Rapson}
W^{'(k+1)}=W^{'}-\frac{f(W^{'})}{\dfrac{d f(W^{'})}{d W^{'}}}
\end{equation}

De la ecuación de la densidad de energía, podemos utilizarla como a la función a la que queremos encontrar la raíz

\begin{equation} \label{ecuación_f}
f(W^{'})=W^{'}-e^{'}-p
\end{equation}

Donde $e^{'}=W^{'}-p$ y que $\dfrac{d f(W^{'})}{d w} \equiv \dfrac{de}{dW^{'}}$ dado por la ecuación \ref{derivada_E_W}. 
Para iniciar el proceso de iteración se tiene que hacer una suposición, para esto, con ayuda de las ecuación  es \ref{ecuación  _de_energia} 
y \ref{modulos de los momentos} podemos llegar a que la presión es:
%===============================================================================================

\begin{equation} \label{presion_de_newton}
p=\frac{\abs{m}^{2}-W^{2}\abs{v}^{2}+4W^{2}-4EW}{4W}
\end{equation}

Como podemos ver el denominador es una función convexa cuadrática que depende $\abs{v}^{2}$ y $W$ y cumple con que $W$ este fuera del intervalo de las raíces positivas y negativas.

Al denominador de la  ecuación podemos encontrar $W$ ya que $\abs{v}^{2}=1-1/\gamma^{2}$ suponiendo $\gamma \rightarrow \infty$ podemos despejar $W$

\begin{equation}\label{suposicion_de_W}
W=\frac{-(-2E)+\sqrt{(-2E)^{2}-(3)(\abs{m}^{2})}}{3}
\end{equation}

Con esto ya se puede hacer las aproximaciones para obtener $W$, y podemos calcular las siguientes relaciones 
\begin{eqnarray}
\abs{v}^{2} &=& \frac{\abs{m}^{2}}{W^{2}}\label{prim_v2}\\ 
u^{2}&=&\frac{\abs{v}^{2}}{1-\abs{v}^{2}}\label{u2}\\
\gamma &=& \sqrt{1+u^{2}}
\end{eqnarray}
Y las nuevas primitivas.\\

Velocidades:
\begin{eqnarray}
v_{x}&=&\frac{u_{2}}{W}\\
v_{y}&=&\frac{u_{3}}{W}\\
\end{eqnarray}

Densidad de masa 
\begin{equation}
\rho=\frac{D}{\gamma}
\end{equation}

Presión térmica
\begin{eqnarray}
\chi&=&\frac{W-D}{\gamma^{2}}-\frac{D \abs{u}^{2}}{(1+\gamma)\gamma^{2}}\\
p&=&\frac{\Gamma-1}{\Gamma} \chi
\end{eqnarray}

Aunque en el método newtoniano se podían desacoplar las ecuación  es sobre el mismo bucle, 
para el método relativista se tuvo que crear una subrutina para poder desacoplar las primitivas y así poder obtenerlas en función de las conservadas. 
Al igual que en el método newtoniano vamos a correr el bucle sobre nuestra malla. La subrutina de desacoplamiento se llama \emph{uprim}, 
y esta recibe un parámetro de entrada, que es la conservada \emph{u} y nos devuelve una variable primitiva. 
Las variables conservadas las reasignamos a unas nuevas llamadas \emph{qu} para que podamos usar la subrutina, una vez que se desacoplan, 
los valores de salida los reasignamos a otras llamadas \emph{qpp}. Se tienen que reasignar dado que la subrutina pide números reales y no matrices.

\begin{lstlisting}[frame=single] 
        
    qu(1) = u(1,i,j)
    qu(2) = u(2,i,j)
    qu(3) = u(3,i,j)
    qu(4) = u(4,i,j)

    ! Desacoplamiento de las primitivas
    call uprim(u,qp)

    qpp(:,i,j)=qp

    rho = qpp(1, i,j)
    vx  = qpp(2, i,j)
    vy  = qpp(3, i,j)
    P   = qpp(4, i,j)


\end{lstlisting}

La subrutina de desacoplamiento para las variables conservadas que se mencionan en esta sección recibirá las variables conservadas
y devolverá las primitivas. La subrutina \emph{newrap} se usará en particular 
(ver apéndice \ref{ap_newrap}) para resolver la ecuación \ref{eq_Newton_Rapson}.

\begin{lstlisting} [frame=single]
  
  m2 = sum(qu(2:3)**2)               ! v^2

  call newrap(qu, w, m2)

  alpha = m2 / w**2   ! alpha < 1 !
  u2  = alpha/(1.0-alpha)

  lor = sqrt(1.0 + u2)

  ! velocities
  qp(2:3) = qu(2:3) / w

  ! determination of the mass density
  qp(1) = qu(1)/lor

  ! thermal pressure
  chi = (w - qu(1)*(1.0+u2/(lor+1.0)))/(1.0+u2)

  qp(4)   = (gamma - 1.0)/gamma * chi

  qp(4) =max(qp(4),1d-10*qp(1))
\end{lstlisting}

Las variables \emph{qp} mostradas en el código son las primitivas mientras que las otras son auxiliares mostradas en el capitulo 
\ref{Cap_Desacoplamiento_de_las_ecuación  es_de_la_hidrodinámica_relativista}


\section{Diferencias finitas} \label{sec:Diferencias_finitas}
Ya hemos visto que ecuación  es describen la hidrodinámica, ahora nos toca resolverlas, dado que analíticamente es difícil, se va a hacer computacionalmente, para 
eso se hará uso del método numérico de las diferencias finitas.
Si tenemos una función $f(x)$ lo suficientemente diferenciable la podemos aproximar por el Teorema de Taylor  en la vecindad de un punto $x_0+\Delta x$ entonces si conocemos todas sus $n$ derivadas de la función $f(x)$ en el punto $x_0$ podemos aproximar de la siguiente manera

\begin{equation}\label{Serie_Taylor}
f\left( x_0 + \Delta x\right) = f\left( x_0 \right)+
\sum_n \frac{\left( \Delta x \right) ^2}{k!}f^{(k)} \left(x_0
\right)
\end{equation}

Si truncamos la serie de Taylor y quitamos los términos de segundo orden podemos escribir la ecuación \ref{Serie_Taylor} como:

\begin{equation}
f\left( x_0 + \Delta x \right) = f(x_0) - \Delta x f^{(1)} (x_0) + O \left( \Delta x \right)
\end{equation}

Despejando $f(x_0)$ queda lo que se conoce como diferencias finitas hacia adelante
\begin{equation}\label{fwd}
f_{fwd}^{'}=\frac{f\left(x + \Delta x \right) - f(x) }{\Delta x}=\frac{f_{i+1}-f_{i}}{\Delta x}
\end{equation}

también se puede hacer en el entorno $x_0- \Delta x$ y siguiendo los mismos pasos anteriores llegamos a lo que se le conoce como diferencia finita hacia atrás.


\begin{equation}\label{bwd}
f_{back}^{'}=\frac{f\left(x \right) - f(x - \Delta x) }{\Delta x}=\frac{f_{i}-f_{i-1}}{\Delta x}
\end{equation}

Si obtenemos el promedio de las ecuación es \ref{fwd} y \ref{bwd} obtenemos la central:

\begin{equation} \label{Centrada}
f_{central}^{'}=\frac{f\left(x + \Delta x\right) - f(x - \Delta x) }{2\Delta x}=\frac{f_{i+1}-f_{i-1}}{2 \Delta x}
\end{equation}



\subsection{Lax-Friederich}

Si consideramos la siguiente ecuación diferencial parcial

\begin{equation} \label{ecu_conser}
u_t+ f\left(u \right)_t=0
\end{equation}

Un método conservativo para resolver este tipo de ecuación diferencial es 

\begin{equation}\label{esquema conservativo}
u_i^{n+1} = u_i^{n} -\frac{\Delta t}{\Delta x} \left(f_{i-\frac{1}{2}} - f_{i+\frac{1}{2}}\right)
\end{equation}



Si hacemos la siguiente elección de flujo 

\begin{equation}
f_{i+\frac{1}{2}} = f_{i+\frac{1}{2}} \left(
u_{i} , u_{i+1}\right) =\frac{1}{2} \left(f_{i} + f_{i+1} \right) 
\end{equation}

Y para 

\begin{equation}
f_{i-\frac{1}{2}} = f_{i-\frac{1}{2}} \left(
u_{i} , u_{i-1}\right) =\frac{1}{2} \left(f_{i} - f_{i-1} \right) 
\end{equation}

Y lo sustituimos en la ecuación \ref{esquema conservativo} nos queda el siguiente resultado

\begin{equation}\label{ec_inestable}
u_i^{n+1} = u_i^{n} + \frac{1}{2}\frac{\Delta t}{\Delta x} \left(f_{i+1} - f_{i-1} \right)
\end{equation}

Pero esta solución es inestable, por el primer término del lado derecho de la ecuación, para hacerlo estable Peter Lax y Kurt Friedrichs sustituyeron este término por $(u_{i+1}^n-u_{i-1}^n)$  por lo que podemos reescribir la ecuación \ref{ec_inestable} como

\begin{equation}\label{ec_estable}
u_i^{n+1} =(u_{i+1}^n-u_{i-1}^n) + \frac{1}{2}\frac{\Delta t}{\Delta x} \left(f_{i+1} - f_{i-1} \right)
\end{equation}

En el código, la implementación de la ecuación \ref{ec_estable} en 2 dimensiones es la siguiente

\begin{lstlisting} [frame=single]

  call fluxes(nx,ny,neq,gamma,u,f,g,bound)

  do i=1,nx
    do j=1,ny
      up(:,i,j)=0.25*( u(:,i-1,j)+u(:,i+1,j)+u(:,i,j-1)+u(:,i,j+1) ) &
                  -dtx*0.5*(f(:,i+1,j)-f(:,i-1,j) ) &
                  -dty*0.5*(g(:,i,j+1)-g(:,i,j-1) )
    end do
  end do
  
\end{lstlisting}

Donde \emph{dtx, dty} son las derivadas con respecto a \emph{x}, \emph{y} y \emph{up} es la conservada posterior en el tiempo.



\subsection{HLL} \label{secc:HLL}

	Otro método para resolver las ecuación  es de la hidrodinámica es usar el método de Harten-Van-Leer. Definiendo el flujo numérico intercelda de Gudonov

\begin{equation}
F_{i+\frac{1}{2}}=F \left( U_{i+\frac{1}{2}} \right)
\end{equation}

Para el cual $U_{i+\frac{1}{2}}(0)$ tiene la misma solución para $U_{i+\frac{1}{2}}(x/t)$ con lo que el problema de Riemann se reduce a :

\begin{equation} \label{ecuación  _discreta_conservada}
\begin{array}{ll}
U_t + F \left( U \right)_x = 0 \\
U \left(x,0 \right) = 
\left\lbrace
\begin{array}{rr}
U_L \quad \textup{si} \quad x<0  \\
U_R \quad \textup{si} \quad x>0
\end{array}
\right.
\end{array}
\end{equation}

\begin{figure} %fuente de la imagen:libro del toro/
  \centering
    \includegraphics[width=0.5\textwidth]{Figuras/HLL_onda.png}
  \caption{Plano x-t que muestra que muestra un volumen definido}
  \label{fig:Plano x_t}
\end{figure}

Si consideramos un control de volumen $\left[x_L, x_R \right]\times \left[ 0 , T \right]$, tales que $x_L \leq TS_L$ y $x_R \geq TS_R$ (ver Figura \ref{fig:Plano x_t}) donde $S_L$ y $S_R$ son las velocidades de las ondas mas rápidas de los 
estados iniciales $U_L$ y $U_R$ respectivamente y $T$ es un tiempo definido. Si usamos la forma integral de la 
ecuación   \ref{ecuación  _discreta_conservada} en nuestro volumen definido $\left[x_L, x_R \right]\times \left[ 0 , T \right]$

\begin{equation*}\label{Forma_integral_conservadas}
\int_{x_L}^{x_R} \left[ U\left( x, T \right) -
 U\left( x, 0 \right) \right] dx = 
 \int_{0}^{T} \left[ F \left(U\left( x_L, t \right) \right) -
 F \left(U\left( x_R, t \right) \right) \right] dt 
\end{equation*}
Entonces
\begin{equation}\label{integral_consistencia}
\int_{x_L}^{x_R} U\left( x, T \right) dx =\int_{x_L}^{x_R} U\left( x, 0 \right) dx+
\int_{0}^{T}  F \left(U\left( x_L, t \right) \right)dt -
\int_{0}^{T}  F \left(U\left( x_R, t \right) \right) dt
\end{equation}

Usando las condiciones de la ecuación \ref{ecuación  _discreta_conservada} podemos evaluar la integral

\begin{equation*}
\int_{x_L}^{x_R} U\left( x, T \right) dx = 
x_R U_R -x_L U_L+T F_L-T F_R
\end{equation*}
Donde $F_L = F \left( U_L \right)$ y $F_R = F \left( U_R \right)$, entonces

\begin{equation}\label{Condición_de_consistencia}
\int_{x_L}^{x_R} U\left( x, T \right) dx = 
x_R U_R -x_L U_L+T \left( F_L- F_R \right)
\end{equation}

Si separamos ahora la ecuación \ref{integral_consistencia} en 3 integrales de la siguiente manera:

\begin{equation}
\int_{x_L}^{x_R} U\left( x, T \right) dx = 
\int_{x_L}^{T S_L} U \left(x, T \right)dx+
\int_{T S_L}^{T S_R} U \left(x, T \right)dx+
\int_{T S_R}^{x_R} U \left(x, T \right)dx
\end{equation}

Si ahora evaluamos el tercer y el primer término en el lado derecho, obtenemos:

\begin{equation}\label{condicion_consistencia_2}
\int_{x_L}^{x_R} U\left( x, T \right) dx =
\int_{T S_L}^{T S_R} U \left(x, T \right)dx+
\left( T S_L - x_L \right) U_L+
\left( x_L - T S_R \right) U_R
\end{equation}

Si combinamos la ecuación \ref{Condición_de_consistencia} y 
\ref{condicion_consistencia_2}

\begin{equation*}
x_R U_R -x_L U_L+T \left( F_L- F_R \right) =
\int_{T S_L}^{T S_R} U \left(x, T \right)dx+
\left( T S_L - x_L \right) U_L+
\left( x_L - T S_R \right) U_R
\end{equation*}

Entonces 

\begin{equation*}
\int_{T S_L}^{T S_R} U \left(x, T \right)dx=
\left( T S_L - x_L \right) U_L+ x_L U_L +
\left( x_L - T S_R \right) U_R-x_R U_R -
T \left( F_L- F_R \right)
\end{equation*}

Con lo que al final nos queda

\begin{equation} \label{ull_sin_promedio}
\int_{T S_L}^{T S_R} U \left(x, T \right)dx=
T \left( S_R U_R - S_L U_L + F_L - F_R \right)
\end{equation}

Si dividimos la ecuación \ref{ull_sin_promedio} por la diferencia de las velocidades máximas de las señales de las ondas, obtenemos el promedio de la función que esta entre las velocidades de la onda, entonces

\begin{equation}
\frac{1}{T \left( S_R -S_L \right)}\int_{T S_L}^{T S_R} U \left(x, T \right)dx =
\frac{S_R U_R - S_L U_L + F_L - F_R}{S_R - S_L}
\end{equation}

Si conocemos las velocidades de la onda podemos escribir la ecuación como 

\begin{equation}\label{u_hll}
U^{hll} = \frac{S_R U_R - S_L U_L + F_L - F_R}{S_R - S_L}
\end{equation}

Ahora si aplicamos la forma integral ( como en el caso de la ecuación \ref{Condición_de_consistencia}) a lado izquierdo de nuestro plano obtenemos lo siguiente

\begin{equation}
\int_{T S_L}^{0} U\left( x, T \right) dx = 
-T S_L U_L+T \left( F_L- F_{0L} \right)
\end{equation}
Donde $F_{0L}$ es el flujo a lo largo del eje $t$. Si despejamos $F_{0L}$, nos queda lo siguiente

\begin{equation}\label{ec F_0L}
F_{0L} = F_L - S_L U_L + \frac{1}{T}  \int_{T S_L}^{0} U\left( x, T \right) dx
\end{equation}

Esta última ecuación nos servirá para calcular los flujos usando el método de Harten-Van-Leer, el cual dividían el plano en tres espacios:

\begin{figure} %fuente de la imagen:libro del toro/
  \centering
    \includegraphics[width=0.5\textwidth]{Figuras/HLL.png}
  \caption{Aproximación de 3 estados distintos en el plano x-t, en el cual se trata de calcular los flujos en la región $U^{hll}$ limitados por las velocidades de señal de la onda}
  \label{fig:HLL}
\end{figure}

\begin{equation}
U \left(x,t \right) = 
\left\lbrace
\begin{array}{rr}
U_L \quad \textup{si} \quad \frac{x}{t}< S_L  \\
U_{hll} \quad \textup{si} \quad S_L< \frac{x}{t} <S_R \\
U_R \quad \textup{si} \quad  \frac{x}{t} > S_R
\end{array}
\right.
\end{equation}

Los flujos $F_R$ y $F_L$ pueden ser calculados directamente ya que solo dependen de $U_R$ y $U_L$ respectivamente pero $F_{hll} \neq F \left( U_{hll} \right)$, así que resolvemos la integral de la ecuación \ref{ec F_0L} para así obtener el flujo a través del eje t

\begin{equation*}
F_{hll} = F_L -S_L U_L+ \frac{1}{T}U_{hll}\left(0- TS_L\right)
\end{equation*}

Entonces
\begin{equation}\label{f_hll 1}
F_{hll} = F_L +S_L \left( U_{hll} -U_L \right)
\end{equation}

Si sustituimos \ref{u_hll} en \ref{f_hll 1} obtenemos:

\begin{equation*}
F_{hll} = F_L +S_L \left( \frac{S_R U_R - S_L U_L + F_L - F_R}{S_R - S_L} -U_L \right)
\end{equation*}
Entonces
\begin{equation*}
F_{hll} = \frac{F_L S_R -F_L S_L+S_L S_R U_R-S_L^2 U_L+S_L  F_L- S_L F_R-S_R S_L U_L + S_L^2 U_L}{S_R-S_L}
\end{equation*}

Eliminando términos semejantes queda
\begin{equation}
F_{hll} = \frac{S_R F_L -S_L F_R + S_L S_R \left(U_R-U_L \right)}{S_R -S_L}
\end{equation}

Con lo que el flujo intermedio de la celda de Godunov esta dado por:

\begin{equation} \label{fluxesHLL}
F_{i+\frac{1}{2}}^{hll} = 
\left\lbrace
\begin{array}{rr}
F_L \quad \textup{si} \quad 0 \leq S_L  \\
\frac{S_R F_L -S_L F_R + S_L S_R \left(U_R-U_L \right)}{S_R -S_L} \quad \textup{si} \quad S_L \leq 0 \leq S_R \\
F_R \quad \textup{si} \quad  0 \geq  S_R
\end{array}
\right.
\end{equation}

La implementación de esta parte se lleva a cabo en la subrutina \emph{ulax}.

\begin{lstlisting} [frame=single]
  call fluxesHLL(fhll, ghll)
      
        do i=1,nx
          do j=1,ny
            up(:,i,j) = u(:,i,j)                                     &
                        + 0.5*dtx*(fhll(:,i-1,j)-fhll(:,i+1,j))      &
                        + 0.5*dty*(ghll(:,i,j-1)-ghll(:,i,j+1))
          end do
        end do

\end{lstlisting}

Donde \emph{dtx}, \emph{dty} son las derivadas con respecto a \emph{x}, \emph{y} y \emph{fhll}, \emph{ghll} son los flujos calculados con el método de HLL.

La subrutina llama a \emph{fluxesHll}, que es la subrutina que se encarga de calcular los flujos usando el método de HLL, el siguiente código solo es una parte
y solo calcula la ecuación \ref{fluxesHLL}, las variables $i$, $j$ refieren a todos los puntos de la malla, los subíndices \emph{l}, \emph{r}, \emph{u} y \emph{d}
significan izquierda, derecha, arriba y abajo respectivamente y \emph{S} son las velocidades de las ondas mas rápidas de los estados iniciales \emph{U} 
, las variable \emph{dtx} es la derivada con respecto a \emph{x} y \emph{dty} la derivada con respecto a \emph{y}. Para ver toda la subrutina completa ver el apéndice \ref{ap_newrap}

\begin{lstlisting} [frame=single]
  do i=1,nx
  do j=1,ny
    
    if (0 .le. s_l(i,j)) then !less or equal 0<=sl
      fhll(:,i,j)=f_l(:, i,j)

    else if (s_l(i,j) .le. 0 .and. 0 .le. s_r(i,j)) then !sl<=0<=sr
  
      fhll(:,i,j)=(s_r(i,j)*f_l(:,i,j)-s_l(i,j)*f_r(:,i,j)+s_l(i,j)*s_r(i,j)*(u_r(:,i,j)-u_l(:,i,j)))/&
      (s_r(i,j)-s_l(i,j))


    else if (s_r(i,j) .le. 0) then !sr<=0
        fhll(:,i,j)=f_r(:,i,j)

    endif

    if(0 .le. s_d(i,j)) then
      ghll(:,i,j)= g_d(:,i,j)

    else if(s_d(i,j) .le. 0 .and. 0 .le. s_u(i,j)) then

      ghll(:,i,j)=(s_u(i,j)*g_d(:,i,j)-s_d(i,j)*g_u(:,i,j)+s_d(i,j)*s_u(i,j)*(u_u(:,i,j)-u_d(:,i,j)))/&
      (s_u(i,j)-s_d(i,j))

    else if(s_u(i,j) .le. 0) then
        ghll(:,i,j) = g_u(:,i,j)

    endif
  end do
end do
\end{lstlisting}



\section{Ecuaciones de Rankine-Hugoniot}\label{sec:Ecuacion_Rankine_Hugoniot}

\subsection{Rankine-Hugoniot Newtoniano}

Ya hemos visto como se resuelven las ecuación  es de la hidrodinámica, pero hay un problema, cuando un jet  choca contra el medio que lo rodea se genera una onda de choque
, esta onda de choque es una discontinuidad de la masa, presión y energía entre el jet y el medio que los rodea. Este problema se resuelve con las ecuación  es de 
Rankine-Hugoniot.
Vamos a considerar las relaciones que hay entre los 2 estados que se forman durante una onda de choque. Supongamos que la onda de choque se mueve 
hacia la derecha (ver Figura \ref{fig_move_shock}) sobre un flujo estacionario ($v = 0$) con una velocidad $v_s$, la presión y la densidad, que están enfrente de la onda, son asumidas como
$\rho_0$ y $P_0$ mientras que los flujos que están comprimidos detrás del frente de onda se mueven con una velocidad $v_p$ y que tienen densidad y presión $\rho_p$, $P_p$.

\begin{figure}
  \centering
  \includegraphics[scale=0.7]{./Figuras/Teoria/move_shock.png}
  \caption{Onda de choque en movimiento sobre un flujo estacionario, tanto la onda como el flujo que esta a la izquierda se mueven a la derecha}\label{fig_move_shock}
\end{figure}
  

Si consideramos nuestro sistema de referencia posicionado sobre la onda (ver Figura \ref{fig_stationary_shock}), usando las transformaciones de Galileo, las direcciones de las velocidades
se invierten y la velocidad del flujo que entra al choque es $v_s$ y la que sale es $v_s-v_p$

\begin{figure} 
  \centering
  \includegraphics[scale=0.7]{./Figuras/Teoria/stationary_shock.png}
  \caption{Al tomar nuestro sistema de referencia sobre la onda, nuestro choque en movimiento se transforma en una choque estacionario, y los
  flujos que están a la derecha e izquierda de la onda se mueven hacia la izquierda } \label{fig_stationary_shock}
\end{figure}

Las ecuación  es de Rankine-Hugoniot parten de las ecuación  es de la hidrodinámica considerando un sistema cerrado usando las ecuación  es \ref{conservación_masa_hidrodinamica}, \ref{conservación  _momento_hidrodinamica} y \ref{conservación  _energia_hidrodinamica} donde no varia con el tiempo, es decir, que $\dfrac{\partial}{\partial t}=0$, con lo que se pueden reescribir de la siguiente manera:


La conservación de masa
\begin{equation}
  \nabla \cdot \left( \rho \mathbf{u} \right)=0
\end{equation}

El momento
\begin{equation}
  \nabla \cdot \left( \rho \mathbf{u u} \right) + \nabla p = 0
\end{equation}

Ecuación de la energía

\begin{equation}
 \nabla \cdot \left[ \mathbf{u} \left( e+P \right) \right] = 0
\end{equation}

Considerando una dimensión:

\begin{equation}
\dfrac{d \left( \rho u \right)}{d x} = 0
\end{equation}

\begin{equation}
\dfrac{d \left( \rho u^2 \right)}{d x}+ \dfrac{d P}{d x}=0
\end{equation}

\begin{equation}
\dfrac{d \left( u\left[e +P \right] \right)}{d x} = 0
\end{equation}

Si integramos, la ecuación  es se van a igualar a constantes y podemos reescribir las ecuación  es del siguiente modo usando la ecuación de estado $e = \frac{1}{2} \rho u^2 + \frac{P}{\Gamma-1}$ donde $e \left[\frac{\mathrm{erg}}{\mathrm{cm^3}}\right]$ 
es la densidad de energía por unidad de volumen

\begin{equation}\label{RH_masa}
\rho_j u_j = \rho_m u_m
\end{equation}

\begin{equation}\label{RH_momento}
\rho_j u_{j}^{2}+P_j = \rho_m u_{m}^{2}+P_m
\end{equation}

\begin{equation}\label{RH_Energia}
\frac{1}{2} u_{j}^{2}+ \frac{\Gamma}{\Gamma-1} \frac{P_{j}}{\rho_{j}} =
 \frac{1}{2} u_{m}^{2}+ \frac{\Gamma}{\Gamma-1} \frac{P_{m}}{\rho_{m}}
\end{equation}

Donde $\rho \left[\frac{\mathrm{g}}{\mathrm{cm}^3}\right]$ representa la densidad, $u \left[\frac{\mathrm{cm}}{\mathrm{s} }\right]$ la velocidad,  $P \left[\frac{\mathrm{dyn}}{\mathrm{cm}^2} \right]$ la presión, $\Gamma$ el índice adiabático adimensional donde para velocidades ultrarrelativistas $\Gamma = 4/3$ y para no relativistas $\Gamma = 5/3$ y los índices \textit{j, m} que  relacionan a las propiedades del jet y del medio respectivamente. Usando la ecuación   \ref{RH_masa}, podemos definir el flujo como $j \equiv \rho_j u_j = \rho_m u_m$, sustituyendo en la ecuación \ref{RH_momento} podemos reescribirla como:

\begin{equation}\label{RH_momento_j}
P_{j}+\frac{j^2}{\rho_{j}}=P_{m}+\frac{j^2}{\rho_{m}}
\end{equation}

y la ecuación \ref{RH_Energia} llegamos a:

\begin{equation}\label{RH_Energia_j}
\frac{1}{2} \frac{j^{2}}{\rho_{j}^2}+\frac{\Gamma}{\Gamma-1} \frac{P_{j}}{\rho_{j}}=
\frac{1}{2} \frac{j^{2}}{\rho_{m}^2}+\frac{\Gamma}{\Gamma-1} \frac{P_{m}}{\rho_{m}}
\end{equation}

Despejando $j$ de la ecuación \ref{RH_momento_j} obtenemos:
\begin{equation}\label{j^2}
-j^{2}=\frac{P_{j}-P_{m}}{\frac{1}{\rho_{m}}-\frac{1}{\rho_{j}}}
\end{equation}

Ahora sustituyendo la ecuación en \ref{j^2} en \ref{RH_Energia_j} obtenemos:

\begin{equation*}
\frac{1}{2} \left( \frac{P_{m}-P_{j}}{\frac{1}{\rho_{j}}-\frac{1}{\rho_{m}}} \right)
\left(\frac{1}{\rho_{j}^{2}}-\frac{1}{\rho_{m}^{2}} \right)
=
\frac{\Gamma}{\Gamma-1}
\left( \frac{P_{m}}{\rho_{m}}-\frac{P_{j}}{\rho_{j}} \right)
\end{equation*}

$\Rightarrow$

\begin{equation*}
\frac{1}{2}	\left( P_{m} - P_{j} \right)
\left( \frac{1}{\rho_{j}}+\frac{1}{\rho_{m}} \right)
=
\frac{\Gamma}{\Gamma-1}
\left( \frac{P_{m}}{\rho_{m}}-\frac{P_{j}}{\rho_{j}} \right)
\end{equation*}

$\Rightarrow$

\begin{equation*}
\frac{1}{\rho_{m}} \left( \frac{1}{2} P_{m}- \frac{1}{2} P_{j}-
\frac{\Gamma}{\Gamma-1} P_{m} \right)
=
\frac{1}{\rho_{j}} \left( \frac{1}{2} P_{j}- \frac{1}{2} P_{m}-
\frac{\Gamma}{\Gamma-1} P_{j} \right)
\end{equation*}

$\Rightarrow$

\begin{equation*}
\frac{1}{\rho_{m}} \left[  \left(\frac{\Gamma + 1}{\Gamma - 1} \right) P_{m} + P_{j} \right]
=
\frac{1}{\rho_{j}} \left[  \left(\frac{\Gamma + 1}{\Gamma - 1} \right) P_{j} + P_{m} \right]
\end{equation*}

Con lo que nos queda:

\begin{equation}\label{RH_no_rel_choque_no_fuerte}
\frac{\rho_{m}}{\rho_j} =
\frac{\left( \Gamma +1 \right) P_{m}+ \left( \Gamma -1 \right) P_{j
}}{\left(\Gamma +1 \right) P_{j}+ \left( \Gamma -1 \right) P_{m}}
= \frac{u_j}{u_m}
\end{equation}

Si consideramos choque fuerte, es decir, $P_j \gg P_m$, implicaría que $P_m \simeq 0$, por lo que

\begin{equation}
\rho_j = \frac{\Gamma +1}{\Gamma-1} \rho_m 
\end{equation}
Tomando a $\Gamma = 5/3$ da

\begin{equation}
\rho_j = 4 \rho_m
\end{equation}

\begin{equation}
u_j = \frac{1}{4} u_m
\end{equation}

\begin{equation}
P_{j} = \frac{3}{4}\rho_m u_m^{2}
\end{equation}




\subsection{Rankine-Hugoniot Relativista}

% Si suponenmos las condiciones de Rankine-Hugoniot en un jet relativista, la presión interna del jet va a a ser mucho más grande que la del medio que lo está
% rodeando. Si colocamos nuestro marco de referencia sobre el jet, su masa será:

% \begin{equation}
%   M = \pi \rho v_j R^2 t
% \end{equation}

% Donde $M$, $\rho$, $v_j$ y $R$ son la masa, densidad, velocidad y radio del jet respectivamente, $t$ es el tiempo. Si multiplicamos todo por $c^2$,
% que es la velocidad de la luz al cuadrado, obtenemos la energía.

% \begin{equation}
%   E = Mc^2 = \pi \rho v_j R^2 t c^2
% \end{equation}



% Como tenemos nuestro sistema de referencia en la onda de choque inicial ($\gamma_j$) sobre nuestro objeto compacto. La energía del jet es:
% \begin{equation}
%   E = \gamma_{\infty} E
% \end{equation}
% Donde $\gamma_{\infty}$ es factor de lorentz con el que se mueve el jet cuando ya se ha alejado del objecto compacto.
% Y si ahora nos movemos al sistema de referencia visto desde la Tierra, la energía se vuelve:



% \begin{equation}
%   E = \gamma_{j} \gamma_{\infty} E_j
% \end{equation}

% Donde $\gamma_j$ es el factor de lorentz del jet y $E_j$ es la energía 
% del jet.

% Si derivamos con respecto al tiempo obtenemos la luminosidad.


% \begin{equation}
%   L = \dfrac{dE}{dt} = \gamma_{j} \gamma_{\infty}  \pi \rho v_j R^2 c^2
% \end{equation}

% Podemos obtener la densidad del jet despejando $\rho$ de la anterior ecuación

% \begin{equation}
%   \rho_j = \frac{L}{\gamma_{j} \gamma_{\infty}  \pi v_j R^2 c^2}
% \end{equation}

Considerando la conservación de masa:

\begin{equation}
  \rho_j v_j \gamma_j=  \rho_{\infty} v_{\infty} \gamma_{\infty}
\end{equation}

Donde $v_j$, $v_\infty$ es la velocidad del jet y del medio que lo rodea respectivamente, despejando $\rho_{\infty}$ obtenemos:

\begin{equation}\label{eq_masa_densidad_despejada}
  \rho_\infty = \rho_j \frac{v_j \gamma_j }{v_\infty \gamma_\infty}
\end{equation}

Si ahora consideramos la ecuación de momentos

\begin{equation}
  \left( e_j + P_j\right) v^2_j \gamma^2_j + P_j = \left( e_{\infty} + P_{\infty}\right) v^2_{\infty} \gamma^2_{\infty} + P_{\infty}
\end{equation}

La densidad de energía se compone de la energía interna del sistema y de la energía cinética

\begin{equation} \label{eq_densidad_de_energia}
  e = e_{int} + e_k
\end{equation}

La energía interna para un sistema ultra-relativista es $ P = \frac{1}{3}e_{int}$ lo que implica que $e_{int} = 3P$, mientras que la densidad de energía
cinética es $e_k = \rho c^2$ pero como estamos considerando a $c = 1$ esto se reduce a $e_k = \rho$. Si sustituimos todo esto en la ecuación
\ref{eq_densidad_de_energia} entonces nos queda:

\begin{equation} \label{eq_momentos_relativista_modificada}
  \left( \rho_j + 4P_j \right) v^2_j \gamma^2_j  +P_j = \left( \rho_\infty + 4P_\infty \right) v^2_{\infty} \gamma^2_{\infty}  +P_\infty
\end{equation}

Con la ecuación \ref{eq_masa_densidad_despejada} podemos modificar la ecuación \ref{eq_momentos_relativista_modificada}, con lo que nos queda:

\begin{equation}
  \left( \rho_j + 4P_j \right)v_j \gamma^2 +P_j = \left( \rho_\infty + 4P_\infty \right)v_\infty \gamma^2 +P_\infty
\end{equation}

y tomando un choque fuerte $P_j \gg P_\infty $ Podemos reducir la ecuación a 

\begin{equation} \label{eq_momentos_choque_fuerte}
  \left( \rho_j + 4P_j \right)v_j^2 \gamma^2 +P_j =  \rho_\infty  v_\infty^2 \gamma_{\infty}^{2} 
\end{equation}

Sustituyendo la ecuación \ref{eq_masa_densidad_despejada} en \ref{eq_momentos_choque_fuerte} obtenemos

\begin{equation} \label{eq_momentos_choque_fuerte}
  \left( \rho_j + 4P_j \right)v_j^2 \gamma^2 +P_j = \rho_j v_j v_\infty \gamma_j \gamma_\infty
\end{equation}

Suponiendo $\left( \rho_j + 4P_j \right)v_j^2 \gamma^2 \gg P_j $



\begin{equation}
  \left( \rho_j + 4P_j \right)v_j \gamma_j = \rho_\infty v_\infty  \gamma_\infty
\end{equation}

Despejando la presión obtenemos

\begin{equation}
  P_j = \frac{\rho_j}{4}\left(\frac{v_\infty}{v_j} \frac{\gamma_\infty}{\gamma_j}-1\right)
\end{equation}

Como $v_j \thickapprox v_\infty \thickapprox c$

\begin{equation}
  P_j = \frac{\rho_j}{4}\left( \frac{\gamma_\infty}{\gamma_j}-1\right)
\end{equation}

Para poder simular las condiciones de un jet, necesitamos saber la densidad, velocidad y presión del mismo, como los jet tienen velocidades cercanas a la
de la luz, se pueden tomar valores cercanos a esta constante. La densidad se obtendrá en la siguiente sección 


\subsection{Luminosidad}
Si suponemos las condiciones de Rankine-Hugoniot en un jet relativista y si colocamos nuestro marco de referencia sobre el jet, la energía será:

\begin{equation} \label{eq._energia_relativista}
  E = \gamma_{\infty} M c^2
\end{equation}

Donde $\gamma_{\infty}$ es factor de Lorentz con el que se mueve el jet cuando ya se ha alejado del objeto compacto y como sabemos 
que $L \equiv \dfrac{d E}{d t}$ entonces derivando la ecuación \ref{eq._energia_relativista} respecto al tiempo

Entonces tenemos 

\begin{equation} \label{eq._luminosidad_relativista}
  \dfrac{d E}{d t} = \gamma_{j} \dot{ M }  c^2
\end{equation}

La taza de flujo de masa $\dot{M}$ se puede escribir como

\begin{equation}
  \dot{M} = \rho_j u_j A_j
\end{equation}

Donde $\rho$ es la densidad, $u_j = v_j \gamma_{j}$ es la cuadrivelocidad,  y $A_j$ es el área, que en este caso, corresponde la parte esférica del jet
entonces la luminosidad se puede escribir como:

\begin{equation}
  L \equiv \dfrac{d E}{d t} = 4\pi r_j^2 \rho_j v_j \gamma_j \gamma_{\infty} c^2
\end{equation}

Despejando la densidad

\begin{equation}
  \rho = \frac{L}{4\pi r_j^2 v_j \gamma_j \gamma_{\infty} c^2}
\end{equation}

Al obtener la densidad ya podemos simular el jet dado que la luminosidad se puede obtener de detectores de Rayos $\gamma$ y las velocidades también ? preguntar a 
Diego


\section{Solución de Sedov-Taylor} \label{seccion_sedov_taylor}

%Ojo Julio
Una forma de verificar si el código funciona correctamente es verificar si una onda de choque esférica (o circular en 2D) sigue la solución  
{\color{blue} analítica y auto similar obtenida por Sedov y Taylor \cite{PAFD}}. 
Cabe señalar que la onda de choque esférica es descrita por la inyección de energía 
dentro de un radio el cual está inmerso en un medio en reposo. 

\subsection{Aproximación en función del análisis dimensional}
Si consideramos un choque fuerte,
es decir, $P_w \gg P_m$,
entonces las únicas variables que podemos tomar son $E$, $\rho_w$ y $t$ que es el tiempo.
Las dimensiones de estas cantidades son 

\begin{equation}
  \left[ \rho_w \right] = \frac{M}{L^3}
\end{equation}

\begin{equation}
  \left[E\right] = M \frac{L^2}{T^2}
\end{equation}

\begin{equation}
  \left[ t\right] = T
\end{equation}

La única cantidad que se puede construir que contenga la energía el tiempo y la densidad y que su dimensión sea
la longitud es:

\begin{equation}
  \left[ \left( \frac{Et^2}{ \rho_w } \right)^{\frac{1}{5}}\right] = L
\end{equation}

Por lo que para cualquier radio de onda de cualquier choque, debe depender de estas variables

\begin{equation}
  R(t) = \eta  \left( \frac{Et^2}{ \rho_w } \right)^{\frac{1}{5}} \varpropto t^{\frac{2}{5}}
\end{equation}

Por lo que la constante $\eta$ debe ser de la siguiente manera

\begin{equation}
  \eta \equiv \frac{r}{\left( \frac{Et^2}{\rho_w} \right)^{\frac{1}{5}}}
\end{equation}



\subsection{Aproximación en función de la conservación de la energía}

Otra aproximación es buscando el cociente de presiones y densidades en función del número Mach.
Podemos estimar el radio $R(t)$ de una onda de choque en función del tiempo, sabemos que el volumen de una esfera es $V = \frac{4}{3} \pi * R^3$ y la densidad de la onda es constante entonces el volumen $V = \frac{M}{\rho_w}$. La velocidad media de la onda es $v(t) \sim \frac{R(t)}{t} $
entonces podemos aproximar la energía cinética de la onda como 

\begin{equation}
	E_{kin} \sim \frac{1}{2} M v^2 \sim \rho_w R^3 \left( \frac{R}{t} \right) ^2 = \rho_w \frac{R^5}{t^2}
\end{equation}

Ahora la energía térmica de una explosión es 

\begin{equation} \label{eq_energia_termica}
	E_{th} \sim \frac{3}{2}PV
\end{equation}

Para encontrar la presión, primero se definirán algunas ecuaciones antes. La entalpía estancada se define en términos de la entalpía y la velocidad de la onda 

\begin{equation}
	h_0 = h + \frac{1}{2} v^2
\end{equation}

La cual la podemos reescribir como

\begin{equation} \label{eq_entalpía}
	h_0 = h + \left( 1 +\frac{1}{2} \frac{v^2}{h} \right)
\end{equation}

Ahora sabemos que la velocidad del sonido puede ser definida como 

\begin{equation} \label{eq_sonido_1}
	a = \sqrt{\frac{\Gamma P}{\rho}}
\end{equation}

y también como 

\begin{equation} \label{eq_sonido_2}
	a = \sqrt{(\Gamma-1) h}
\end{equation}

Donde $a$ es la velocidad del sonido, $h$ la entalpía, $v$ la velocidad, $P$ la presión y $\rho$ la densidad. también definimos el número de Mach como 

\begin{equation} \label{eq_Mach}
	M = \frac{v}{a}
\end{equation}

Si combinamos la ecuación \ref{eq_sonido_1} con la ecuación \ref{eq_Mach} y lo sustituimos en la ecuación \ref{eq_entalpía} nos queda la entalpía en función de la velocidad de la onda y del sonido

\begin{equation} \label{h_0_función_aire_mach}
	h_0 = \frac{a^2}{\Gamma - 1} \left( 1 + \frac{\Gamma - 1}{2} M^2 \right)
\end{equation}

Esta ecuación servirá más adelante. Volviendo a las ecuación  es de Rankine-Hugoniot podemos usar la ecuación  
\ref{RH_Energia} y \ref{entalpía_función_presion_densidad} para obtener la ecuación   de la energía en función de la entalpía, lo que nos va a quedar al final como

\begin{equation}\label{RH_entalpía}
  h_w +\frac{1}{2}v_w^2 = h_w +\frac{1}{2}v_w^2
\end{equation} 

Donde $ h $ es la entalpía y $v$ la velocidad, los subíndices $w$, $m$ corresponden a la onda y al medio respectivamente. Si dividimos la ecuación \ref{RH_momento} entre \ref{RH_masa}, y cambiando los subíndices $j$ por $w$ obtenemos la siguiente ecuación 

\begin{equation} \label{RH_momento_masa}
  v_w + \frac{P_w}{\rho_w v_w} = v_m + \frac{P_m}{\rho_m v_m}
\end{equation}

Usando la ecuación \ref{eq_sonido_1} en la ecuación \ref{RH_momento_masa} podemos reescribir la ecuación de 
la siguiente manera:

\begin{equation} \label{eq_diferencia_velocidades}
  v_w-v_m = \frac{1}{\Gamma} \left(\frac{a_m^2}{v_m} - \frac{a_w^2}{v_w} \right)
\end{equation}

Por simplicidad definimos la entalpía estancada como 

\begin{equation}
  h_w + \frac{1}{2} v_w = h_w + \frac{1}{2} v_w \equiv h_0
\end{equation}

Por lo que podemos definir a la velocidad del sonido como:

\begin{equation} \label{eq_sonido_onda}
  a_w = \left( \Gamma-1 \right) h_w = \left( \Gamma-1 \right) \left( h_0 -\frac{1}{2} v_w^2 \right)
\end{equation}

\begin{equation} \label{eq_sonido_medio}
  a_m = \left( \Gamma-1 \right) h_m = \left( \Gamma-1 \right) \left( h_0 -\frac{1}{2} v_m^2 \right)
\end{equation}

Sustituyendo las ecuación  es \ref{eq_sonido_onda} y \ref{eq_sonido_medio} en la ecuación \ref{eq_diferencia_velocidades}
nos queda

\begin{equation}
  v_w-v_m = \frac{\Gamma - 1}{\Gamma} \left[ \frac{h_0}{v_m} - \frac{h_0}{u_w} + \frac{1}{2} \left( v_w - v_m\right) \right]
\end{equation}

Dividiendo por $ \left( v_w - v_m \right)$

\begin{equation}
  1 = \frac{\Gamma - 1}{\Gamma} \left( \frac{h_0}{v_m v_w} + \frac{1}{2} \right)
\end{equation}

Que al final la podemos reescribir como

\begin{equation} \label{eq_denominador}
  \frac{1}{v_w v_m} = \frac{1}{\left( \Gamma -1\right) h_0} \frac{\Gamma + 1}{2}
\end{equation}

Otra ecuación que nos va a servir es la definición de Mach ya que como

\begin{equation}\label{velocidad_aire_mach}
  v_w^2 = a_w^2 M_w^2
\end{equation}

Entonces usando \ref{h_0_función_aire_mach} podemos reescribir la ecuación \ref{velocidad_aire_mach} como
como

\begin{equation} \label{eq_numerador}
  v_w^2 = M_w^2 \frac{\left( \Gamma -1 \right) h_0}{1+\frac{\Gamma - 1}{2} M_w^2}
\end{equation}

Usando la ecuación de momentos \ref{RH_masa} podemos reescribirla como

\begin{equation} \label{eq_fraccion_momentos}
  \frac{\rho_m }{\rho_w} = \frac{ v_w}{v_m} = \frac{v_w^2}{v_w v_m}
\end{equation}

Como ya sabemos cuanto vale el numerador (ec. \ref{eq_numerador}) y el denominador (ec. \ref{eq_denominador})
de la fracción, podemos definir el cociente de las densidades \ref{eq_fraccion_momentos} como 

\begin{equation}\label{eq_fraccion_momentos_función_mach}
  \frac{\rho_m}{\rho_w} = \frac{\left( \Gamma +1 \right) M_w^2}{2+ \left( \Gamma -1 \right) M_w^2}
\end{equation}

Entonces para encontrar las presiones combinamos las ecuación  es \ref{RH_momento} y \ref{RH_masa} tenemos

\begin{equation}
  P_m - P_w = \rho_w v_w^2 - \rho_m v_m^2 = \rho_w v_w^2 \left( 1 - \frac{v_m}{v_w}\right) = \rho_w v_w^2 \left( 1 - \frac{\rho_w}{\rho_m}\right)
\end{equation}

Usando la siguiente relación $\rho v^2 = \Gamma P M^2$, que se obtiene de las ecuación  es \ref{eq_sonido_1} y
\ref{eq_Mach},dividiendo por $P_w$ y sustituyendo  $\frac{\rho_m}{\rho_w}$ de la ecuación 
\ref{eq_fraccion_momentos_función_mach}
nos queda

\begin{equation} \label{eq_fraccion_Presiones}
  \frac{P_m}{P_w} = 1 + \frac{2 \Gamma}{\Gamma + 1} \left( M_w^2 -1\right)
\end{equation}

Usando la siguiente relación $ P_w = \frac{\rho a_w }{\Gamma}$ se puede obtener el límite del choque fuerte

\begin{equation}
  P_m  \simeq \frac{2 \rho_w v_w^2}{\Gamma + 1}
\end{equation}

Con esto ya podemos obtener la energía térmica de la ecuación \ref{eq_energia_termica} por lo que nos queda 

\begin{equation}
  E_{th} \sim P_m R^3 \sim \rho_w v_w^2 R^3 \sim \rho_w \frac{R^5}{t^2}
\end{equation}

Como tiene el mismo orden que el de la energía cinética, entonces la energía total que es una cantidad conservada, se espera que

\begin{equation}
  E_t = E_{kin} + E_{th} \sim \rho_w \frac{R^5}{t^2}
\end{equation}

Despejando el radio, obtenemos lo siguiente

\begin{equation} \label{eq_Radio_proporcional}
  R(t) \varpropto \left(\frac{E t^2}{\rho_w}\right)^{\frac{1}{5}}
\end{equation}

Vemos que la Figura \ref{fig_Radio_v_tiempo} parece una función logarítmica y su radio de expansión no crece muy rápidamente conforme avanza el tiempo,
Para tiempos $t = 4$ y $t = 100$, el radio es $R = 1.74$ y $R = 6.30$ respectivamente

\begin{figure}
  \centering
    \includegraphics[width=0.5\textwidth]{Figuras/Teoria/Radio_vs_tiempo.png}
  \caption{Radio de expansión de una onda de choque conforme avanza el tiempo } \label{fig_Radio_v_tiempo}
\end{figure}

Si derivamos la ecuación \ref{eq_Radio_proporcional}, obtenemos la velocidad

\begin{equation}
  \dfrac{de}{dt} = \frac{2}{5}\frac{R(t)}{t} \varpropto t^{\frac{-3}{5}}
\end{equation}

A diferencia del radio, la velocidad, se asemeja más a una asíntota que decrece muy rápidamente (ver \ref{fig_velocidad_vs_radio})

\begin{figure}
  \centering
    \includegraphics[width=0.5\textwidth]{Figuras/Teoria/Velocidad_vs_tiempo.png}
  \caption{Velocidad del radio de expansión de una onda de choque conforme avanza el tiempo } \label{fig_velocidad_vs_radio}
\end{figure}



% ======================================NUEVO CAPITULO ================================================
% ======================================NUEVO CAPITULO ================================================
% ======================================NUEVO CAPITULO ================================================
% ======================================NUEVO CAPITULO ================================================

\chapter{Verificación del código}\label{cap:Verificacion_del_codigo}


\section{Pruebas unidimensionales}

Para tener la seguridad de que el código hidrodinámico funcione bien se realizaron pruebas  unidimensionales (1D)
de un tubo de Sod ya sea en el régimen newtoniano así como en el relativista (en ambos regímenes se usó
el método de Lax así como el HLL, y se compararon los resultados con la solución analítica). 
Estas pruebas consisten en determinar como se comporta un gas ideal en un tubo 1D 
el cual se tiene inicialmente una discontinuidad. Esto es, que en un valor determinado,
el gas cambia abruptamente los valores de la densidad, presión y velocidad.




\subsection{Casos newtonianos}

Los valores de $\rho$, $p$ y $v$ de los modelos newtonianos se indican en el cuadro
\ref{Cuadro_parametros_sod_tube}. El primer  modelo (denominado caso 1 newtoniano), produce una onda 
de rarefacción\footnote{
  Una onda de rarefacción es la progresión de partículas que se aceleran lejos de una zona 
  comprimida. Siempre se mueven a regiones de mayor densidad y no presentan discontinuidades
} que se mueve a la izquierda y una onda de choque\footnote{
  Una onda de choque se moverá siempre a las zonas de menor densidad y presentará discontinuidades además
  de que su valor siempre será una constante.
} que se mueve a la derecha. La condición inicial\footnote{
  Dado que para el primer caso tendremos valores de densidad
  y presión más grandes que el lado derecho de la posición crítica ($x_0$), 
  podemos decir que del lado izquierdo 
  de $x_0$ será una una onda de rarefacción que se moverá a la izquierda, mientras que del lado
  derecho será una onda de choque que se moverá a la derecha.
} del gas consiste en tener el dominio 
unidimensional dividido en un lugar del dominio (en $x_0$) con ciertos valores de densidad ($\rho_l$), 
velocidad ($v_l$) y presión ($p_l$) del lado izquierdo\footnote{Los subíndices \emph{l} y \emph{r} 
vienen del ingĺés \emph{left} y \emph{right} que significan izquierda y derecha respectivamente.}, 
mientras que del lado derecho tendrán densidad ($\rho_r$), velocidad ($v_r$) y  presión ($p_r$). 
La elección del índice adiabático $\Gamma = 7/5$, un dominio $x \in [0,1]$ cm y $x_0 = 0.5$ cm 
se dio para poder simular y comparar los mismos modelos
mostrados por  Lora-Clavijo \emph{et al.} (2013).

\begin{table}[htbp]
  \begin{center}
  \begin{tabular}{|c|c|c|c|c|c|c|}
  \hline 
  \textbf{Caso} & \textbf{$p_l$} [$\text{dyn}/\text{cm}^2$] & \textbf{$p_r$} [$\text{dyn}/\text{cm}^2$]& \textbf{$v_l$} [$\text{cm}/\text{s}$]& \textbf{$v_r$} [$\text{cm}/\text{s}$]& \textbf{$\rho_l$} [$\text{g}/\text{cm}^3$]& \textbf{$\rho_r$} [$\text{g}/\text{cm}^3$] \\ 
  \hline 
  Caso 1 & 1.0  & 0.1  & 0.0 & 0.0 & 1.0  & 0.125 \\ 
  \hline 
  Caso 2 & 0.4  & 0.4  & -1.0 & 1.0 & 1.0  & 1.0  \\ 
  \hline 
  \end{tabular}
  \caption{\label{Cuadro_parametros_sod_tube} Valores iniciales de la presión ($p$), velocidad ($v$)
  y densidad ($\rho$), del lado izquierdo ($p_l, v_l, \rho_l$) y derecho ($p_r, v_r, \rho_r$)
  para los casos newtonianos. Para todos los
  casos el dominio espacial será $x \in [0,1]$ cm, la posición crítica sería
  $x_0 = 0.5$ cm y un índice adiabático $\Gamma = 7/5$}
  \end{center}
\end{table}


En la Figura \ref{caso_new_rar_shock} se muestra la evolución temporal en tres 
tiempos característicos $t = 0, \, 0.2, \, 0.4$
de la densidad, 
presión y velocidad para el caso 1 newtoniano usando el método Lax. En la Figura se ve como 
una onda de choque viaja hacia la derecha a través de la región de alta densidad.
En el panel superior izquierdo
notamos como para 0.2 segundos el perfil de densidad $\rho$ tiene dos brincos situados 
aproximadamente en 
x=0.70 cm, donde la densidad ($\rho \approx 0.48 \,  \text{g}/ \text{cm}^3$), 
cambia su densidad a $\rho \approx 0.28 \,  \text{g}/ \text{cm}^3$, mientras la velocidad y la presión 
mantienen los mismos valores 
($v \approx 0.99$ cm/s y $p \approx 0.3 \,  \text{dyn}/ \text{cm}^2 $).
En x=0.85 cm, la velocidad ($v \approx 0.99$ cm/s), la presión 
($p \approx 0.3  \,  \text{dyn}/ \text{cm}^2 $)  y La densidad 
($\rho \approx 0.28 \,  \text{g}/ \text{cm}^3$) 
tienen una discontinuidad y cambian sus valores a los valores iniciales del estado derecho del tubo
que no han sido afectados por la onda. 
La primera discontinuidad hace referencia a la onda de contacto que tiene la misma velocidad
que que los estados de la izquierda y la derecha $V_c = v_l = v_r$ donde $V_c$
es la velocidad de la onda de contacto por lo que esto hace que las presiones sean iguales 
$p_l = p_r$. Esta primera discontinuidad solo se 
presenta en el panel de las densidad, la otra discontinuidad que se presenta en los 3 páneles es la 
discontinuidad de la onda de choque.
Para el tiempo t=0.4 s
la discontinuidad de la onda de choque ya no se presenta en nuestro dominio de los 3 páneles, mientras 
que la cabeza de la onda de rarefacción todavía se muestra en $x \approx 0.01$ cm. La única discontinuidad 
presente es la de onda de 
contacto, donde $\rho$ cambia su valor de $\rho \approx 0.48 \,  \text{g}/ \text{cm}^3$
a $\rho \approx 0.28 \,  \text{g}/ \text{cm}^3$, mientras la velocidad $v \approx 0.99$ cm/s y 
presión $p \approx 0.3 \,  \text{dyn}/ \text{cm}^2 $ mantienen
sus mismos valores.
% Esta onda es más lenta que la de rarefacción ya que se encuentra en $x \approx 0.85$ cm.

  \begin{figure}
    \centering
        \subfigure{\includegraphics[scale = 0.16]{./Figuras/verificacion_del_codigo/caso_newtoniano/caso_new_rar_shock_rho.png}}
        \subfigure{\includegraphics[scale = 0.16]{./Figuras/verificacion_del_codigo/caso_newtoniano/caso_new_rar_shock_v.png}}
        \subfigure{\includegraphics[scale = 0.16]{./Figuras/verificacion_del_codigo/caso_newtoniano/caso_new_rar_shock_p.png}}
      \caption{\label{caso_new_rar_shock} Evolución temporal del caso 1 newtoniano (usando el método Lax) donde se muestra
      las magnitudes de densidad (arriba a la izquierda), presión (abajo) y 
      velocidad (arriba a la derecha). 
      Se usó un índice adiabático $\Gamma = 7/5$ .}
  \end{figure}

  El caso 1 newtoniano también fue resuelto utilizando el método HLL y usando varias resoluciones. 
  En la Figura 
  \ref{comparacion_analitico_newtoniano_caso_1}, se muestra la comparación de la $\rho, \, v$ y $p$
  entre los métodos Lax 
  y HLL a 10,000 píxeles. 
  El panel de arriba a la izquierda muestra la densidad $\rho = 1.0 \,  \text{g}/ \text{cm}^3$ 
  donde presenta un cambio en 
  $x \approx 0.24 $ cm y baja gradualmente hasta $x \approx 0.48$ cm donde la densidad ahora es $\rho \approx
  0.45 \,  \text{g}/ \text{cm}^3$.  Podemos ver que tanto Lax y HLL tienen el 
  mismo comportamiento
  que el método analítico. 
  Para las discontinuidades en $x \approx 0.74$ cm (contacto) y $x \approx 0.93$ cm (choque) cambian los valores de densidad $
  \rho \approx 0.48 \,  \text{g}/ \text{cm}^3$ a $\rho \approx 0.35 \,  \text{g}/ \text{cm}^3$ y 
  después a $\rho =0.1 \,  \text{g}/ \text{cm}^3$ 
  respectivamente. Aquí podemos ver
  tanto Lax como HLL presentan dificultad con la discontinuidad de contacto, 
  pero logra adaptarse muy bien para la discontinuidad de choque, en si ambos métodos reproducen 
  casi a 
  la perfección toda la región donde se desarrolla la discontinuidad en resoluciones de 10,000 píxeles.
  En el panel de arriba a la derecha se muestra como tanto HLL como Lax también reproducen la 
  solución analítica esperada para
  la velocidad. Ésta, al igual que la densidad cambian 
  sus valores gradualmente de $v = 0$ cm/s en $x \approx 0.20$ cm a $v \approx 0.99$ cm/s en 
  $x \approx 0.50$ cm,
  donde se mantiene contante hasta la discontinuidad de choque en $x \approx 0.93$ cm y regresa a $v = 0$ cm/s.
  En este punto, tanto Lax como HLL, no tiene ninguna dificultad al reproducir a discontinuidad de 
  choque.
  El panel de abajo a la izquierda muestra como HLL y Lax reproducen la solución 
  analítica para
  la presión, el cual cambia su valor 
  $p = 1.0 \,  \text{dyn}/ \text{cm}^2 $ en $x \approx 0.48$ cm
  bajando hasta $p = 0.4 \,  \text{dyn}/ \text{cm}^2 $ en $x \approx 0.48$ cm, 
  donde igual que la velocidad
  mantiene su mismo valor hasta $x \approx 0.93$ cm donde su valor decae a $p = 0.1 \,  \text{dyn}/ \text{cm}^2 $



El panel de abajo a la derecha de la Figura 
\ref{comparacion_analitico_newtoniano_caso_1} muestra como varia los resultados
usando Lax con distintas resoluciones ($n_x = 10^2, \, 10^3, \,10^4$), además, se muestra la solución
analítica.
Queda claro como 
conforme se incrementa la resolución (píxeles\footnote{Para fines prácticos, usaremos la palabra "píxel" para describir los $\Delta x$ en 
las que se divide nuestro dominio espacial, es decir, si el dominio tiene una resolución de $n$ 
píxeles, entonces, $\Delta x = 1/n$ }) los resultados numéricos se apegan más a la analítica. 
En específico, para el problema 1 newtoniano, a partir de resoluciones mayores a 1000 píxeles
los resultados 
se apegan mucho a la solución analítica.

\begin{figure}
  \centering
    \includegraphics[width=1.0\textwidth]{./Figuras/verificacion_del_codigo/rarefaction.png}
  \caption{Representación del caso 1 newtoniano a $t = 0.25$ s. Las magnitudes mostradas con
  la densidad (arriba a la izquierda), la velocidad (arriba a la derecha) y la presión (abajo a la 
  izquierda).
  El panel de abajo hacia la derecha muestra la comparación a distintas resoluciones para la densidad. Los demás páneles muestran
  las diferencias de los métodos HLL y Lax con una resolución de 10,000 píxeles.
  \label{comparacion_analitico_newtoniano_caso_1}}
\end{figure}

En el 2° caso newtoniano, los valores de la 
densidad y presión serán las mismas en ambos estados del tubo, pero las velocidades serán de igual magnitud
pero de sentido contrario. Esta condición inicial producirá 2 ondas de rarefacción. Una se moverá hacia
la izquierda y la otra hacia la derecha (véase el cuadro \ref{Cuadro_parametros_sod_tube} para más
detalles).
La Figura \ref{caso_new_rar_rar} muestra, usando
el método de Lax, la evolución temporal del 2° caso newtoniano. El panel de arriba a la izquierda muestra la densidad.
El panel arriba a la derecha muestra la velocidad y el de abajo, la presión. En todos los páneles se 
muestra $t = 0 \, , t = 0.2 \, ,t = 0.4$.

\begin{figure}
  \centering
      \subfigure{\includegraphics[scale = 0.16]{./Figuras/verificacion_del_codigo/caso_newtoniano/caso_new_rar_rar_rho.png}}
      \subfigure{\includegraphics[scale = 0.16]{./Figuras/verificacion_del_codigo/caso_newtoniano/caso_new_rar_rar_v.png}}
      \subfigure{\includegraphics[scale = 0.16]{./Figuras/verificacion_del_codigo/caso_newtoniano/caso_new_rar_rar_p.png}}
    \caption{ Igual que en la Figura \ref{caso_new_rar_shock} pero para el caso 2 newtoniano. 
    El índice adiabático sigue siendo $\Gamma = 7/5$
    \label{caso_new_rar_rar}.}
\end{figure}

Para los valores $t = 0$ s, la densidad ($\rho$) y la presión ($p$) son las mismas, 
mientras que para el panel de
la velocidad  tiene una discontinuidad en $x = 0.5$ cm. 
Para el tiempo $t = 0.2$ s, se puede ver una 
simetría ya que la cabeza de la onda de rarefacción tienen lugar en los puntos $x \approx 0.16$ cm
donde la densidad $\rho = 1.0 \,  \text{g}/ \text{cm}^3$ y 
la presión $p = 0.40 \,  \text{dyn}/ \text{cm}^2 $,
decrecen hasta $\rho \approx 0.25 \,  \text{g}/ \text{cm}^3$ y $p = 0.05 \,  \text{dyn}/ \text{cm}^2 $ en forma parabólica hasta conectar con la cola de 
la onda de rarefacción, mientras que la velocidad $v = -1.0 $ cm/s decrece linealmente en magnitud 
hasta $v = 0$ cm/s. Entre los puntos $x \approx 0.4$ cm y $x \approx 0.6$ cm, que es la zona de contacto, mantienen
un valor constante para conectar con la cola de la onda de rarefacción donde la densidad y presión 
crecen parabólicamente hasta $x \approx 0.84$ cm, mientras que la velocidad lo hace linealmente al mismo puntos.
Esa región es la cabeza de la onda y conecta con los estados iniciales del lado derecho del tubo, 
donde no ha habido perturbación, y por ende, $\rho = 1.0 \,  \text{g}/ \text{cm}^3$, 
$v = 1$ y $p = 0.4 \,  \text{dyn}/ \text{cm}^2 $.
Para el tiempo $t = 0.4$ s se aprecian las colas de las ondas de rarefacción
en los puntos $x  \approx 0.3$ y $x  \approx 0.7$, donde la densidad mantiene un valor contante 
$\rho \approx 0.25 \,  \text{g}/ \text{cm}^3$, la velocidad $v \approx 0.0$ cm/s y la presión $p = 0.05 \,  \text{dyn}/ \text{cm}^2 $.

La comparación del método de Lax y HLL se muestra en la Figura \ref{comparacion_analitico_newtoniano_caso_2}.
Como se puede ver en los páneles de densidad (arriba a la izquierda) en el punto $x \lesssim  0.05$ cm,
la densidad no esta perturbada y tiene un valor $\rho = 1.0 \,  \text{g}/ \text{cm}^3$,
la velocidad (arriba a la derecha)
$v = -1.0$ cm/s y la presión (abajo a la izquierda) $p = 0.4 \,  \text{dyn}/ \text{cm}^2 $. Tanto Lax como HLL 
se ajustan casi perfectamente 
para valores constantes. Después del punto de la cabeza de la onda los valores de la densidad y 
la presión empiezan a disminuir a $\rho \approx 0.2 \,  \text{g}/ \text{cm}^3$ y 
$p \approx 0.05 \,  \text{dyn}/ \text{cm}^2 $
en forma parabólica, mientras que la velocidad disminuye (en magnitud) linealmente hasta $v = 0$ cm/s.
Cabe señalar que los valores más bajos a los que disminuyen las magnitudes de densidad, velocidad y 
presión son en el punto $x \approx 0.33$ cm, el cual es la cola de la onda de rarefacción. Al comparar
las resoluciones, se nota que para resoluciones $\leq 1,000$ no se apega al método analítico sobre 
todo en la cabeza de la onda.

De $x \approx 0.33$ cm , la densidad, velocidad y presión 
mantienen valores constantes de $\rho \approx 0.1 \,  \text{g}/ \text{cm}^3$, $v \approx 0$ cm/s y 
$p = 0.05 \,  \text{dyn}/ \text{cm}^2 $ 
hasta llegar al punto $x \approx 0.66$ cm donde conectan con la cola de la onda de rarefacción. En
esta región donde se enlazan las colas de la onda Lax no tiene problemas con ningún tipo de resolución.
Del punto de la cola de la onda de rarefacción de la derecha, la densidad y presión empiezan a subir 
sus valores en forma parabólica hasta a $\rho  =  1.0 \,  \text{g}/ \text{cm}^3 $ y 
a $p = 0.4 \,  \text{dyn}/ \text{cm}^2 $, 
mientras la velocidad incrementa
linealmente a a $v = 1.0$ cm/s. El punto final hasta donde incrementan sus valores es la cabeza de la onda 
de rarefacción derecha el cual es $ x \approx 0.93 $ cm.
En el panel de abajo a la derecha muestra la solución analítica, así como los resultados obtenidos 
usando Lax\footnote{
  Se usó el método de Lax ya que es más rápido que HLL
}
con distintas resoluciones ($n_x = 10^2, \, 10^3, \,10^4$). Al igual que el caso 1 newtoniano,
conforme se incrementa la resolución, los resultados numéricos se apegan más a la solución analítica.
A partir de resoluciones mayores a 1000 los resultados se apegan mucho a la solución analítica.




\begin{figure}
  \centering
    \includegraphics[width=1.0\textwidth]{./Figuras/verificacion_del_codigo/rarefaction-rarefaction.png}
  \caption{ Igual que en la Figura \ref{comparacion_analitico_newtoniano_caso_1} pero para el 
  caso 2 newtoniano a $t = 0.25$ s. Los páneles vienen en 
  el mismo orden que la Figura \ref{comparacion_analitico_newtoniano_caso_1}.
  El panel de abajo hacia la derecha muestra la comparación a distintas resoluciones para la densidad. 
  Los demás páneles muestran las diferencias de los métodos HLL y Lax con una resolución 
  de 10,000 píxeles.
  \label{comparacion_analitico_newtoniano_caso_2}} 
\end{figure}

% Para los casos newtonianos se pueden usar Lax o HLL ya que ambos se ajustan con el método analítico
% (Lora-Clavijo \emph{et. al.} 2013)
% y con una resolución $\leq 1000$ píxeles.

Podemos concluir, para el caso newtoniano, que nuestros módulos de 
Lax como HLL reproducen de forma casi perfecta la solución analítica para este 
problema de Riemann
(Lora Clavijo \emph{et al.} 2013), ya sea en las regiones donde se muestra una discontinuidad
o donde cambia suavemente. 
Cabe destacar que el método Lax resultó ser 220 \% más rápido que HLL. 
\footnote{Probado con una resolución de 10,000 píxeles con un procesador AMD Ryzen 3 3300U de
2.10 Ghz} 
.

% ========================================================================================================
% ========================================================================================================


\subsection{Casos relativistas}

Para los casos relativistas, se tomarán valores a partir de los cuales $v \rightarrow c$,
y además se requiera usar el índice adiabático\footnote{En esta tesis,
para fines explícitos, se tomará gamma mayúscula ($\Gamma$) como el índice adiabático. Mientras que 
gamma minúscula ($\gamma$) como el factor de Lorentz.} $\Gamma = 4/3$. Los valores con los que se 
toman los casos del caso relativista están en el cuadro \ref{Cuadro_parametros_sod_tube_rel} y 
cabe señalar que la velocidad de la luz está normalizada a $c = 1$  y que el dominio
será $x \in [0,1]$, $x_0 = 0.5$ cm.

\begin{table}[htbp]
  \begin{center}
  \begin{tabular}{|c|c|c|c|c|c|c|}
  \hline 
  \textbf{Caso} & \textbf{$p_l$} [$\text{dyn}/\text{cm}^2$] & \textbf{$p_r$} [$\text{dyn}/\text{cm}^2$] & \textbf{$v_l$}/c & \textbf{$v_r$}/c  & \textbf{$\rho_l$} [$\text{g}/\text{cm}^3$]& \textbf{$\rho_r$} [$\text{g}/\text{cm}^3$]\\ 
  \hline 
  Caso 1 & 13.33  & 0.0  & 0.0 & 0.0 & 10  & 1 \\ 
  \hline 
  Caso 2 & 0.05  & 0.05  & -0.2 & 0.2 & 0.1  & 0.1  \\ 
  \hline 
  \end{tabular}
  \caption{\label{Cuadro_parametros_sod_tube_rel} Valores iniciales 
  de la presión ($p$), velocidad ($v$)
  y densidad ($\rho$), del lado izquierdo ($p_l, v_l, \rho_l$) y derecho ($p_r, v_r, \rho_r$)
  para los casos relativistas. Para todos los
  casos el dominio espacial será $x \in [0,1]$, la 
  posición crítica será $x_0 = 0.5$ y un índice adiabático $\Gamma = 4/3$}
  \end{center}
\end{table}

Tanto en el caso 1 relativista como el caso 2 relativista, la evolución temporal se hará usando 
el método de HLL ya que es más preciso que el método de Lax\footnote{
  Lax falla al tratar de simular las condiciones iniciales de este problema.
}. 
Además de que necesita de altas resoluciones $\geq 10,000$ píxeles para poder funcionar.

\begin{figure}
  \centering
      \subfigure{\includegraphics[scale = 0.16]{./Figuras/verificacion_del_codigo/caso_relativista/caso_rel_rar_shock_rho.png}}
      \subfigure{\includegraphics[scale = 0.16]{./Figuras/verificacion_del_codigo/caso_relativista/caso_rel_rar_shock_v.png}}
      \subfigure{\includegraphics[scale = 0.16]{./Figuras/verificacion_del_codigo/caso_relativista/caso_rel_rar_shock_p.png}}
    \caption{Igual que en la Figura \ref{caso_new_rar_shock} pero para el caso 1 relativista. A diferencia 
    del caso newtoniano, en el relativista se usa un índice 
    adiabático $\Gamma = 4/3$. \label{caso_rel_rar_shock_1}}
\end{figure}

En la Figura \ref{caso_rel_rar_shock_1} se muestra la evolución temporal de la $\rho$, $p$ y $v$
cuando $t = 0$ s el panel de la densidad 
(arriba a la izquierda) 
y presión (abajo) presentan una discontinuidad en $x = 0.5$ cm donde $\rho = 10 \,  \text{g}/ \text{cm}^3$
y $p = 13.33 \,  \text{dyn}/ \text{cm}^2 $ del lado izquierdo, mientras que del lado derecho $\rho = 0  \,  \text{g}/ \text{cm}^3$
y $p = 0 \,  \text{dyn}/ \text{cm}^2 $. La velocidad no presenta discontinuidad y mantiene su 
velocidad nula en toda la región del tubo. 
Al tiempo $t =0.2$ s la densidad y presión 
descienden linealmente, mientras que la velocidad asciende en $x = 0.42$ cm
que viene siendo la cabeza de la onda de rarefacción. El punto donde alcanzan 
sus máximos y mínimos es la cola de la onda que esta situada en el punto $x \approx 0.58$ cm,
la densidad y presión llegan a $\rho \approx 1.12 \,  \text{g}/ \text{cm}^3$ y 
$p \approx 1.12\,  \text{dyn}/ \text{cm}^2 $ mientras que las velocidad
se acerca a la de la luz $v \approx 0.75c$. La densidad vuelve a subir en el punto $x \approx 0.63$ cm
y llega a $\rho = 5.6  \,  \text{g}/ \text{cm}^3$ mientras que la velocidad y presión se mantienen iguales. Para el punto
$x \approx 0.68$ cm, la densidad, presión y velocidad bajan a valores nulos que es la zona del tubo que
no ha sido perturbada por la onda.
Para el tiempo $t = 0.4$ s, la densidad y presión, que mantienen sus valores de 
$\rho = 10 \,  \text{g}/ \text{cm}^3$ y 
$p = 13.33 \,  \text{dyn}/ \text{cm}^2 $ bajan sus valores en $x \approx 0.25$ cm, mientras que la velocidad , la cual es nula,
asciende. En el punto $x \approx 0.68$ cm la presión alcanza $p \approx 1.12\,  \text{dyn}/ \text{cm}^2 $ y la densidad
$\rho \approx 1.12 \,  \text{g}/ \text{cm}^3$, mientras que la velocidad sube a $v \approx 0.75c$. 
La densidad vuelve a subir
en $x \approx 0.75$ cm y llega a $\rho \approx 0.65 \,  \text{g}/ \text{cm}^3$ y 
se mantiene así hasta el punto $x \approx 0.83$ cm
donde la densidad, así como la velocidad y presión descienden sus valores a nulos.

En la Figura \ref{caso_rel_rar_shock} se muestra la comparación entre la solución obtenida con Lax
(1000 píxeles) y la solución analítica
el panel de arriba a la izquierda muestra la densidad, el de 
arriba a la derecha la velocidad, la presión es el de abajo a la izquierda. Cabe recalcar que  el 
tiempo mostrado es  $t = 0.35$ s.
A la región $x \lesssim 0.33$ cm, la densidad tiene un valor $\rho = 10 \,  \text{g}/ \text{cm}^3$, 
la presión $p = 13.33 \,  \text{dyn}/ \text{cm}^2 $
y la velocidad es nula se puede ver que HLL se ajusta casi perfectamente para los valores constantes
sin importar la resolución. 
En el punto $x \approx 0.33$ cm se localiza la cabeza de la onda donde la presión y la densidad 
empiezan a decaer en forma parabólica, mientras que que la velocidad comienza subir en forma linealmente.
El punto donde la densidad y presión alcanzan su mínimo mientras que la velocidad, su máximo
se localiza en $x \approx 0.62$ cm. La resolución para esta región es $\geq 500$ 
píxeles. En el punto $x \approx 0.75$ cm únicamente 
la densidad tiene una discontinuidad, al cual se le conoce como discontinuidad de la onda de contacto
debido a esta región la resolución a tomar en cuenta debe ser $\geq 1,000$ o incluso $\geq 10,000$.
En esta discontinuidad la densidad pasa de $ \rho \approx 3 \,  \text{g}/ \text{cm}^3$ a
$ \rho \approx 9 \,  \text{g}/ \text{cm}^3$ mientras que la presión 
y velocidad se mantienen en $p \approx 2\,  \text{dyn}/ \text{cm}^2 $ 
y $v = 0.75$ c. La última discontinuidad que es la de choque,
todas las magnitudes ($\rho$, $v$ y $p$) disminuyen a su valor nulo y la resolución a tomar en cuenta
puede ser $\geq 1,000$ píxeles dado que HLL no presenta dificultades con esta discontinuidad.

En el panel de abajo a la derecha muestra la solución analítica, así como los resultados obtenidos 
usando HLL con distintas resoluciones ($n_x = 10^2, \, 10^3, \,10^4$).  
En específico, para el problema 1 relativista, a partir de resoluciones mayores a 1000 los
resultados se apegan mucho a la solución analítica.


\begin{figure}
  \centering
    \includegraphics[width=1.0\textwidth]{./Figuras/verificacion_del_codigo/caso_relativista/caso_rel_rar_shock.png}
  \caption{Igual que en la Figura \ref{comparacion_analitico_newtoniano_caso_1} pero para el caso 1 
  relativista 
  a $t = 0.35$ s. El panel de abajo 
  hacia la
  derecha muestra la comparación a distintas resoluciones para la densidad. Los demás páneles muestran
  las diferencias de los métodos HLL y Lax con una resolución de 10,000.}\label{caso_rel_rar_shock}
\end{figure}

%=========================================================================================================

El caso 2 relativista representa 2 ondas de rarefacción, a diferencia del caso 1 las velocidades que 
alcanzan las 
ondas no son velocidades relativistas.
\begin{figure}
  \centering
      \subfigure{\includegraphics[scale = 0.16]{./Figuras/verificacion_del_codigo/caso_relativista/caso_rel_rar_rar_rho.png}}
      \subfigure{\includegraphics[scale = 0.16]{./Figuras/verificacion_del_codigo/caso_relativista/caso_rel_rar_rar_v.png}}
      \subfigure{\includegraphics[scale = 0.16]{./Figuras/verificacion_del_codigo/caso_relativista/caso_rel_rar_rar_p.png}}
    \caption{Igual que en la Figura \ref{caso_new_rar_shock} pero para el caso 2 relativista. 
    El panel de arriba a la izquierda muestra la densidad.
    El de arriba a la derecha la velocidad y el de abajo la presión}\label{caso_rel_shock_shock}
\end{figure}
A diferencia del caso 1 relativista, el caso 2 relativista no presenta discontinuidades,
por lo que se pueden usar resoluciones $\leq 100.$ píxeles. En la Figura 
\ref{caso_rel_shock_shock},  
el panel de arriba a la izquierda muestra la densidad, el de arriba a la derecha la velocidad
y el de abajo la presión.
Al igual que el caso newtoniano al simular las 2 ondas presenta una simetría con respecto a la 
discontinuidad.
En el tiempo $t = 0$ s, tanto la densidad como la presión mantienen los mismos valores 
$\rho =0.1 \,  \text{g}/ \text{cm}^3$
y $p = 0.05$. mientras que la velocidad tiene una discontinuidad en $x = 0.5$ cm donde la parte del lado
izquierdo de la discontinuidad es $v = -0.2$ c,  (el signo negativo solo muestra que las ondas de choque
tienen una dirección opuesta entre si). 
mientras que del lado derecho es $v = 0.2$ c.
Al tiempo $t = 0.2$ s la cabeza de la onda de rarefacción se presenta en los punto 
$x \approx 0.35$ cm donde los valores de la presión ($p = 0.05$) y densidad 
($\rho =0.1 \,  \text{g}/ \text{cm}^3$) comienzan a descender linealmente,
mientras que la velocidad ($v = 0.2$ c) desciende hasta el punto $x \approx 0.45$ cm. En este punto los valores de la densidad, presión y velocidad
son $\rho \approx 0.065 \,  \text{g}/ \text{cm}^3$, $p \approx 0.028\,  \text{dyn}/ \text{cm}^2 $ 
y $v \approx 0$ c. Los valores cambian en 
$ x \approx 0.55$ cm
que es la cola de la onda de rarefacción derecha, y comienzan a subir
hasta $x \approx 0.65$ cm donde ahora la densidad es 
$\rho = 0.1 \,  \text{g}/ \text{cm}^3$, la presión $p = 0.05 \,  \text{dyn}/ \text{cm}^2 $ y 
la velocidad $v = 0.2$ c
Para el tiempo $t = 0.4$ s se sigue manteniendo el mismo sistema que al tiempo $t =0.2$ s, ya que en
$x \approx 0.25$ cm los valores de la densidad ($\rho = 0.1 \,  \text{g}/ \text{cm}^3$), 
presión ($p = 0.05 \,  \text{dyn}/ \text{cm}^2 $) 
y velocidad ($v = -0.2$ c) descienden en forma lineal hasta el punto $x \approx 0.35$ cm, donde 
se mantienen constantes los valores ($\rho \approx 0.065  \,  \text{g}/ \text{cm}^3$, 
$p \approx 0.028\,  \text{dyn}/ \text{cm}^2 $ y $v \approx 0$ c) 
hasta el punto $x \approx 0.65$ cm donde vuelven a ascender linealmente
hasta el punto $x = 0.75$ cm donde los valores regresan a $\rho = 0.1 \,  \text{g}/ \text{cm}^3$, 
$p = 0.05 \,  \text{dyn}/ \text{cm}^2 $ y la velocidad 
cambia de sentido $v = 0.2$ c.


\begin{figure}
  \centering
    \includegraphics[width=1.0\textwidth]{./Figuras/verificacion_del_codigo/caso_relativista/caso_rel_rar_rar.png}
  \caption{Igual que en la Figura \ref{comparacion_analitico_newtoniano_caso_1} pero para el caso 2 
  relativista. Se usó un índice adiabático $\Gamma = 4/3$ y $t = 0.25$ s.
  } \label{caso_rel_shock_shock_2}
\end{figure}

En la Figura \ref{caso_rel_shock_shock_2}, el panel de arriba a la izquierda muestra el perfil de la 
densidad, el de arriba a la derecha la velocidad, el de abajo a la izquierda la presión. Esos 3 muestran
una comparación entre el método analítico y el método numérico (HLL) usando una resolución de 
10,000 píxeles. 
En $x \lesssim 0.33$ cm los valores de la densidad, presión y velocidad son 
$\rho = 0.1 \,  \text{g}/ \text{cm}^3$, $p = 0.05 \,  \text{dyn}/ \text{cm}^2 $
y $v = -0.2$ c. HLL para resoluciones $\geq 500$ se ajusta perfectamente al analítico. Cuando los 
valores descienden en $x \approx 0.37$ cm a $\rho \approx 0.075 \,  \text{g}/ \text{cm}^3$, 
$v \approx 0$ c y $p \approx 0.025\,  \text{dyn}/ \text{cm}^2 $. En
esta región podemos ver que para resoluciones $\leq 100$ pixeles se tiene dificultades para apegarse al
método analítico sobre todo en las regiones de la cabeza y cola de la onda de rarefacción. En el punto $x \approx 0.63$  cm tanto la presión como la densidad y la velocidad vuelven a subir
sus valores hasta el punto $x \approx 0.65$ cm donde alcanzan los valores que tenial del lado izquierdo
donde la densidad ahora es $\rho = 0.1 \,  \text{g}/ \text{cm}^3$, la presión $p = 0.05 \,  \text{dyn}/ \text{cm}^2 $ 
y la velocidad $v = 0.2$ c.
En el panel de abajo a la derecha se muestra la solución analítica, así como los resultados obtenidos 
usando HLL con distintas resoluciones ($n_x = 10^2, \, 10^3, \,10^4$). 
Al igual que el caso 1 relativista,
conforme se incrementa la resolución, los resultados numéricos se apegan más a la solución analítica.
A partir de resoluciones mayores a 1000 los resultados se apegan mucho a la solución analítica.
Por lo tanto podemos concluir que nuestro módulo HLL se ajusta perfectamente para el problema de 
Riemann relativista con resoluciones $\geq 500$ píxeles.

% ===============================================================================================
% ===============================================================================================


\section{Pruebas bidimensionales}

\subsection{Caso newtoniano}

En esta sección se probó el código que se desarrolló en la problema de Sedov-Taylor en la que se libera una gran cantidad de energía en un volumen muy pequeño. 
Esto sucede en el medio interestelar cada vez que se dispara una supernova. 
En este caso se inyecta instantáneamente una gran cantidad de energía \emph{E} en un medio ambienta de densidad uniforme $\rho_m$. 
Posteriormente,  un frente de choque esférico se expandirá hacia el medio ambiente.

\begin{table}[htbp]
  \begin{center}
  \begin{tabular}{|c|c|c|c|c|c|c|c|c|}
  \hline 
  \textbf{Unidades} & \textbf{$p_s$} [$\text{dyn}/\text{cm}^2$] & 
  \textbf{$p_m$} [$\text{dyn}/\text{cm}^2$] & 
  \textbf{$v_{xs}$}/c & \textbf{$v_{xm}$}/c  & \textbf{$v_{ys}$}/c & \textbf{$v_{ym}$}/c  & 
  \textbf{$\rho_s$} [$\text{g}/\text{cm}^3$]& 
  \textbf{$\rho_m$} [$\text{g}/\text{cm}^3$]\\ 
  \hline 
  valores & $2.01 \times 10^{-6}$ & $8.97 \times 10^{-10}$  & 0 & 0 & 0 & 0 &  $3.48 \times 10^{-24}$  & $1.67 \times 10^{-24}$ \\ 
  \hline 
  \end{tabular}
  \caption{\label{Cuadro_parametros_choque_2D_newtoniano} Valores iniciales 
  de la presión ($p$), velocidad ($v$)
  y densidad ($\rho$), del lado interno ($p_s, v_{xs}, v_{ys}, \rho_s$) y 
  externo ($p_m, v_{xm}, v_{ym}, \rho_m$) de la explosión de Sedov-Taylor
  para el caso newtoniano. Los subíndices \emph{s,m} corresponden a \emph{supernova} y \emph{medio} respectivamente. Para todos los
  casos el dominio espacial será $x, y \in [0,4 \times 10^{20}]\times[0,4 \times 10^{20}] \, \text{cm}$, el radio de la onda
  fue de $r = 2$ parsecs y se tomó un índice adiabático $\Gamma = 5/3$.}
  \end{center}
\end{table}


Los valores típicos de una supernova con energía $E_{s}  \approx  1 \times 10^{52}$ ergs estan mencionados en el Cuadro \ref{Cuadro_parametros_choque_2D_newtoniano}. Se tomó un radio de 
2 parsec. Se usó un índice politrópico de $\Gamma = 5/3$. El tiempo de integración fue $t = 8.1 \times 10^10$ s. La Figura \ref{fig:prueba_2d_newtoniana} se muestra la evolución de la onda. El panel de arriba a la izquierda
muestra la onda de choque al tiempo $t = 0$. El panel de arriba a la derecha muestra la evolución al tiempo $t \approx 10^{10}$ s, se puede observar un incremento del radio con $r \approx 1.6$ pc, asi como
un incremento simétrico. Tambien se observa una sobredensidad de $\rho \approx 3.5 \times 10^{-24} \text{g} \, \text{cm}^{-3}$ en el perimetro de la onda mientras que en el resto la densidad decae a 
$\rho \approx 10^{-24} \text{g} \, \text{cm}^{-3}$. En el panel de abajo, se muestra una onda mas grande con un radio de expansión $r \approx 4.5$ pc al tiempo $t = 8.1 \times 10^{10}$. Sigue manteniendo una simetría y
una sobredensidad $\rho \approx 4.5 \times 10^{-24} \text{g} \, \text{cm}^{-3}$, lo cual muestra que va a arrastrando mas material y que decae en el centro de la onda 
rápidamente a $\rho \approx 10^{-24} \text{g} \, \text{cm}^{-3}$


\begin{figure}
  \centering
      \subfigure{\includegraphics[scale = 0.3]{./Figuras/verificacion_del_codigo/prueba_2d_newtoniana/0_NR.png}}
      \subfigure{\includegraphics[scale = 0.3]{./Figuras/verificacion_del_codigo/prueba_2d_newtoniana/11_NR.png}}
      \subfigure{\includegraphics[scale = 0.3]{./Figuras/verificacion_del_codigo/prueba_2d_newtoniana/99_NR.png}}
    \caption{Explosión de la onda de choque en 2 dimensiones usando un mapa de densidad. El panel de arriba a la izquierda muestra el tiempo inicial $t=0$ s con radio de onda de choque de 2 parsec,
    el de arriba a la derecha  al $t \approx 10^{10}$ s con un radio de $r \approx 1.6$ pc y el panel de abajo muestra el tiempo $t = 8.1 \times 10^{10}$ s con un radio de expansión de $r \approx 4.5$ pc. Todos los 
    páneles tienen una resolución 1000X1000 pixeles. Con un índice politrópico de $\Gamma = 5/3$. Los valores internos de la onda fueron $\rho_{s} = 3.48 \times 10^{-24} \text{g cm}^3$ y 
    $P_{s} = 2.01 \times 10^{-6} \text{dyn}$
    mientras que los del medio ambiente fueron $\rho_{m} = 1.67 \times 10^{-24} \text{g cm}^3$ y $P_{m} = 8.97 \times 10^{-10} \, \text{dyn}$.
    \label{fig:prueba_2d_newtoniana}}
\end{figure}


Al graficar los cambios del radio en el tiempo de integración, se puede obtener la siguiente gráfica (ver Figura \ref{fig:Expansion_HLL_NR}). Se usó una regresión exponencial para poder modelar la expansion del radio 
de la explosión como $r(t) = t^a$ donde nuestro coeficiente $a = 0.413$. Se calculó coeficiente de determinación $R^2= 0.985$ lo cual nos dice que nuestro modelo se ajusta perfectamente a los puntos de la gráfica. Los
puntos en forma de cruz mas grandes corresponden a los radios de la Figura \ref{Cuadro_parametros_choque_2D_newtoniano}. Al comparar nuestro modelo con la teoría podemos notar una discrepancia del 5 \%.


\begin{figure}
  \centering
  \includegraphics[width=1\textwidth]{./Figuras/verificacion_del_codigo/prueba_2d_newtoniana/Expansion_HLL_NR_res_1000.png}
    \caption{Crecimiento del radio de la onda de choque vs tiempo. Las cruces verdes muestran el tamaño del radio de la onda de choque se obtiene en un intervalo de tiempo de $t=8.1 \times 10^{10}$ s.
    La línea sólida roja muestra una regresión exponencial con ecuación $r(t) = t^{0.413}$ con un coeficiente de determinación $R^2= 0.985$. Las cruces más remarcadas corresponden los radios que se muestran en la Figura 
    \ref{fig:prueba_2d_newtoniana}}
    \label{fig:Expansion_HLL_NR}
\end{figure}






\subsection{Caso relativista}
Ya que hemos visto el código funciona correctamente para pruebas relativistas en una dimensión se hará una prueba de una onda de choque relativista en 2 dimensiones y se verificará que se comporta como la solución 
analítica auto-similar de una prueba de un tubo de choque bidimensional. El dominio espacial será $x, y \in [0,1]\times[0,1] \, \text{cm}$, el radio de la onda será $r = 0.5 \, \text{cm}$, se considerará un índice 
adiabático $\Gamma = 4/3$, y los valores que la onda bidimensional tendrá están en el cuadro \ref{Cuadro_parametros_choque_2D}. 

\begin{table}[htbp]
  \begin{center}
  \begin{tabular}{|c|c|c|c|c|c|c|c|c|}
  \hline 
  \textbf{Unidades} & \textbf{$p_i$} [$\text{dyn}/\text{cm}^2$] & 
  \textbf{$p_o$} [$\text{dyn}/\text{cm}^2$] & 
  \textbf{$v_{xi}$}/c & \textbf{$v_{xo}$}/c  & \textbf{$v_{yi}$}/c & \textbf{$v_{yo}$}/c  & 
  \textbf{$\rho_i$} [$\text{g}/\text{cm}^3$]& 
  \textbf{$\rho_o$} [$\text{g}/\text{cm}^3$]\\ 
  \hline 
  Valores & 13.33  & 0.0  & 0.0 & 0.0 & 0.0 & 0.0 & 10.0  & 1.0 \\ 
  \hline 
  \end{tabular}
  \caption{\label{Cuadro_parametros_choque_2D} Valores iniciales 
  de la presión ($p$), velocidad ($v$)
  y densidad ($\rho$), del lado interno ($p_i, v_{xi}, v_{yi}, \rho_i$) y 
  externo ($p_o, v_{xo}, v_{yo}, \rho_o$) de la onda de choque
  para los casos relativistas. Para todos los
  casos el dominio espacial será $x, y \in [0,1]\times[0,1] \, \text{cm}$, el radio de la onda
  será $r = 0.5 \, \text{cm}$ y un índice adiabático $\Gamma = 4/3$.}
  \end{center}
\end{table}

Al tiempo $t = 0$, nuestro dominio estará dividido en 2 conjuntos, a los valores ($\rho$, $v_x$, $v_y$
y $p$) que estén en la región menor al radio serán los valores internos ($\rho_i$, $v_xi$, $v_yi$ 
y $p_i$) mientras que los valores que no estén delimitados por el radio serán los valores externos
($\rho_o$, $v_xo$, $v_yo$ y $p_o$). Al ser $p_i > p_o$ y $\rho_i > \rho_o$, la onda se 
expandirá incrementando el radio, por lo que la región de adentro se volverá más grande.



La Figura \ref{Example_blast_wave} muestra los perfiles de densidad (lineas negras punteadas), que se tomarán de la onda de choque en 2D, para poder comparar con los resultados de las pruebas unidimensionales así 
como la analítica. En la Figura \ref{fig:head_map} muestra un mapa de densidades en el cual se ve la evolución de la onda. En el panel de arriba a la izquierda muestra el estado inicial al tiempo $t = 0$ s, 
el panel de arriba a la derecha muestra su estado al tiempo $t = 0.2$ s y el panel de abajo muestra su estado para el tiempo $t = 0.4$ s. 

\begin{figure}
  \centering
    \includegraphics[width=0.4\textwidth]{./Figuras/verificacion_del_codigo/pruebas_2D/Example.png}
  \caption{Las lineas punteadas indican los perfiles de densidad, velocidad y presión que se 
  tomaran de los radios que son 
  paralelos al eje x ,  perpendiculares (0°, 45°,90°) y se compararán con las pruebas en 
  1 dimensión.}\label{Example_blast_wave}
\end{figure}




\begin{figure}
  \centering
      \subfigure{\includegraphics[scale = 0.3]{./Figuras/verificacion_del_codigo/pruebas_2D/head_map/00.png}}
      \subfigure{\includegraphics[scale = 0.3]{./Figuras/verificacion_del_codigo/pruebas_2D/head_map/20.png}}
      \subfigure{\includegraphics[scale = 0.3]{./Figuras/verificacion_del_codigo/pruebas_2D/head_map/40.png}}
    \caption{Explosión de la onda de choque en 2 dimensiones usando un mapa de calor donde muestra
    la evolución de la densidad. El panel de arriba a la izquierda muestra el tiempo inicial $t=0$ s,
    el de arriba a la derecha  al $t = 0.2$ s y el de abajo al tiempo $t = 0.4$ s. 
    \label{fig:head_map}}
\end{figure}

%Ojo Julio
{\color{blue} En la gráfica \ref{fig:Expansion_radio_vs_tiempo} se muestra la expansión del radio de la explosión de la onda de choque contra el tiempo, y se ajustó una regresión exponencial para poder calcular 
el exponente con el que crece el radio, donde se muestra que crece como $R(t) \propto t^{0.426}$ y que tiene un $R^2 = 0.946$.}

%Ojo Julio
\begin{figure}
    \begin{center}
      \includegraphics[width=0.95\textwidth]{./Figuras/verificacion_del_codigo/pruebas_2D/expansion_radio/Expansion.png}
    \end{center}
    \caption{{\color{blue} Perfil del radio de la onda de choque en función del tiempo (cruz +). Además, se muestra el perfil para cuando el radio crece como $R(t) \propto t^{0.426}$ (línea roja).
    El coeficiente de determinación es $R^2 = 0.946$} .}    
    \label{fig:Expansion_radio_vs_tiempo}
\end{figure}

Para mostrar que la solución bidimensional es la misma independientemente de la dirección del perfil
radial se compararán los perfiles radiales tomados a 0°, 45°, 90°. 
%Ojo Julio
{\color{blue}En la Figura \ref{fig:comparacion_perfil_radial} se muestra como los perfiles de densidad, velocidad, y presión de la onda de choque son básicamente unidimensionales 
(esto es, que la solución es la misma si el radio de la onda de choque es horizontal, perpendicular o si forma un ángulo de 45° con respecto al eje x).} 
En el panel de arriba a la izquierda se muestran
los perfiles radiales tomados a 0°, 45°, 90°: los tres perfiles muestran una
una densidad $\rho \approx 10 \, \text{g}/\text{cm}^3$ en $x \approx 0.2$
donde comienza a disminuir hasta $\rho \approx 1.5 \, \text{g}/\text{cm}^3$ en $x \approx 0.6$. Después comienza a subir hasta 
$\rho \approx 5 \, \text{g}/\text{cm}^3$ para los perfiles de 90 y 0, mientras que para el de 45 sube hasta 5.5 en el punto
$x \approx 0.7$ cm. Después vuelve a disminuir a $\rho \approx 0 \, \text{g}/\text{cm}^3$ en el punto $x \approx 0.8$ para los 
3 perfiles radiales
Para los perfiles a 0°, 45° y 90° de la velocidad (panel arriba a la derecha), 
se observa que en $x \approx 0.2$ cm la velocidad empieza a incrementar desde el reposo
$v = 0$ hasta $v \approx 0.7c$ en $x \approx 0.6$ cm, esto aplica para los 3 perfiles de velocidad. 
Después la velocidad
va decreciendo hasta $v \approx 0.62c$ en $x \approx 0.75$ cm para los perfiles de 0 y 90, mientras que para 
el perfil de 45 lo hace en $x \approx 0.73$ cm. A partir de este punto la velocidad vuelve a cero.
El el panel de abajo muestra los perfiles a 0°, 45° y 90° de la presión. 
Para el punto $x \approx 0.18$ cm
la presión comienza a disminuir de $p = 13.33 \,  \text{dyn}/ \text{cm}^2$ 
hasta $p \approx 1 \,  \text{dyn}/ \text{cm}^2$ en $x \approx 0.7$ cm para los 
3 perfiles. En $x \approx 0.71 $ cm vuelve a disminuir a $p = 0 \,  \text{dyn}/ \text{cm}^2$ en 
$x \approx 0.74 $ cm para los perfiles de 
0 y 90, mientras que para el perfil de 45 lo hace en $x \approx 0.73 $ cm.
Al comparar las gráficas, se puede ver que los perfiles de 0 y 90, tiene un error del 0\% 
para la densidad, velocidad y presión. Mientras que
al compararse con el de 45 vemos que se tiene un error para la densidad del 0.04 \%,
para la velocidad del 4.8 \% y para la presión del 6.6 \%.
Por lo tanto al no haber una discrepencia grande en los valores de los 3 perfiles radiales podemos ver
que nuestro codigo relativista no cambia a pesar de la dirección que tengan los perfiles radiales de 
densidad, velocidad y presión.

\begin{figure}
  \centering
      \subfigure{\includegraphics[scale = 0.16]{./Figuras/verificacion_del_codigo/pruebas_2D/combinacion/densidad.png}}
      \subfigure{\includegraphics[scale = 0.16]{./Figuras/verificacion_del_codigo/pruebas_2D/combinacion/velocidad.png}}
      \subfigure{\includegraphics[scale = 0.16]{./Figuras/verificacion_del_codigo/pruebas_2D/combinacion/presion.png}}
    \caption{La figura muestra la comparación que hay entre los distintos perfiles de
    densidad al tiempo $t = 0.4$ s. El panel de arriba a la izquierda muestra el perfil a 0° (linea roja
    sólida), el de arriba a la derecha muestra a 90° (linea verde punteada) y el panel de abajo 
    (linea azul de raya y punto) muestra a 45°}
    \label{fig:comparacion_perfil_radial}
\end{figure}

En la Figura \ref{fig:comparacion_perfil_mean} se muestra la comparación entre la prueba unidimensional
y la prueba bidimensional, que es la media de los 3 perfiles radiales que se discutieron anteriormente.
El panel de arriba a la izquierda muestra la densidad, el de arriba a la derecha, la velocidad y 
el de abajo la presión. Para ambas pruebas, se mantienen sus valores de 
$\rho = 10 \,  \text{g}/ \text{cm}^3$ y $p = 13.33 \,  \text{dyn}/ \text{cm}^2 $ 
luego, bajan sus valores en $x \approx 0.25$ cm
mientras que la velocidad , la cual es nula,
asciende. En el punto $x \approx 0.68$ cm la presión alcanza 
$p \approx 2.0\,  \text{dyn}/ \text{cm}^2 $ y la densidad
$\rho \approx 2.0 \,  \text{g}/ \text{cm}^3$, mientras que la velocidad sube a $v \approx 0.75c$
en una dimensión. En 2 dimensiones la densidad baja hasta  $\rho \approx 1.0 \,  \text{g}/ \text{cm}^3$ y
la presión $p \approx 1.8\,  \text{dyn}/ \text{cm}^2 $ mientras que la velocidad en $x \approx 0.68$ cm
sube a $v \approx 0.8c$ y baja a $v \approx 0.7c$ en $x \approx 0.68$ cm.

\begin{figure}
  \centering
      \subfigure{\includegraphics[scale = 0.16]{./Figuras/verificacion_del_codigo/pruebas_2D/mean/densidad.png}}
      \subfigure{\includegraphics[scale = 0.16]{./Figuras/verificacion_del_codigo/pruebas_2D/mean/velocidad.png}}
      \subfigure{\includegraphics[scale = 0.16]{./Figuras/verificacion_del_codigo/pruebas_2D/mean/presion.png}}
    \caption{Comparación entre la media de los 3 perfiles radiales (linea sólida azul), la 
    la prueba unidimensional (linea de puntos rojos) y la analítica (linea negra sólida)
    .}\label{fig:comparacion_perfil_mean}
\end{figure}

La densidad en 1D y 2D vuelve a subir
en $x \approx 0.69$ cm y llega a $\rho \approx 5 \,  \text{g}/ \text{cm}^3$ y 
se mantiene así hasta el punto $x \approx 0.7$ cm 
donde la densidad, así como la velocidad y presión descienden sus valores a cero.
Cabe señalar que la mayor diferencia que hay entre la 1D y 2D para la densidad es del 70 \%, mientras que para la 
velocidad es del 238 \% y la de la presión es del 79 \%.
Al comparar los perfiles bidimensionales con el analítico, para la densidad se tiene un error promedio del
13 \%, para la velocidad del 12 \% y para la presión del 16 \% y errores máximos del 156 \%, 19 \% y 119 \%. 
Para el caso unidimensional con el analítico, el error medio para la $\rho$, $v$ y $p$ son 
3 \%,  7\% y  4 \% y error máximo 119 \%, 20 \% y 3 \% respectivamente.
Al no haber tanta diferencia del resultado de los perfiles radiales bidimensionales con 
los perfiles unidimensionales y con el analítico se puede concluir que el código relativista es válido
en una y dos dimensiones.



% %==============================CAPITULO JET======================================
% %============================================================================


\chapter{Jets relativistas bidimensionales}

En este capítulo se estudiará la evolución de un jet 2D relativista (usando el módulo 2D HLL que se detalla en el capítulo \ref{cap:Verificacion_del_codigo}) a través de medios con distinta densidad. La primera prueba consistirá en estudiar el caso en que el jet atraviesa un medio constante y  comparar los resultados con aquellos de Mignone \emph{et al.} (2005). 
Después, se estudiarán los efectos que un medio que varía en función de la distancia produce en los jets relativistas.
El jet se modelará como un flujo cilíndrico con densidad $\rho_j$, presión $P_j$,  y velocidad $v_j \simeq c$ 
(ver el cuadro \ref{Cuadro: propiedades-jet-comparacion} para más detalles). El índice adiabático será $\Gamma = 5/3$, 
consistente con Mignone \emph{et al.} (2005). 
Los cuales en la primera prueba serán constantes y luego variarán en función de x. El medio ambiente a su vez, 
se modelará con una densidad y presión en reposo. Un jet de grosor $\Delta y$ será inyectado 
en la frontera $x_{min}$ a la mitad de la altura $y_{max}$ (esto es, desde $\frac{y_{max}-\Delta y}{2}$ hasta 
$\frac{y_{max}+ \Delta y}{2}$). 
El resto de la frontera de $x_{min}$ así como las otras fronteras ($x_{max}$, $y_{min}$, $y_{max}$) tendrán condiciones a la
frontera de \emph{outflow}. 
El dominio espacial será $x, y \in [0,40]\times[0,20]$ con una resolución de 3200x1600 píxeles). 
Cabe señalar que el diámetro del jet será de $\delta y=80$ píxeles. El tiempo de integración total será $t = 100$, y el parámetro de Courant será $\text{Co} = 0.5$.

%====================================TABLE=================================
\begin{table}[htbp]
\begin{center}
\begin{tabular}{|c|c|c|}
\hline 
\textbf{Parámetro} & \textbf{Descripción} & \textbf{valor} \\ 
\hline 
$\rho_{j}$ &  Densidad del jet & 0.1  \\ 
\hline 
$\rho_{a}$ &  Densidad del medio ambiente & 10   \\
\hline 
$P_{j}$ & Presión interna del jet& $0.01 $ \\ 
\hline 
$P_{a}$ &  Presión del medio ambiente & $0.01 $  \\ 
\hline 
$v_{x_{j}}$ & Velocidad interna en el eje x del jet & 0.99 c \\ 
\hline 
$v_{y_{j}}$ & Velocidad interna en el eje y del jet & 0.0 c \\ 
\hline 
$v_{x_{a}}$ & Velocidad en el eje x del medio ambiente & 0.0 c \\
\hline 
$v_{y_{a}}$ & Velocidad en el eje y del medio ambiente & 0.0 c \\ 
\hline 
Co & Número de Courant & 0.5 \\ 
\hline 
\end{tabular}
\caption{\label{Cuadro: propiedades-jet-comparacion} Valores iniciales de la densidad ,
velocidad  y presión  en el jet ($\rho_j$, $v_{xj}$, $v_{yj}$, $p_j$) en unidades adimensionales y 
del medio ambiente ($\rho_a$, $v_{xa}$, $v_{ya}$, $p_{a}$). El jet será relativista, con un dominio 
$x, y \in [0,40]\times[0,20]$ y un índice adiabático $\Gamma = 5/3$.}
\end{center}
\end{table}

\section{Jet 2D RHD en un medio ambiente constante}

Para el caso en que el jet, el cual tiene una densidad $\rho_j = 0.1$, una velocidad que
solo cambia en el eje x de $v_{x_{j}} = 0.99 \, c$, una presión $p_j = 0.01$ que se mueve a través de un medio en reposo y con densidad y presión constante en el cual 
$\rho_a = 0.1 $ y $P_a = 0.01$  consistente con los valores que Mignone \emph{et al.} (2005) utilizó. 
En la Figura \ref{fig:evolucion_temporal_del_jet} se muestra la evolución del jet a 
través del medio constante en mapas de densidad
a los tiempos $t = 1$ (arriba a la izquierda), $t = 10$ (arriba a la derecha), 
$t = 20$ (abajo a la izquierda) y $t = 40$ (abajo a la derecha).
En el tiempo $t = 1$, el jet apenas es visible y solo llega a $x \approx 1$. 
Cuando $t = 10$ el longitud del jet llega  a $x \approx 6$ de longitud, las onda de colimación, se observan a 
lo longitud de y=10 y llegan a $x \approx 4$. Los  isocontornos blancos delimitan el material menos denso del jet. 
El capullo se expande a 5 unidades de longitud, mientras que de ancho alcanza 4 unidades.
Para el tiempo $t = 20$, el jet se expande a 
$x \approx 10$ unidades de longitud. El capullo mide 4 unidades de ancho a partir de parte $y > 10$, 
las ondas de colimación no cambian, mantienen su posición en $x \approx 4$ a lo longitud del eje $y = 10$, el jet se incrementa
a 7 unidades de longitud. Para el tiempo $t = 40$, el jet incrementa su tamaño a $x \approx 21$ de longitud. 
El capullo sobrepasa las 10 unidades de ancho, es decir, sale de nuestro dominio. Las ondas de colimación siguen manteniendo el mismo 
tamaño en $x \approx 4$ y el capullo sobresale de nuestro dominio. Podemos notar que el capullo está compuesto 
de materia poco densa que el jet, donde alcanza valores mínimos de $\rho \approx 0.05 $ y 
máximos de $\rho  \approx 20$ en la
parte externa del capullo. Cabe destacar que el jet conserva el mismo ancho y que la longitud del capullo crece 
en la misma proporción a la longitud del jet.

El jet del estudio de  Mignone \emph{et. al.}(2005) utiliza un jet relativista con simetría axial en 
coordenadas cilíndricas,  en este modelo, se usaron coordenadas cartesianas. 
Ellos usan una ecuación de estado ideal con $\Gamma = 5/3$, el dominio es 
$0 \leqslant r \leqslant 12$ y $0 \leqslant z \leqslant 35$ con una malla de 240 x 700. Las condiciones de
frontera son \emph{outflow} en todas su frontera excepto donde esta la eyección. El número de Courant que
usa es 0.5 y el tiempo de integración es hasta $t = 80$. Los valores del jet y su medio ambiente son los
mismos que la del Cuadro \ref{Cuadro: propiedades-jet-comparacion}. El jet que se usó en esta tesis tiene las 
misma ecuación estado que se usó en Mignone \emph{et. al.}(2005), se integró en el mismo tiempo, y aunque son distintos 
sistemas de coordenadas tiene el mismo dominio, así como las mismas condiciones de frontera. Se diferencia únicamente,
por la resolución, ya que el jet de este estudio tiene aproximadamente 3 veces más, la resolución usada en Mignone 
\emph{et. al.}(2005).

\begin{figure}
  \centering
  \includegraphics[width=1\textwidth]{./Figuras/jet/evolucion/evolucion.png}
    \caption{Mapas de densidad en los que se muestra la evolución del jet relativista 
    con los valores del Cuadro 
    \ref{Cuadro: propiedades-jet-comparacion}. El panel de arriba a la izquierda muestra al 
    tiempo t = 1, el de arriba a la derecha muestra al tiempo t = 10, 
    el de abajo a la izquierda muestra al tiempo t = 20 y el de abajo al tiempo t = 40.
    También se emplearon isocontornos (lineas punteadas) que se se componen de 5 niveles. En los 
    cuales el nivel mas bajo, que es blanco, corresponde a $\rho \approx 0.13$, blanco ligeramente grisáceo 
    para $\rho \approx 0.36$,
    gris para $\rho \approx 1$, gris oscuro para $\rho  \approx 2.7$ y el nivel mas alto, que es negro, 
    corresponde a $\rho \approx 7.3$.}
    \label{fig:evolucion_temporal_del_jet}
\end{figure}

En la Figura \ref{fig:Decaimiento_constante_densidad_jet} se muestra la comparación entre los perfiles de 
densidad a $y = 10$
del jet al tiempo t = 0, donde no se muestra una onda; a t = 5 donde la onda se localiza en $x \approx 2$; 
a t = 20, la onda se localiza a $x \approx 8$; y  t = 50, donde la onda se localiza en $x \approx 20$. 
El pánel de arriba a la izquierda muestra la densidad, en donde
podemos observar que para $x \approx 2, 8, 20$,  la densidad alcanza un valor máximo de 
$\rho  \approx 33$.
En el pánel de arriba a la derecha se muestra la velocidad, donde se puede ver que el jet mantiene una velocidad 
$v \approx 0.9 \, c$ y decae en $x \approx 17.5$ para llegar a $v \approx 0 $ en  $x \approx 21$. 
En el pánel de abajo a la
izquierda se muestra la presión, la cual se puede observar que es parecido a la gráfica de la 
densidad, ya que mantiene sus máximos en los puntos $x \approx 2, 8, 20$, pero también se observa que conforme avanza el
tiempo los picos empiezan a decrecer, ya que para $t = 5, \, p \approx 1.6 $; 
para $t = 20, \, p \approx 1.49$ y para 
$t = 50, \, p  \approx 1.3$. Para el pánel de abajo a la derecha, se hizo la misma comparación que el pánel 
de la densidad, solo que esta vez se compararón  distintas resoluciones: baja resolución "LD" (con 1600 pixeles en el eje X y 800 en el eje Y) 
y alta resolución "HD" (3200x1600), con lo que se puede observar que 
para mayor resolución el jet se vuelve ligeramente más rápido. En este panel se puede notar que
los picos están desfasados conforme avanza el tiempo. Los efectos de la resolución son despreciables ya 
que la mayor diferencia, debida a la resolución, en la magnitud de los picos es relativamente pequeña (menor a 5\%). Ésta, se presenta a t=50 y corresponde 
a $\frac{\rho_{alta} - \rho_{baja}}{\rho_{alta}} \approx 0.04$.


\begin{figure}
  \centering
  \includegraphics[width = 1.0\textwidth]{./Figuras/jet/perfiles/perfiles_constantes.png}
  \caption{Perfiles de densidad, velocidad, presión y distintas resoluciones, similar a la Figura 
  \ref{fig:comparacion_perfil_radial} a tiempos t=0, 5,20, 50.
  El perfil es de $y = 10$.
  La densidad (arriba a la izquierda), velocidad (arriba a la derecha), la presión (abajo a la izquierda). 
  En el panel de abajo a la derecha se comparan la densidad usando 2 distintas resoluciones. La de alta resolución 
  (HD)
  es de 3200x1600 pixeles y la de baja resolución (LD) es de 1600x800 pixeles.}\label{fig:Decaimiento_constante_densidad_jet}
\end{figure}

En la Figura \ref{fig:comparacion_temporal_del_jet} se muestra los resultados obtenidos en baja 
(1600x800 pixeles) y alta (3200x1600) resolución, a 
las mismas escalas de densidad, en comparación con los resultados de Mignone \emph{et. al.} (2005).
La Figura muestra de izquierda a derecha las gráficas de Mignone (izquierda), el modelo con baja resolución 
(medio) y el modelo con alta resolución (derecha). Los páneles de arriba muestran al tiempo t = 40, mientras que los 
de abajo al tiempo t = 80.
Al enfocarse en los páneles de arriba, al tiempo t = 40, se puede observar que el jet de estudio de Mignone tiene un largo 
de $x \approx 18$ unidades, mientras que el de baja resolución tiene un largo de 19 unidades y 
el jet de alta densidad  tiene un largo de 19 unidades por lo que podemos observar un error del 5 \% 
con el baja resolución y del 0 \% con el de alta resolución. El capullo esta compuesto de material de alta densidad 
que rodea al jet, su ancho se mantiene en aproximadamente 5 unidades.
Los choques de colimación son más grandes en el estudio de Mignone \emph{et. al.} (2005)
que en el de baja resolución, 
dado que se localizan en $x \approx 10$ y $x \approx 5$ respectivamente, una diferencia del 50 \%. 
Los choques de alta resolución se localizan en $x \approx 4$  lo que muestra una diferencia de mas del 
60 \% con el estudio de Mignone \emph{et. al.}(2005).
Se observa también que el capullo de baja resolución mide más de 11 unidades de ancho para $y > 10$, 
mientras que para la resolución alta mide 9 unidades, lo que nos da un error, solo para la resolución alta del 33 \%.
Tanto en el capullo de baja resolución como en el de alta resolución no se muestran turbulencias 
como en el estudio de Mignone, lo que da a entender  que el código que fue desarrollado para esta tesis 
es más viscoso.
Para los páneles de abajo, el tiempo ahora es t = 80, se puede observar que el capullo sobresale del dominio para los 
3 casos que estamos comparando. Las ondas de colimación para alta como baja resolución
estan en $x = 3$ mientras que las de Mignone se siguen manteniendo en $x = 6$, lo que nos sigue dando un error 
del 50 \%, por lo que éstas no dependen del tiempo. 
El jet para el estudio de Mignone mide $x \approx 35$ de largo, para el jet de resolución baja mide $x \approx 32$ unidades y 
para el de resolución alta $x \approx 31$ unidades, lo que nos da un error del 9 \% para resolución baja y con el 
de resolución alta un error del 11 \%.
Otra diferencia fue que el jet de 
alta resolución esta más comprimido, ya que tiene un grosor aproximado de 1.0 unidades 
mientras que el de baja resolución tiene una resolución aproximada de 1.4 unidades, 
aunque no está tan comprimido como el jet mostrado en el estudio de Mignone \emph{et. al.}(2005). así también, no podemos encontrar, tanto para baja como 
alta resolución las turbulencias generadas en el jet de Mignone, pero también no se pueden comparar las densidades del 
mismo debido a que no presenta los valores.



\begin{figure}
  \centering
  \includegraphics[width=1\textwidth]{./Figuras/jet/comparacion/multiple_comparation.png}
    \caption{Mapas de densidad en los que se compara el jet relativista con los valores del Cuadro 
    \ref{Cuadro: propiedades-jet-comparacion}. La Figura está dividida en 3 columnas. La de la izquierda muestra
    el estudio de Mignone, la de en medio el jet a baja resolución (1600x800 pixeles), y la de la derecha el jet a alta 
    resolución (3200x1600 pixeles).
    }\label{fig:comparacion_temporal_del_jet}
\end{figure}



\section{Jet 2D RHD en un medio ambiente variable}




Esta sección estudia los efectos que un medio que varía en función de la distancia, $\rho = \rho(x)$, 
produce en un jet. La densidad del medio ambiente variará en función del eje, es decir, se modelará como:

\begin{equation}
  \rho(x) \varpropto \frac{1}{x} \,\,\,\,\,\,\,\,\, \text{y} \,\,\,\,\,\,\,\,\, \rho(x) \varpropto \frac{1}{x^2}
\end{equation}

Donde $\rho_a$ es el valor mostrado en el Cuadro \ref{Cuadro: propiedades-jet-comparacion}.



La Figura \ref{fig:Decaimiento_lineal_densidad_jet} muestra los perfiles de densidad, velocidad y presión, es decir, 
solo mostrará los valores que estén sobre $y = 10$ {para el caso cuando $\rho \varpropto \frac{1}{x}$}. En el panel de arriba 
a la izquierda  muestra la densidad. El de arriba a la derecha, la velocidad. El de abajo a la izquierda la presión y 
el de abajo a la derecha muestra la comparación entre
uno de mayor resolución con uno de baja resolución. Al tiempo t = 0, al no haber una inyección del jet, únicamente la densidad 
decae como $\frac{1}{x}$. Al tiempo t = 5, la densidad, en x = 3 tiene un máximo local que llega a 
$\rho  \approx 33$, en t = 20, el pico se situa en $x \approx 12$ con un máximo de 
$\rho  \approx 16.5$ y para el 
tiempo t = 50, el pico se situa en $x \approx 33$ con $\rho  \approx 3.6$ por lo que los máximos disminuyen en función de la 
densidad del medio ambiente.
La velocidad tiene un máximo en $x \approx 0.2$, donde alcanza  $v \approx 0.6$. Para los tiempos t = 20, 50 ya no 
presenta máximos sino que decae rápidamente en las posiciones $x \approx 8, \, 25$ respectivamente.
La presión tiene un valor máximo al tiempo t = 5 con $p \approx 1.64$, aunque si un cambio menos abrupto en $x = 3$. 
En los tiempos t = 20, 50 los máximos globales se sitúan en $x \approx 12$ y en $x \approx 33$ 
con $p \approx  1$ y $p  \approx 0.6$ respectivamente.
En el panel inferior derecho, al comparar la densidad en resoluciones de 3200x1600 y 1600x800,
notamos que a diferencia del medio constante, no hay un desfase tan grande.  Los efectos de la resolución en este caso 
también son pequeños. Lo anterior se debe a que la mayor diferencia, debida a la resolución, 
en la magnitud de los picos es menor a 10\% ($\frac{\rho_{alta} - \rho_{baja}}{\rho_{alta}} \approx 0.04$ a t=50). 

\begin{figure}
  \centering
  \includegraphics[width = 1.0\textwidth]{./Figuras/jet/perfiles/perfiles_lineales.png}
  \caption{Perfiles de densidad, velocidad, presión y distintas resoluciones, similar a la Figura 
  \ref{fig:comparacion_perfil_radial} para un perfil de densidad que varia en función de la distancia como
  $\rho \varpropto \frac{1}{x}$ a tiempos t=0, 5,20, 50.
  El perfil es de $y = 10$.
  La densidad (arriba a la izquierda), velocidad (arriba a la derecha), la presión (abajo a la izquierda). 
  En el panel de abajo a la derecha se comparan la densidad usando 2 distintas resoluciones. La de alta resolución
  es de 3200x1600 y la de baja resolución es de 1600x800.}\label{fig:Decaimiento_lineal_densidad_jet}
\end{figure}


En la Figura \ref{fig:Decaimiento_cuadratico_densidad_jet} se muestra la densidad de medio ambiente 
que decae como $\rho \varpropto \frac{1}{x^2}$. Aquí podemos ver que los  picos de la densidad estan en
x = 3, 10, 33 con  la densidad $\rho \approx 54, \,  10, \,  7.3$ 
respectivamente. En la velocidad podemos observar
los frente de choque de la onda que se ubican en $x \approx 3, \, 13, \, 37$ con los valores de velocidad
$v \approx 0.42, \, 0.58, \, 0.60$ respectivamente. Para más detalles véase el Cuadro \ref{Cuadro:valores_frente_choque}.
En el caso de la presión, al tiempo t = 5, desciende rápidamente en $x \approx 0.5$ de 
$p  \approx 10$ a $p  \approx 2.7$.
En el tiempo t = 20, la presión tiene un valor máximo en $x \approx 10$, 
donde mantien un valor  de 
$p \approx e^2 \approx 7.3$. Para el tiempo t = 50
la presión desciende exponencialmente del punto $x \approx 28$ donde $p \approx e^{-5} \approx 0.006$.
Al comparar las resoluciones de 1600x800 con la de 800x400 se vuelve a ver un desfase.Los efectos de la resolución 
en este caso dejan deben ser tomado en cuenta. Lo anterior se debe a que la mayor diferencia, debida a la resolución, 
en la magnitud de los picos es mayor al 60\% ($\frac{\rho_{alta} - \rho_{baja}}{\rho_{alta}} \approx 0.63$ a t=50). 
Por ello, si se deseará estudiar un jet relativista a través de un medio que disminuye como $1/x^2$, o $1/r^2$ es 
necesario hacer estudios de convergencia para determinar la resolución necesaria para que no influya en los resultados. 

\begin{figure}
  \centering
  \includegraphics[width = 1.0\textwidth]{./Figuras/jet/perfiles/perfiles_cuadraticos.png}
  \caption{Perfiles de densidad, velocidad, presión y distintas resoluciones, similar a la Figura 
  \ref{fig:comparacion_perfil_radial} para un perfil de densidad que varia en función de la distancia como
  $\rho \varpropto \frac{1}{x^2}$ a tiempos t = 0, 5,20, 50.
  El perfil es de $y = 10$.
  La densidad (arriba a la izquierda), velocidad (arriba a la derecha), la presión (abajo a la izquierda). 
  En el panel de abajo a la derecha se comparan la densidad usando 2 distintas resoluciones. La de alta resolución
  es de 3200x1600 y la de baja resolución es de 1600x800.}\label{fig:Decaimiento_cuadratico_densidad_jet}
\end{figure}

En resumen, los picos de la densidad y presión 
son en general más pequeños que los mostrados en donde la densidad del medio es constante, pero también recorren mayor 
distancia que los anteriormente mencionados. En cambio, para el caso en que la densidad decae como $\frac{1}{x^2}$, los picos son mayores pero recorren menos distancia que el medio anteriormente mencionado.


\begin{table}[htbp]
  \begin{center}
  \begin{tabular}{|c|c|c|c|c|c|c|c|c|c|c|c|c|}
    \hline 
    & \multicolumn{4}{|c|}{$\rho = \rho_a$} & \multicolumn{4}{|c|}{$\rho_a \propto x^{-1}$} & \multicolumn{4}{|c|}{$\rho_a \propto x^{-2}$}\\
    \hline
    t & $x_{\text{fch}}$ & $\rho_{\text{fch}}$ & $p_{\text{fch}}$ & $v_{\text{fch}}$ & $x_{\text{fch}}$ & 
    $\rho_{\text{fch}}$ & $p_{\text{fch}}$ & $v_{\text{fch}}$ & $x_{\text{fch}}$ & $\rho_{\text{fch}}$ & $p_{\text{fch}}$ & 
    $v_{\text{fch}}$ \\
    \hline
     5 & 3  & 3.5 & 0.6 & 0.38    &  3  & 3.5 &  0.5  & 0.4     & 3  & 4   & 0.8  & 0.42 \\
    \hline
    20 & 9  & 3.5 & 0.6 & 0.4    &  12 & 2.6 &   0   & 0.55    & 13 & 2   & -1.8 & 0.5 \\
    \hline
    50 & 21 & 3.5 & 0.5 & 0.6    &  34 & 1.1 & -0.5  & 0.64     & 37 & 1.8 & -2.0 & 0.64 \\
    \hline

  \end{tabular}
  \caption{\label{Cuadro:valores_frente_choque} Valores de densidad presión y velocidad que se muestran en el frente
  de choque de la onda del jet. Los valores obtenidos fueron tomados de las Figura \ref{fig:Decaimiento_constante_densidad_jet},
  \ref{fig:Decaimiento_lineal_densidad_jet} y \ref{fig:Decaimiento_cuadratico_densidad_jet}}
  \end{center}
  \end{table}


Conforme disminuye más rápidamente en función de la distancia \emph{x}, 
se vuelve más rapido.
En la Figura \ref{fig:perfiles_comparacion_jet} se muestra la comparación que hay cuando la densidad del medio
ambiente cambia constantemente ($\rho \varpropto \rho_a$), cuando varia inversamente lineal ($\rho \varpropto x^{-1}$) y 
cuando varia inversamente cuadrático ($\rho \varpropto x^{-2}$) al tiempo $t = 50$. Las lineas
punteadas muestran los exponentes intermedios entre los valores de 0, 1 y 2 en pasos de 0.2, es decir la densidad
del medio ambiente descenderá como $x^0, \, x^{-0.2}, \, . . . \,  ,x^{-0.8} , \, x^{-1}  , \, x^{-1.2}
, \, . . . \,  , x^{-2}$.
El primer valor máximo que acentua más en el decaimiento inversamente cuadrático que se localiza en $x \approx 2$ y tiene
un valor $\rho \approx 0.6$. Mientras para el decaimiento inversamente lineal se localiza en $x \approx 3$ y su valor es 
$\rho  \approx  0.3$. Para el caso constante, no remarca un máximo en especial. El segundo valor máximo, 
para el caso 
constante se localiza en $x \approx 20$ y alcanza $\rho  \approx 33$, conforme se aumente el exponente el valor 
máximo se desplaza, así como su disminución de la densidad ya que para el caso inversamente lineal el máximo se localiza 
en $x \approx 32$ con $\rho \approx 1$ y para el caso inversamente cuadrático el valor máximo se localizó en 
$x \approx 34$ con $\rho \approx 0.36 $.  Podemos observar que el primer valor máximo de la gráfica, el medio
que varía com $x^{-2}$ tiene un valor mayor que el del medio constante, mientras que para el segundo valor máximo se invierte
y el medio constante tiene un valor mayor que el del medio que varía como $x^{-2}$.
La velocidad de la onda aumenta un 60 \% para medio inversamente lineal en correspondencia con el medio constante y un 6 \%
para el medio inversamente cuadratico en correspondencia con el inversamente lineal.  



\begin{figure}
  \centering
    \includegraphics[width=1\textwidth]{./Figuras/jet/perfiles/densidades_comparacion.png}
  \caption{Gráfica de comparación de la posición vs el logaritmo de la densidad 
  donde el medio ambiente es constante $\rho \varpropto k$ (verde), varía inversamente lineal
  $\rho \varpropto x^{-1}$ (amarillo) y varía inversamente cuadrático $\rho \varpropto x^{-2}$ (rojo)
  al tiempo t = 50.  Las lineas punteadas muestran los valores exponenciales entre 0 y 1 y entre 1 y 2,
  es decir, que la densidad descenderá como $x^0, \, x^{-0.2}, \, . . . \,  ,x^{-0.8} , \, x^{-1}  , \, x^{-1.2}
  , \, . . . \,  , x^{-2}$.}\label{fig:perfiles_comparacion_jet}
\end{figure}




\chapter{Conclusiones}

El objetivo principal de esta tesis fue construir un código que fuera tanto newtoniano como relativista utilizando distintos métodos numéricos de Riemann, realizar pruebas numéricas para poder verificar que tanto el código relativista como el newtoniano funcionan correctamente. Finalmente se reprodujo el jet bidimensional relativista en un medio constante de Mignone \emph{et. al.} 2007 y se estudió 
como un medio cuya densidad disminuye en función de la distancia ($\rho \propto R^{-1}$ y $\rho \propto R^{-2}$) para observar como se afecta la evolución del jet.

Dado que los jets astrofísicos, cuya evolución es descrita por la hidrodinámica pueden ser eventos muy violentos en los cuales se alcanzan velocidades cercanas a la de la luz, 
las ecuaciones de la hidrodinámica 
deberán tomar en cuenta los efectos de la relatividad especial. Debido a lo anterior, el código numérico que se construyó con un módulo newtoniano y uno relativista 
(ambos en una y dos dimensiones).
{\color{blue} Para poder resolver estas ecuaciones se usaron 2 métodos numéricos, los cuales son el método de Lax y el de HLL. El primero es un método numérico para resolver ecuaciones diferenciales parciales
usando diferencias finitas mientras que el método de HLL es parecido pero teóricamente mejor.}

Las primeras pruebas que se realizaron consistieron en pruebas unidimensionales en el régimen newtoniano. Estas consistieron en resolver varias configuraciones del tubo de Sod, esto es, 
analizar la evolución de dos fluidos con distinta densidad, velocidad y presión que inicialmente estaban separados en una posición determinada. Cabe señalar que en estas pruebas se analizaron las diferencias 
que se tienen al utilizar el método de Lax o el de HLL. Para el caso donde se tienen distintas densidades (y las presiones y las velocidades son iguales) tanto el método de Lax como el método de HLL son indistintos 
y se reproduce la solución analítica. En este caso el método de Lax resulto ser computacionalmente más rápido que el de HLL. Para el caso en donde ahora las velocidades son distintas y las densidades y presiones son 
iguales también se reprodujo la solución analítica indistintamente del método de Riemann que se utilizara. El método de Lax resultó ser notoriamente más rápido que HLL por lo que para el régimen unidimensional y para 
velocidades newtonianas, es recomendable usar este método.


%Ojo Julio
Las siguientes pruebas consistieron en verificar que en los tubos de Sod, en donde los fluidos evolucionaran con velocidades cercanas a la de la luz, se reprodujeran correctamente con el 
módulo relativista. 
{\color{blue}
Para ambos métodos se puede observar que se tienen errores cuando se aproximan a los valores discontinuos, principalmente en las discontinuidades de contacto (superficies que separan 
zonas de diferente densidad y temperatura). Mientras que para las discontinuidades de choque (ocurren cuando los gradientes de presión son lo suficientemente grandes como para generar movimientos de compresión supersónicos),  %INTERSTELLARSHOCK WAVES Christopher F. McKee 
se muestra una mayor aproximación a la solución analítica. Y para las zonas donde no se presentan discontinuidades, ambos métodos se aproximaban bastante para resoluciones bajas.} Para los casos, donde el régimen es relativista 
pero se mantiene la misma dimensión, el método de Lax falla 
por completo, sin importar la resolución que se establezca, donde se encontraron errores graves, por lo que no se pudo hacer una comparación entre ambos métodos. A diferencia del caso newtoniano, 
el relativista no presenta dificultades en las discontinuidades de contacto y tiene una buena aproximación al caso analítico de onda donde las densidades son distintas. Para el caso donde las 
velocidades son distintas, la aproximación de HLL para el caso analítico, fue casi igual para resoluciones altas y bajas por lo que  HLL para el régimen unidimensional relativista  
hace un correcto modelaje de las ecuación  es de la hidrodinámica relativista. En resumen, para efectos de costo computacional, el método de Lax se recomienda usar ya que es ligeramente mas rápido 
que HLL, pero en el régimen relativista HLL es el único que muestra resultados viables.

%Ojo Julio
{\color{blue} ACA MENIOCNARIA LAS PRUEBAS BIDIMENCIONALES NEWTONIANAS} {\color{red} No se hicieron pruebas bidimensionales newtonianas}

%Ojo Julio
{\color{blue} LUEGO MENIOCNARIA LAS PRUEBAS BIDIMENCIONALES RELATIVISTAS}

En las pruebas bidimensionales, dado que el método de HLL fue el único que paso las pruebas del tubo de Sod, {\color{red} se descartó el método de Lax y el régimen newtoniano}, ya que lo que importa es que el código pueda 
reproducir flujos relativistas. Al comparar los perfiles de densidad y 
comparándose con los unidimensionales los valores de la densidad vienen teniendo el mismo resultado. Por lo que mi código no tiene una preferencia de dirección.

Cuando se comparón los perfiles de onda de choque bidimensional con la parte unidimensional y el analítico se mostró que  el perfil en 2 dimensiones es en promedio mucho mas cercano al valor analítico, 
sobre todo en la velocidad donde se observa que salvo en las discontinuidades de choque y de contacto los valores son los mismos que el el analítico, mientras que en la parte unidimensional mantiene valores mas alejados al analítico. 
Por lo tanto el código en 2 dimensiones es mejor que cuando lo presentamos en el caso unidimensional y por lo tanto el código se puede usar para hacer simulaciones de jet que se mueven a  velocidades relativistas.

% Diego revisó hasta acá

Al compararse el jet de este código, con el del estudio de Mignone \emph{et al} 2007 se encontraron similitudes, así como algunas diferencias.  La principal es que no presenta turbulencias, es decir, nuestro jet es 
mas viscoso y no se puede mejorar sin importar la resolución que se le aplique. Pero si comparte la morfología tales como la longitud del jet y las ondas de colimación así como a parte cilíndrica del jet, 
sin embargo no se comparte el ancho o tiene una diferencia mayor debido a que, como ya se había mencionado antes, el jet de esta tesis es mas viscoso. Sobre este punto, que es la viscosidad, se puede observar que esta 
se reduce conforme se aumente la resolución, sin embargo, el jet de estudio de Mignone no necesita mucha resolución, por lo que el que la viscosidad del jet de este estudio no es un problema de resolución sino mas 
bien usar un algoritmo mas robusto para resolver las ecuación  es de la hidrodinámica  tales como HLLC que es otro método numérico . Aunque el jet del estudio Mignone \emph{et al} 2007 detalla todas las 
características de su jet, no podemos compararlo perfectamente debido a la falta de mención de los valores de la densidad.

Al simular el jet se encontró que contiene las características esenciales de todo jet, como lo es, el capullo y las ondas de colimación. Las gráficas muestran que se mueve a velocidades relativistas y que no presenta 
errores al llegar a tales velocidades. Al comparar las resoluciones, se puede notar que entre mas resolución se use el jet, es ligeramente mas lento.

Dado las comparaciones que se hicieron con en pruebas unidimensionales, bidimensionales y con el estudio de Mignone \emph{et al} (2017) se implementó este código para pruebas en medios ambiente que decaen en función de la 
distancia. Al estudiar el jet se pasa de  un medio, que es homogéneo, a no que decae como $1/x$ o $1/x^2$, se encontró, como se esperaba, que los picos de densidad y presión decaían en proporción a como lo hacia el 
medio ambiente. En los paneles de velocidad se podría observar que el jet alcanzaba velocidades relativista y que también en los paneles de densidad, al compararse los perfiles se pudo observar que conforme el medio 
ambiente decae mas rápido el jet también se mueve más rápido, esto debido a que el material que barre es menor  y que es algo que se esperaría.

A pesar de que el jet fue probado en distintas situaciones, se puede mejorar bastante y mejorar considerablemente el tiempo de cómputo ya que el código gasta muchos recursos, sobre todo en cuestión de tiempo. 
Las mejoras son remarcadas en los siguientes puntos:

\begin{itemize}
\item \textbf{Implementación de otro solucionador de Riemann}: A pesar de que HLL, es un algoritmo robusto y fácil de implementar promedia la solución completa al problema de Riemann en un solo estado y, por lo tanto, carece de la capacidad para resolver ondas intermedias únicas tales como las ondas de contacto. Por lo que una buena mejora sería usar el método de HLLC que restaura la estructura de onda completa dentro de la gráfica de ondas de Riemann.

\item \textbf{Implementación del algoritmo AMR}: Como ya se vio anteriormente las áreas de mayor atención son las zonas de ondas  de contacto y para poder reproducirlas lo mejor posible se necesita una mejor resolución. Sin embargo el costo computacional es alto y como no toda la región es de nuestro interés, la implementación de este algoritmo ayudaría a tener una mejor aproximación de estas zonas a un costo computacional bajo sin perder toda la información de nuestro jet.

\item \textbf{Paralelismo}: Algo que realmente podría hacer mas  rápido el código es hacerlo paralelo, es decir, que se puedan hacer varios procesos al mismo tiempo y no, necesariamente hacerlos secuencialmente y poder adaptarlo a la malla que generamos. Si pudiéramos hacer esto podríamos hacer uso de GPUs haciendo uso de CUDA Fortran o de algún otro compilador que pudiera trabajar GPU programando en Fortran.

\item \textbf{Paradigma}: Si bien Fortran es una gran herramienta para el computo científico, es difícil darle mantenimiento al mismo programa ya que tiene un paradigma imperativo por lo que usar un paradigma orientado a objetos tal como en Julia podría facilitar el mejoramiento del código implementando las anteriores características ya mencionadas.

% \item Tanto en Lax, como en HLL para el caso newtoniano muestran dos tipos de onda. Una se va alejando del centro de la misma (\emph{forward shock}) y una onda que se va acercando a la misma (\emph{reverse shock}). Estas se muestran independientemente si se mueve la onda de choque hacia las fronteras. (Sección \ref{sec:onda de choque})

% \item En el caso newtoniano, la velocidad de expansión de la onda fue un 1.3\% más veloz que el método de HLL. Pero HLL muestra una discontinuidad menós viscosa (ver Figura \ref{fig:Lax-hll-newtoniano1}) con una diferencia de densidades menores al $<0.5 \% $ lo que hace 
% a HLL un mejor resolvedor de Riemann para mí código. Aún así se pueden usar ambos métodos para el choque de onda. (Sección\ref{subs:Lax_vs_HLl})

% \item Para el caso relativista la expansión de la onda de choque tiene un descenso rápido de velocidad llegando al 19 \% de la velocidad de la luz al tiempo $< 0.7$ s sin importar la presión o energía que contenga. (Sección \ref{subs:Lax_vs_HLL_relativista}).

% \item Una diferencia entre las ondas de choque newtonianas y relativistas es que no muestran una discontinuidad, aunque siguen presentando las ondas \emph{forward} y \emph{reverse shock} como en el caso newtoniano. (Sección \ref{subs:Lax_vs_HLL_relativista}).

% \item Para el caso del jet relativista, el capullo es 4 veces mas denso usando Lax que HLL. Otra diferencia con los jet relativistas es que HLL muestra una eyección 4 veces más densa que las condiciones que se habian impuesto, mientras que con Lax se muestrasn las condiciones dadas. Por tal motivos Lax es un mejor método para mi programa y HLL es un método ineficiente. (Sección \ref{sec:Diferencia_de_los_metodos_numéricos_sobre_el_jet}).

% \item El jet en condiciones de densidad mantiene su velocidad cercana a la de la velocidad de la luz. A pesar de que el jet tuvo únicamente velocidad sobre el eje Y, este no es totalmente colimado y mantiene un ángulo de apertura de $\sim 10^o$. (Sección \ref{sec:SGRB_en_medio_de_densidad_bajo}).

% \item El jet en un medio de densidad alto al chocar con un medio de densidad 10 mas denso que el mismo ya no muestra velocidades relativistas dado que su velocidad disminuye a velocidades newtonianas. (Sección \ref{sec:SGRB_en_medio_de_densidad_alto}).

\end{itemize}


\appendix
\chapter{Código}\label{aped.A}

El programa está escrito en lenguaje FORTRAN se compone de un módulo principal el cual está compuesto de un programa principal y este a su vez llamará a varias subrutinas:
\begin{itemize}
\item \textbf{initconds}: Esta subrutina calculará los valores iniciales que le demos al programa

\item \textbf{output}: Devuelve un archivo con los datos que se calculan con el método de Lax

\item \textbf{Courant}: Calcula el paso temporal

\item \textbf{ulax}: Calcula el paso siguiente de las variables conservadas

\item \textbf{boundaries}: En esta parte puedes definir las fronteras a utilizar como outflow o las condiciones para las del jet

\item \textbf{fluxes}: Calculo de los flujos
\end{itemize}

Al usar las ecuacuaciones hidrodinámica relativistas, se agregan 2 subrutinas más:

\begin{itemize}
\item \textbf{uprim}: Este módulo es agregado para poder desacoplar las variables conservadas

\item \textbf{newraph}: Calcula el método de Newton-Rapson será de gran utilidad en el desacoplamineto de las variables conservadas y así obtener nuestras primitivas
\end{itemize}

\begin{lstlisting}[frame=single] 
do i=0,nx+1
  do j=0,ny+1
   
   x=float(i)*dx 	! obtain the position x_i
   y=float(j)*dy 	! obtain the position y_j
   rad=sqrt((x-xc)**2+(y-yc)**2)
   
   if (rad < 0.1) then
   
     lorin=1/sqrt(1-(vxin**2+vyin**2))
     hin=1.+gamma/(gamma-1.)*pin/rhoin
           
     u(1,i,j)=rhoin*lorin
     u(2,i,j)=rhoin*vxin*lorin**2*hin
     u(3,i,j)=rhoin*vyin*lorin**2*hin
     u(4,i,j)=rhoin*lorin**2*hin-pin
    
    else
    
     lorout=1./sqrt(1.-(vxout**2+vyout**2))
     hout=1.+gamma/(gamma-1.)*pout/rhoout
     
     u(1,i,j)=rhoout*lorout
     u(2,i,j)=rhoout*vxout*lorout**2*hout
     u(3,i,j)=rhoout*vyout*lorout**2*hout
     u(4,i,j)=rhoout*lorout**2*hout-pout
     

\end{lstlisting}
y para los fluidos en la subrutina de fluxes
\begin{lstlisting}[frame=single]
          f(1,i,j)=rho*vx*lor
          f(2,i,j)=rho*vx*vx*lor**2*h+P
          f(3,i,j)=rho*vx*vy*lor**2*h
          f(4,i,j)=rho*vx*lor**2*h

          g(1,i,j)=rho*vy*lor
          g(2,i,j)=rho*vx*vy*lor**2*h
          g(3,i,j)=rho*vy*vy*lor**2*h+P
          g(4,i,j)=rho*vy*lor**2*h
\end{lstlisting}
Como el código es una iteración, solo la primera vez que itere estaremos bien, pero, al siguiente bucle saldrá mal debido a que nuestros resultados nos están arrojando en principio las variables conservadas, y lo que se requiere es obtener las primitivas.

\section{Condición inicial}
En la subrutína \textit{initconds} se calcularán las condiciones iniciales, tomando los valores de los parámetros del módulo de \textit{globals}, que en este caso son: la densidad $(\rho)$, las velocidades tanto en $x$ como en $y$ $(v_x, v_y)$, la presión $(p)$ y $\Gamma$. Con estas constantes dadas se calcularán nuestras variables conservadas.

\begin{lstlisting}[frame=single] 
!==============================================================================
! In this module we set the initial condition
!------------------------------------------------------------------------------
      subroutine initconds(time,tprint,itprint)
      use globals
      implicit none
      real, intent(out) :: time, tprint
      integer, intent (out) :: itprint
      integer ::i,j
      real :: x,y, rad

!------------------------------------------------------------------------------
! For the 2D circular blast:
! u(1,i,j) = rho(i,j)
! u(2,i,j) = vx(i,j)
! u(3,i,j) = vy(i,j)
! u(4,i,j) = etot(i,j) = eint + ekin = P/(gamma-1)
!------------------------------------------------------------------------------
      do i=0,nx+1
        do j=0,ny+1
          x=float(i)*dx          ! obtain the position $x_i$
          y=float(j)*dy          ! obtain the position $y_j$
          rad=sqrt((x-xc)**2+(y-yc)**2)

          if (rad < 0.3) then
            u(1,i,j)=rhoin
            u(2,i,j)=rhoin*vxin
            u(3,i,j)=rhoin*vyin
            u(4,i,j)=pin/(gamma-1.)+0.5*u(2,i,j)*u(2,i,j)/u(1,i,j) + 0.5/u(1,i,j)*u(3,i,j)*u(3,i,j)
          else
            u(1,i,j)=rhoout
            u(2,i,j)=rhoout*vxout
            u(3,i,j)=rhoout*vyout
            u(4,i,j)=pout/(gamma-1.) + 0.5/u(1,i,j)*u(2,i,j)*u(2,i,j) + 0.5/u(1,i,j)*u(3,i,j)*u(3,i,j)

          end if

        end do
      end do

!------------------------------------------------------------------------------
! end of the 2D circular blast initial condition
! reset the counters and time to 0
!------------------------------------------------------------------------------
      time=0
      tprint=0
      itprint=0

      return
      end subroutine initconds
!------------------------------------------------------------------------------
! end of the init condition module
!==============================================================================

\end{lstlisting}
En esta parte dan los valores iniciales para nuestra malla tanto en $x$ como en $y$ en el tiempo $t=0$

\subsection{Condición de Courant}
Esta parte del código tiene que ver con los incrementos $\Delta t$, los cuales se van a calcular en este módulo, para poder calcularlos tenemos que tener en cuenta la convergencia y la estabilidad de nuestras ecuación  es diferenciales parciales (ecuación \ref{u_posterior_tensor}). La condición de convergencia establece que la solución de la ecuación numérica se aproxima a la solución con ecuación diferencial parcial original si todos los intervalos finitos tienden a cero, una condición necesaria para la convergencia es que los errores, por ejemplo los debidos al redondeo, no se incrementen con en tiempo. Esta es la llamada la condición de estabilidad. Es una condición tan importante que implica ciertas restricciones al tamaño del paso de tiempo en un proceso explícito. Un análisis de estabilidad para esquemas explícitos a partir de la teoría de las características para soluciones continuas lleva a la conclusión que dichos esquemas, para ser estables, deben cumplir la condición de Courant, que es: 

\begin{equation}
\Delta t \leq \frac{\Delta x}{u+C}
\end{equation}

Donde $C$ es el número de Courant y nos limita a que nuestros $\Delta t$ no sean tan grandes
\begin{lstlisting}[frame=single]
!==============================================================================
! CFL criterium module
!------------------------------------------------------------------------------
      subroutine courant(dt)
      use globals
      implicit none
      real, intent(out) ::dt
      real :: rho, vx, vy, P, cs
      integer :: i,j

!------------------------------------------------------------------------------
! Calculate the CFL criterium
!------------------------------------------------------------------------------
      dt=1E30
      do i=0,nx+1
        do j=0,ny+1
          rho=u(1,i,j)
          vx=u(2,i,j)/rho
          vy=u(3,i,j)/rho
          P=(u(4,i,j)-0.5*rho*(vx**2+vy**2))*(gamma-1.)
          cs=sqrt(gamma*P/rho) !Speed of sound
          dt=min( dt,Co*dx/(abs(vx)+cs) )
          dt=min( dt,Co*dy/(abs(vy)+cs) )

        end do
      end do

      return
      end subroutine courant

\end{lstlisting}



\chapter{Condiciones de frontera} \label{aped.B}


Las condiciones de frontera se usará para obtener los valores de nuestras variables conservadas en los extremos de nuestra malla de puntos, con el fin de evitar errores numéricos, las condiciones de frontera que generalmente se usan son de cuatro tipos las de \textit{outflow}, las de \textit{reflexión}, las \textit{periódicas} y las de \textit{jet}. Las del tipo \textit{outflow} serán aquellas en las que una vez los valores sobre la malla (ondas) queden fuera de esta, ya no sabremos que pasó después con estos datos, las de \textit{reflexión} serán aquellas en las que nuestros datos en vez de salir se reflejarán y las \textit{periódicas} serán parecidas a las de \textit{reflexión} solo que en vez de reflejarse las ondas, estas entrarán del lado contrario de donde salieron  y las de tipo \textit{jet}, será para que de un lado de nuestra malla salga una fuente de partículas.

Para entender lo que son las condiciones de frontera, vamos suponer una malla de puntos, esta malla tendrá $n+2$ filas y $m+2$ columnas, para identificar los puntos vamos a indexarlos empezando desde el 0 hasta $n+1$ en el caso de las filas y de 0 hasta $m+1$ para el de las columnas.

\begin{figure}[H]
\centering
\includegraphics[width=0.5\textwidth]{./Figuras/malla.png}
\caption{Malla de puntos que resalta las fronteras, los puntos más oscuros representan representan una frontera \emph{fantasma}, es decir, ese conjunto de puntos estará allí como apoyo para resolver los cálculos que hará la computadora dado que tanto el método de Lax, como el método de HLL usan los puntos posteriores y anteriores en el espacio del punto que queremos saber su valor, pero no se graficarán} \label{fig: malla de puntos}
\end{figure}

La Figura \ref{fig: malla de puntos} muestra 
los puntos más negros como unos puntos que nos van a 
servir de apoyo para calcular los valores de los puntos más grises, esto debido a que para calcular 
los valores de algún punto $(x_i, y_j)$ necesitamos el punto posterior $x_{i+1}, y_{j+1}$ y el punto 
anterior $x_{i-1}, y_{j-1}$.

Ahora, cuando llegamos a los últimos puntos de nuestro frontera, 
por ejemplo, el punto $x_{0}, y_{0}$, no podremos calcularlo debido a que  no tendremos conocimiento acerca del punto anterior $x_{(-1)}, y_{(-1)}$, entonces  tendremos que darles valores específicos a estos puntos, pero no podemos darles cualquier valor, en las siguientes secciones vamos a ver que valores válidos le podemos dar para el funcionamiento del código. 

\subsection{Condiciones de frontera \emph{outflow}}

Las condiciones \emph{outflow}, son en las que los valores de nuestra frontera fantasma(puntos negros) que toman los  mismos valores que su antecesor (puntos grises), los fenómenos físicos que atraviesan nuestra frontera, pasarán como si tuviéramos más dominio hacia afuera y una vez que salgan perderemos información sobre esta.

\begin{figure}[H]
\centering
\subfigure[onda expandiéndose antes de cruzar la frontera]{\includegraphics[scale = 0.50]{./Figuras/Apendice/outflow1}}
\subfigure[onda expandiéndose después de cruzar la frontera]{\includegraphics[scale = 0.50]{./Figuras/Apendice/outflow2}}
\caption{Al pasar la onda nuestra frontera, esta sigue su trayecto normal como si el dominio fuera infinito} \label{fig:reflexion}
\end{figure}

Las ecuación  es que obedece nuestra frontera son las siguientes:
\begin{eqnarray}
\textbf{U}(0,j)&=&\textbf{U}(1,j) \\
\textbf{U}(n+1,j)&=&\textbf{U}(n,j) \\
\textbf{U}(i,0)&=&\textbf{U}(i,1) \\
\textbf{U}(i,m+1)&=&\textbf{U}(i,m) 
\end{eqnarray}

Donde $i,j \leq n,m \in \mathbb{N}$

\subsection{Condiciones de frontera \emph{reflexión}}

Las condiciones de reflexión funcionan como una pared en la que no se le permite al fenómeno físico escapar y al toparse con estas fronteras se reflejarán. Los valores que tendrán los puntos negros, serán los valores negativos de los puntos grises.

\begin{figure}[H]
\centering
\subfigure[onda expandiéndose antes de llegar a la frontera]{\includegraphics[scale = 0.50]{./Figuras/Apendice/reflexion1}}
\subfigure[onda reflejándose en las fronteras]{\includegraphics[scale = 0.50]{./Figuras/Apendice/reflexion2}}
\caption{Al llegar la frontera la onda se reflejará y chocará con la misma} \label{fig:reflexion}
\end{figure}

Las ecuación  es que obedece nuestra frontera son las siguientes:
\begin{eqnarray}
\textbf{U}(0,j)&=&-\textbf{U}(1,j) \\
\textbf{U}(n+1,j)&=&-\textbf{U}(n,j) \\
\textbf{U}(i,0)&=&-\textbf{U}(i,1) \\
\textbf{U}(i,m+1)&=&-\textbf{U}(i,m) 
\end{eqnarray}


\subsection{Condiciones de frontera \emph{Periódicas}}
Las condiciones periódicas son cuando nuestra frontera fantasma toma los valores de la frontera opuesta del lado que están, es decir, si un fenómeno físico pasa a través de la parte de arriba de nuestro dominio (ver Figura \ref{fig:periodicas}), este, saldrá por la parte de abajo y viceversa, lo mismo aplica para los fenómenos pasen por la parte izquierda o derecha de nuestro dominio.
 
\begin{figure}[H]
\centering
\subfigure[Onda antes de tocar la frontera de abajo]{\includegraphics[scale = 0.50]{./Figuras/Apendice/periodicas1.png}}
\subfigure[La onda pasa la frontera de abajo y sale por la parte de arriba]{\includegraphics[scale = 0.50]{./Figuras/Apendice/periodicas2.png}}
\caption{Al llegar la onda abajo se puede ver que se transporta al lado de arriba, se eligió esa posición de la onda, solo para resaltar la periodicidad de la onda, ya que si se hubiera puesto en el centro, la onda chocaría contra si mismo y no podríamos ver el paso de la onda.} \label{fig:periodicas}
\end{figure}

Las ecuación  es que representan este tipo de frontera son las siguientes:

\begin{eqnarray}
\textbf{U}(0,j)&=&\textbf{U}(n,j) \\
\textbf{U}(n+1,j)&=&\textbf{U}(1,j) \\
\textbf{U}(i,0)&=&\textbf{U}(i,m) \\
\textbf{U}(i,m+1)&=&\textbf{U}(i,1) 
\end{eqnarray}


\subsection{Condiciones de frontera \emph{Jet} }

Las condiciones de tipo jet, son en las que dado un lado de nuestra frontera (pueden se varios), se va a inyectar un energía y masa a una cierta velocidad constante todo el tiempo en una de las partes de la frontera.

 
\begin{figure}[H]
\centering
\subfigure[La densidad del medio sin la inyección del jet]{\includegraphics[scale = 0.50]{./Figuras/Apendice/jet1.png}}
\subfigure[Jet inyectándose en el medio]{\includegraphics[scale = 0.50]{./Figuras/Apendice/jet2.png}}
\caption{El jet, es básicamente inyectar masa y energía en una parte de nuestra frontera, a cada tiempo que evoluciona.} \label{fig:periodicas}
\end{figure}

Para obtener las ecuación es de esta frontera, igualamos los valores de la frontera \emph{fantasma} con las variables primitivas de nuestro jet, las siguientes ecuación son para la parte de abajo de nuestro dominio.

No relativista

\begin{eqnarray}
\textbf{U}(1,0,j)&=&\rho_{jet} \\
\textbf{U}(2,0,j)&=& \rho_{jet} v_{x_{jet}}\\
\textbf{U}(3,0,j)&=& \rho_{jet} v_{y_{jet}}\\
\textbf{U}(4,0,j)&=& E_{jet}
\end{eqnarray}

Relativista

\begin{eqnarray}
\textbf{U}(1,0,j)&=&\rho_{jet} \gamma \\
\textbf{U}(2,0,j)&=& \rho_{jet} v_{x_{jet}} \gamma^2 h \\
\textbf{U}(3,0,j)&=& \rho_{jet} v_{y_{jet}} \gamma^2 h \\
\textbf{U}(4,0,j)&=& \rho_{jet} \gamma^2 h-P
\end{eqnarray}

donde $j\in \left[ a,b \right]$ y $a\geq 0, \, b\leq n+1$. Los puntos de la frontera que no sean del jet se pueden combinar con los 3 tipos de frontera mencionados anteriormente.


\chapter{Subrutina de Newton-Rapson}\label{ap_newrap}

Subrutina de Newton-Rapson, los valores de entrada son $m^2$ y las conservadas y devuelve $W$
\begin{lstlisting} [frame=single]
  subroutine newrap(qu, w, m2)

!    use parameters, only : neq, gamma
    use globals , only : neq, gamma
    implicit none
    ! real, intent(in) :: qu(neq)
    real, intent(in)  :: m2
    real, intent(out) :: w
    real, parameter :: eps = 1d-10
    real :: a, b, c, mu, alpha, u2, lor, chi, dpdchi, dpdrho
    real :: w0, dpdw, f, dfdw, pg, dv2dw, dchidw, drhodw, qu(neq)
    integer :: k

    !print*, qu(4)**2, m2+qu(1)**2, qu(4)

    !if(qu(4)**2 .lt. m2+qu(1)**2 .or. qu(4).le. 0.0) then !.lt. -> '<'; .le. -> '<=='
    !  print*,'error in newrap'
    !  stop
    !end if

    a = 3.0;   b = 2.0 * (-qu(4));   c = m2
    if(b**2-a*c .lt. 0.0) then
      print*,'b**2-a*c<0'
      stop
    end if
    w = ( - b + sqrt(b**2-a*c) ) / a   ! initial guess for w = rho * h * lor**2

    w0 = w
    mu = 1.0
  100 continue
    do k = 1, 40

      alpha = m2 / w**2   ! alpha < 1 !

      u2  = alpha/(1.0-alpha)

      if(u2 .lt. 0.0) then
        print*,'u2<0cc'
        print*,qu
        stop
      end if

      lor = sqrt(1.0 + u2)

      chi = (w - qu(1)*(1.0+u2/(lor+1.0)))/(1.0+u2)

      ! ideal gas case        
      pg     = (gamma - 1.0)/gamma * chi
      dpdchi = (gamma - 1.0)/gamma
      dpdrho = 0.0

      f = w - pg - qu(4)  ! f(w) = 0

      if(abs(f) .lt. eps) return

      dv2dw  = lor/w**3*m2
      dchidw = 1.0/lor**2 + dv2dw*(qu(1)+2.0*lor*chi)

      drhodw = dv2dw*qu(1)

      dpdw = dpdchi*dchidw + dpdrho*drhodw

      dfdw   = 1.0 - dpdw    ! df/dw

      w  = w0 - mu * f / dfdw                       ! Newton-Rapson iteration

      if(abs(w-w0).lt.eps) return

      w0 = w

    end do

    if(mu .gt. 0.1) then
      mu = mu/2.0
      goto 100
    end if

  end subroutine newrap

\end{lstlisting}

\chapter{Subrutinas de Flujos calculados mediante el método de HLL}\label{ap_subrutina_fluxesHLL}

\begin{lstlisting} [frame=single]
  !     !Este modulo calcula los flujos hll
      use globals
      implicit none
!     real, intent(in) :: time
    
    real, intent(out)::fhll(neq,0:nx+1,0:ny+1), ghll(neq,0:nx+1,0:ny+1)! esta es la variable que se obtiene y va para la subrutina ulax

    real :: rho_l(0:nx+1,0:ny+1), rho_r(0:nx+1,0:ny+1),rho_u(0:nx+1,0:ny+1),rho_d(0:nx+1,0:ny+1)
    real :: vx_l(0:nx+1,0:ny+1), vx_r(0:nx+1,0:ny+1), vx_u(0:nx+1,0:ny+1), vx_d(0:nx+1,0:ny+1)
    real :: vy_l(0:nx+1,0:ny+1), vy_r(0:nx+1,0:ny+1), vy_u(0:nx+1,0:ny+1), vy_d(0:nx+1,0:ny+1)
    real :: P_l(0:nx+1, 0:ny+1), P_r(0:nx+1, 0:ny+1), P_u(0:nx+1,0:ny+1), P_d(0:nx+1,0:ny+1)

    real :: lor_l(0:nx+1, 0:ny+1), lor_r(0:nx+1, 0:ny+1), lor_u(0:nx+1, 0:ny+1), lor_d(0:nx+1, 0:ny+1)
    real :: h_l(0:nx+1,0:ny+1), h_r(0:nx+1,0:ny+1), h_u(0:nx+1,0:ny+1), h_d(0:nx+1,0:ny+1)

    real ::s_l(0:nx+1,0:ny+1),s_r(0:nx+1,0:ny+1), s_d(0:nx+1,0:ny+1), s_u(0:nx+1,0:ny+1)
    
    

    real :: f_l(neq, 0:nx+1,0:ny+1), f_r(neq,0:nx+1,0:ny+1), u_l(neq,0:nx+1,0:ny+1), u_r(neq,0:nx+1,0:ny+1)
    real :: g_d(neq, 0:nx+1,0:ny+1), g_u(neq,0:nx+1,0:ny+1), u_d(neq,0:nx+1,0:ny+1), u_u(neq,0:nx+1,0:ny+1)

    integer :: i,j

    call RL(nx,ny, neq, gamma, u, rho_l,rho_d, rho_r,rho_u, vx_l,vx_d, vx_r,vx_u, &
  vy_l,vy_r,vy_u, vy_d,P_l,P_d, P_r,P_u)

    call wavespeeds(s_l,s_d, s_r,s_u,rho_l,rho_d, rho_r,rho_u, vx_l,vx_d, vx_r,vx_u, &
  vy_l,vy_r,vy_u, vy_d,P_l,P_d, P_r,P_u )

    
!       !igual calculamos los flujos y conservadas de lado derecho e izquierdo

  if(choose_rel==0)then
  
      do i=1,nx
        do j=1,ny
          u_l(1,i,j) = rho_l(i,j)
          u_l(2,i,j) = rho_l(i,j)*vx_l(i,j)
          u_l(3,i,j) = rho_l(i,j)*vy_l(i,j)
          u_l(4,i,j) = 0.5*rho_l(i,j)*(vx_l(i,j)**2+vy_l(i,j)**2)+P_l(i,j)/(gamma-1.)

          u_d(1,i,j) = rho_d(i,j)
          u_d(2,i,j) = rho_d(i,j)*vx_d(i,j)
          u_d(3,i,j) = rho_d(i,j)*vy_d(i,j)
          u_d(4,i,j) = 0.5*rho_d(i,j)*(vx_d(i,j)**2+vy_d(i,j)**2)+P_d(i,j)/(gamma-1.)

          u_r(1,i,j) = rho_r(i,j)
          u_r(2,i,j) = rho_r(i,j)*vx_r(i,j)
          u_r(3,i,j) = rho_r(i,j)*vy_r(i,j)
          u_r(4,i,j) = 0.5*rho_r(i,j)*(vx_r(i,j)**2+vy_r(i,j)**2)+P_r(i,j)/(gamma-1.)

          u_u(1,i,j) = rho_u(i,j)
          u_u(2,i,j) = rho_u(i,j)*vx_u(i,j)
          u_u(3,i,j) = rho_u(i,j)*vy_u(i,j)
          u_u(4,i,j) = 0.5*rho_u(i,j)*(vx_u(i,j)**2+vy_u(i,j)**2)+P_u(i,j)/(gamma-1.)

!========================================================

          f_l(1,i,j) = rho_l(i,j)*vx_l(i,j)
          f_l(2,i,j) = rho_l(i,j)*vx_l(i,j)**2+P_l(i,j)
          f_l(3,i,j) = rho_l(i,j)*vx_l(i,j)*vy_l(i,j)
          f_l(4,i,j) = vx_l(i,j)*(u_l(4,i,j)+P_l(i,j))

          f_r(1,i,j) = rho_r(i,j)*vx_r(i,j)
          f_r(2,i,j) = rho_r(i,j)*vx_r(i,j)**2+P_r(i,j)
          f_r(3,i,j) = rho_r(i,j)*vx_r(i,j)*vy_r(i,j)
          f_r(4,i,j) = vx_r(i,j)*(u_r(4,i,j)+P_r(i,j))

          g_d(1,i,j) = rho_d(i,j)*vy_d(i,j)
          g_d(2,i,j) = rho_d(i,j)*vx_d(i,j)*vy_d(i,j)
          g_d(3,i,j) = rho_d(i,j)*vy_d(i,j)**2 + P_d(i,j)
          g_d(4,i,j) = vy_d(i,j)*(u_d(4,i,j)+P_d(i,j))

          g_u(1,i,j) = rho_u(i,j)*vy_u(i,j)
          g_u(2,i,j) = rho_u(i,j)*vx_u(i,j)*vy_u(i,j)
          g_u(3,i,j) = rho_u(i,j)*vy_u(i,j)**2 + P_u(i,j)
          g_u(4,i,j) = vy_u(i,j)*(u_u(4,i,j)+P_u(i,j))

        end do
      end do

  elseif(choose_rel==1)then

      do i=1,nx
        do j=1,ny

          lor_l(i,j) = 1/sqrt(1-(vx_l(i,j)**2+vy_l(i,j)**2))
          lor_d(i,j) = 1/sqrt(1-(vx_d(i,j)**2+vy_d(i,j)**2))
          lor_r(i,j) = 1/sqrt(1-(vx_r(i,j)**2+vy_r(i,j)**2))
          lor_u(i,j) = 1/sqrt(1-(vx_u(i,j)**2+vy_u(i,j)**2))

          h_l(i,j)=1.+gamma/(gamma-1.)*P_l(i,j)/rho_l(i,j)
          h_d(i,j)=1.+gamma/(gamma-1.)*P_d(i,j)/rho_d(i,j)
          h_r(i,j)=1.+gamma/(gamma-1.)*P_r(i,j)/rho_r(i,j)
          h_u(i,j)=1.+gamma/(gamma-1.)*P_u(i,j)/rho_u(i,j)

          u_l(1,i,j) = rho_l(i,j)*lor_l(i,j)
          u_l(2,i,j) = rho_l(i,j)*vx_l(i,j)*lor_l(i,j)**2*h_l(i,j)
          u_l(3,i,j) = rho_l(i,j)*vy_l(i,j)*lor_l(i,j)**2*h_l(i,j)
          u_l(4,i,j) = rho_l(i,j)*lor_l(i,j)**2*h_l(i,j)-P_l(i,j)

          u_d(1,i,j) = rho_d(i,j)*lor_d(i,j)
          u_d(2,i,j) = rho_d(i,j)*vx_d(i,j)*lor_d(i,j)**2*h_d(i,j)
          u_d(3,i,j) = rho_d(i,j)*vy_d(i,j)*lor_d(i,j)**2*h_d(i,j)
          u_d(4,i,j) = rho_d(i,j)*lor_d(i,j)**2*h_d(i,j)-P_d(i,j)

          u_r(1,i,j) = rho_r(i,j)*lor_r(i,j)
          u_r(2,i,j) = rho_r(i,j)*vx_r(i,j)*lor_r(i,j)**2*h_r(i,j)
          u_r(3,i,j) = rho_r(i,j)*vy_r(i,j)*lor_r(i,j)**2*h_r(i,j)
          u_r(4,i,j) = rho_r(i,j)*lor_r(i,j)**2*h_r(i,j)-P_r(i,j)

          u_u(1,i,j) = rho_u(i,j)*lor_u(i,j)
          u_u(2,i,j) = rho_u(i,j)*vx_u(i,j)*lor_u(i,j)**2*h_u(i,j)
          u_u(3,i,j) = rho_u(i,j)*vy_u(i,j)*lor_u(i,j)**2*h_u(i,j)
          u_u(4,i,j) = rho_u(i,j)*lor_u(i,j)**2*h_u(i,j)-P_u(i,j)

!==========================================================================

          f_l(1,i,j) = rho_l(i,j)*vx_l(i,j)*lor_l(i,j)
          f_l(2,i,j) = rho_l(i,j)*vx_l(i,j)**2*lor_l(i,j)**2*h_l(i,j)+P_l(i,j)
          f_l(3,i,j) = rho_l(i,j)*vx_l(i,j)*vy_l(i,j)*lor_l(i,j)**2*h_l(i,j)
          f_l(4,i,j) = rho_l(i,j)*vx_l(i,j)*lor_l(i,j)**2*h_l(i,j)

          f_r(1,i,j) = rho_r(i,j)*vx_r(i,j)*lor_r(i,j)
          f_r(2,i,j) = rho_r(i,j)*vx_r(i,j)**2*lor_r(i,j)**2*h_r(i,j)+P_r(i,j)
          f_r(3,i,j) = rho_r(i,j)*vx_r(i,j)*vy_r(i,j)*lor_r(i,j)**2*h_r(i,j)
          f_r(4,i,j) = rho_r(i,j)*vx_r(i,j)*lor_r(i,j)**2*h_r(i,j)

          g_d(1,i,j) = rho_d(i,j)*vy_d(i,j)*lor_d(i,j)
          g_d(2,i,j) = rho_d(i,j)*vx_d(i,j)*vy_d(i,j)*lor_d(i,j)**2*h_d(i,j)
          g_d(3,i,j) = rho_d(i,j)*vy_d(i,j)**2*lor_d(i,j)**2*h_d(i,j)+P_d(i,j)
          g_d(4,i,j) = rho_d(i,j)*vy_d(i,j)*lor_d(i,j)**2*h_d(i,j)

          g_u(1,i,j) = rho_u(i,j)*vy_u(i,j)*lor_u(i,j)
          g_u(2,i,j) = rho_u(i,j)*vx_u(i,j)*vy_u(i,j)*lor_u(i,j)**2*h_u(i,j)
          g_u(3,i,j) = rho_u(i,j)*vy_u(i,j)**2*lor_u(i,j)**2*h_u(i,j)+P_u(i,j)
          g_u(4,i,j) = rho_u(i,j)*vy_u(i,j)*lor_u(i,j)**2*h_u(i,j)

        end do
      end do

  endif

      do i=1,nx
        do j=1,ny
          
          if (0 .le. s_l(i,j)) then !less or equal 0<=sl
            fhll(:,i,j)=f_l(:, i,j)

          else if (s_l(i,j) .le. 0 .and. 0 .le. s_r(i,j)) then !sl<=0<=sr
        
            fhll(:,i,j)=(s_r(i,j)*f_l(:,i,j)-s_l(i,j)*f_r(:,i,j)+s_l(i,j)*s_r(i,j)*(u_r(:,i,j)-u_l(:,i,j)))/&
            (s_r(i,j)-s_l(i,j))

      
          else if (s_r(i,j) .le. 0) then !sr<=0
              fhll(:,i,j)=f_r(:,i,j)

          endif

          if(0 .le. s_d(i,j)) then
            ghll(:,i,j)= g_d(:,i,j)

          else if(s_d(i,j) .le. 0 .and. 0 .le. s_u(i,j)) then

            ghll(:,i,j)=(s_u(i,j)*g_d(:,i,j)-s_d(i,j)*g_u(:,i,j)+s_d(i,j)*s_u(i,j)*(u_u(:,i,j)-u_d(:,i,j)))/&
            (s_u(i,j)-s_d(i,j))

          else if(s_u(i,j) .le. 0) then
              ghll(:,i,j) = g_u(:,i,j)

          endif
        end do
      end do

  fhll(:,nx+1,:) = fhll(:,nx-1,:)
!   fhll(:,nx,:)   = fhll(:,nx-1,:)     
  fhll(:,0,:)    = fhll(:,2,:)
!   fhll(:,1,:)    = fhll(:,2,:)     
  fhll(:,:,ny+1) = fhll(:,:,ny-1)
!   fhll(:,:,ny)   = fhll(:,:,ny-1)
  fhll(:,:,0)    = fhll(:,:,2)
!   fhll(:,:,1)    = fhll(:,:,2)

  ghll(:,nx+1,:) = ghll(:,nx-1,:)
!   ghll(:,nx,:)   = ghll(:,nx-1,:)     
  ghll(:,0,:)    = ghll(:,2,:)
!   ghll(:,1,:)    = ghll(:,2,:)     
  ghll(:,:,ny+1) = ghll(:,:,ny-1)
!   ghll(:,:,ny)   = ghll(:,:,ny-1)
  ghll(:,:,0)    = ghll(:,:,2)
!   ghll(:,:,1)    = ghll(:,:,2)

  return


\end{lstlisting}

\chapter{Velocidad de lós métodos numéricos} \label{aped.D}

Esto está en construcción

\begin{thebibliography}{20}


\bibitem{Berger:2013jza} 
  E.~Berger,
  Short-Duration Gamma-Ray Bursts,
  Ann.\ Rev.\ Astron.\ Astrophys.\  {\bf 52}, 43 (2014)
  doi:10.1146/annurev-astro-081913-035926
  [arXiv:1311.2603 [astro-ph.HE]].
  %%CITATION = doi:10.1146/annurev-astro-081913-035926;%%
  %424 citations counted in INSPIRE as of 02 Jan 2019
 
%\bibitem{2012ApJ...760..122G} Gao, Y., \& Law, C.~K.\ 2012, \apj , 760, 122

\bibitem{JBEMGRB}
Hamidani, H., \& Ioka, K. (2020). Jet propagation in expanding medium for gamma-ray bursts. Monthly Notices of the Royal Astronomical Society, 500, 627-642.

\bibitem{Zhang:PGRB}
Zhang, B. (2018). GRB Phenomenology. In The Physics of Gamma-Ray Bursts (pp. 27-121). Cambridge: Cambridge University Press. doi:10.1017/9781139226530.004
% 2019-The Physics of Gamma-Ray Bursts.pdf

\bibitem{MB-HLLC-RSfRF}
A. Mignone, G. Bodo, An HLLC Riemann solver for relativistic flows – II. Magnetohydrodynamics, Monthly Notices of the Royal Astronomical Society, Volume 368, Issue 3, May 2006, Pages 1040–1054, https://doi.org/10.1111/j.1365-2966.2006.10162.x
%0601640.pdf

\bibitem{MB-HLLC-I}
A. Mignone, G. Bodo, An HLLC Riemann solver for relativistic flows — I. Hydrodynamics, Monthly Notices of the Royal Astronomical Society, Volume 364, Issue 1, November 2005, Pages 126–136, https://doi.org/10.1111/j.1365-2966.2005.09546.x

\bibitem{PAFD}
Clarke, C., \& Carswell, B. (2007). Blast waves. In Principles of Astrophysical Fluid Dynamics (pp. 89-106). Cambridge: Cambridge University Press. doi:10.1017/CBO9780511813450.009
%Cathie Clarke, Bob Carswell - Principles of astrophysical fluid dynamics (2007, Cambridge University Press).pdf

\bibitem{Decade-sgrb}
Fong, W. et al. “A DECADE OF SHORT-DURATION GAMMA-RAY BURST BROADBAND AFTERGLOWS: ENERGETICS, CIRCUMBURST DENSITIES, AND JET OPENING ANGLES.” The Astrophysical Journal 815.2 (2015): 102. Crossref. Web.
%1509.02922.pdf

\bibitem{Ecasgrb}
Teboul, O \& Piran, T. (2017). Emission of cocoon afterglow for short Gamma Ray Burst : a counterpart of gravitational waves?. 

\bibitem{ESRM}
Mignone, A., and Jonathan C. McKinney. “Equation of State in Relativistic Magnetohydrodynamics: Variable Versus Constant Adiabatic Index.” Monthly Notices of the Royal Astronomical Society 378.3 (2007): 1118–1130. Crossref. Web.
%mignone2007.pdf

\bibitem{GRB:CAP}
Azzam, W.J. \& Zitouni, Hannachi \& Guessoum, Nidhal. (2017). Gamma-Ray Bursts: Characteristics and Prospects. Journal of Physics: Conference Series. 869. 012065. 10.1088/1742-6596/869/1/012065. 
%Azzam_2017_J._Phys.__Conf._Ser._869_012065.pdf

\bibitem{GRB:PPP}
ZHANG, BING, and PETER MÉSZÁROS. “GAMMA-RAY BURSTS: PROGRESS, PROBLEMS \& PROSPECTS.” International Journal of Modern Physics A 19.15 (2004): 2385–2472. Crossref. Web.
%zhang2004.pdf

\bibitem{PGRB-piran}
Piran, Tsvi. “The Physics of Gamma-Ray Bursts.” Reviews of Modern Physics 76.4 (2005): 1143–1210. Crossref. Web.
%piran2005.pdf

\bibitem{PSGRB}
Lee, William H, and Enrico Ramirez-Ruiz. “The Progenitors of Short Gamma-Ray Bursts.” New Journal of Physics 9.1 (2007): 17–17. Crossref. Web.
%lee2007.pdf

\bibitem{PGRB-RJ}
Kumar, Pawan, and Bing Zhang. “The Physics of Gamma-Ray Bursts \& Relativistic Jets.” Physics Reports 561 (2015): 1–109. Crossref. Web.
%kumar2015.pdf

\bibitem{GRB-SE}
Gehrels, N., E. Ramirez-Ruiz, and D.B. Fox. “Gamma-Ray Bursts in theSwiftEra.” Annual Review of Astronomy and Astrophysics 47.1 (2009): 567–617. Crossref. Web.
%gehrels2009.pdf

\bibitem{PNCCA}
Mignone, A. et al. “PLUTO: A Numerical Code for Computational Astrophysics.” The Astrophysical Journal Supplement Series 170.1 (2007): 228–242. Crossref. Web.
%Mignone_2007_ApJS_170_228.pdf

\bibitem{GRB-levan}
Levan, Andrew et al. “Gamma-Ray Burst Progenitors.” Space Science Reviews 202.1-4 (2016): 33–78. Crossref. Web.
%Levan2016_Article_Gamma-RayBurstProgenitors.pdf

\bibitem{Diego}
Lazzati, Davide et al. “Off-Axis Prompt X-Ray Transients from the Cocoon of Short Gamma-Ray Bursts.” The Astrophysical Journal 848.1 (2017): L6. Crossref. Web.
%Lazzati_2017_ApJL_848_L6.pdf

\bibitem{RHRRC}
Gao, Yang, and Chung K. Law. “RANKINE-HUGONIOT RELATIONS IN RELATIVISTIC COMBUSTION WAVES.” The Astrophysical Journal 760.2 (2012): 122. Crossref. Web.
%Gao_2012_ApJ_760_122.pdf

\bibitem{1973ApJ...179..897T} Thorne, K.~S.\ 1973.\ Relativistic Shocks: the Taub Adiabat.\ The Astrophysical Journal 179, 897.
%1973ApJ...179..897T.pdf

\bibitem{Toro}
Toro E.F. (1997) The HLL and HLLC Riemann Solvers. In: Riemann Solvers and Numerical Methods for Fluid Dynamics. Springer, Berlin, Heidelberg

\bibitem{SGRBr-Avanzo}
D'Avanzo, P.. (2015). Short gamma ray bursts: A review. Journal of High Energy Astrophysics. 7. 10.1016/j.jheap.2015.07.002. 

\bibitem{Exact-solution_riemman-solver}
F.D. Lora-Clavijo \emph(et. al.) (2013). Exact Solution oh the 1D riemann problem in Newtonian and relativistic hydrodynamics. Revista Mexicana de Física. E 59 (2013)28-50 

\bibitem{SURJ-I} %A Simulation Study of Ultra-Relativistic Jets - I. A New Code for Relativistic Hydrodynamics
Jeongbhin Seo et al 2021 ApJ 920 143
\end{thebibliography}



\end{document}